% !TeX spellcheck = pl_PL
\chapter{Założenia projektowe}\label{ch:design-assumptions}
\section{Uwagi wstępne}\label{sec:presumptions}

\lipsum[5]

\section{Słownik pojęć domenowych}\label{sec:dictionary}

Na podstawie rozważań z~rozdziału \ref{ch:knowladge-state} sporządzono następującą listę definicji domenowych istotną z~punktu widzenia projektu:
\begin{itemize}[series=atr, wide, align=left, leftmargin=190pt]
    \atr{BIA}- metoda impedancji bioelektrycznej wykorzystywana do analizy składu ciała
    \atr{BMI}- wskaźnik masy ciała
    \atr{CPM}- całkowita przemiana materii
\end{itemize}

\section{Sformułowanie problemu}\label{sec:problem-specification}

\par
W~tabeli \ref{tabela:sformulowanie-problemu} przedstawiono sformułowanie rozważanego w~pracy problemu wraz z~jego wpływem i~propozycją pomyślnego rozwiązania.

\begin{minipage}{\textwidth}
    \begin{table}[H]
        \centering\caption{Sformułowanie problemu \source{\ownwork}\label{tabela:sformulowanie-problemu}}
        \begin{tabular}{|P{.2\textwidth}|p{.7\textwidth}|}

            \hline
            \cellgray{Problem} &
            \inlinetodo{todo} \\
            \hline

            \cellgray{Dotyczy} &
            \inlinetodo{todo} \\
            \hline

            \cellgray{Wpływ problemu} &
            \begin{itemize}
                \item \inlinetodo{todo}
                \item \inlinetodo{todo}
                \item \inlinetodo{todo}
            \end{itemize} \\
            \hline

            \cellgray{Pomyślne rozwiązanie} &
            \begin{itemize}
                \item \inlinetodo{todo}
                \item \inlinetodo{todo}
                \item \inlinetodo{todo}
            \end{itemize} \\
            \hline
        \end{tabular}
    \end{table}
\end{minipage}

\pagebreak
\section{Pozycjonowanie produktu}\label{sec:product-positioning}

\par
W~tabeli \ref{tabela:pozycjonowanie-produktu} przedstawiono pozycjonowanie opracowywanego produktu względem rynku produktów z dziedziny.

\begin{minipage}{\textwidth}
    \begin{table}[H]
        \centering\caption{Pozycjonowanie produktu \source{\ownwork}\label{tabela:pozycjonowanie-produktu}}
        \begin{tabular}{|P{.2\textwidth}|p{.7\textwidth}|}

            \hline
            \cellgray{Dla} &
            \inlinetodo{todo} \\
            \hline

            \cellgray{Który} &
            \inlinetodo{todo} \\
            \hline

            \cellgray{Nazwa produktu} &
            \inlinetodo{todo} \\
            \hline

            \cellgray{Który} &
            \inlinetodo{todo} \\
            \hline

            \cellgray{Inaczej niż} &
            \inlinetodo{todo} \\
            \hline

            \cellgray{Nasz produkt} &
            \inlinetodo{todo} \\
            \hline
        \end{tabular}
    \end{table}
\end{minipage}

\pagebreak
\section{Podsumowanie użytkowników systemu}\label{sec:users-summary}
\par
W~tabeli \ref{tabela:uzytkownicy} przedstawiono podsumowanie użytkowników projektowanego systemu, ich krótki opis oraz ich podstawowe odpowiedzialności związane z~korzystaniem z~systemu.

\begin{minipage}{\textwidth}
    \begin{table}[H]
        \centering\caption{Użytkownicy \source{\ownwork}\label{tabela:uzytkownicy}}
        \begin{tabular}{|P{.15\textwidth}|P{.25\textwidth}|P{.5\textwidth}|}

            \hline
            \cellgray{Nazwa} & \cellgray{Opis} & \cellgray{Odpowiedzialności}\\

            \hline
            Gość &
            Niezalogowany użytkownik &
            \begin{itemize}
                \item Zakłada konto użytkownika.
                \item Wyświetla stronę główną.
            \end{itemize} \\
            \hline
            Administrator &
            Osoba zarządzająca działaniem aplikacji &
            \begin{itemize}
                \item Przydzielanie i~odbieranie użytkownikom uprawnień.
                \item Zarządzanie definicjami wartości odżywczych, typami diet, typami posiłków, typami dań i~wyposażeniem kuchennym.
            \end{itemize} \\
            \hline
        \end{tabular}
    \end{table}
\end{minipage}

\section{Wymagania funkcjonalne}\label{sec:functional-requirements}
\par
W~tabeli \ref{tabela:wymaganiaFunkcjonalne}
przedstawiono wymagania funkcjonalne dla systemu w~postaci zestawienia potrzeb użytkowników systemu z~cechami związanymi z~realizacją danej potrzeby.
Następnie wymagania sformalizowano w~postaci diagramów przypadków użycia języka UML na rysunku \ref{fig:use-case-diagram:gateway}.

\begin{minipage}{\textwidth}
    \begin{table}[H]
        \centering\caption{Wymagania funkcjonalne \source{\ownwork}\label{tabela:wymaganiaFunkcjonalne}}
        \begin{tabular}{|P{.3\textwidth}|P{.6\textwidth}|}
            \hline
            \cellgray{Potrzeby} & \cellgray{Cechy} \\

            \hline
            Administrator potrzebuje widzieć listę użytkowników &
            \begin{itemize}
                \item Przydzielanie i~odbieranie użytkownikom uprawnień.
            \end{itemize} \\
            \hline
            Użytkownik potrzebuje korzystać ze swojego konta &
            \begin{itemize}
                \item Logowanie do systemu.
                \item Przypomnienie hasła.
                \item Zarządzanie swoimi danymi osobowymi.
            \end{itemize} \\
            \hline
            Użytkownik chce przeglądać witrynę w~swoim języku &
            \begin{itemize}
                \item Obsługa witryny w~wielu językach.
            \end{itemize} \\
            \hline
            Gość potrzebuje korzystać z~systemu &
            \begin{itemize}
                \item Zakładanie konta użytkownika.
                \item Wyrażenie zgody na przetwarzanie swoich danych osobowych.
            \end{itemize} \\
            \hline
        \end{tabular}
    \end{table}
\end{minipage}

%\image{0.7}{img/gateway.png}{Diagram przypadków użycia}{use-case-diagram:gateway}

\newpage

\section{Wymagania niefunkcjonalne}\label{sec:nonfunctional-requirements}
\begin{itemize}
    \item System działa poprawnie w~przeglądarkach Google~Chrome~76, Mozilla~Firefox~69 i~Opera~63.
    \item System działa na urządzenia mobilnych korzystających z~systemu Android~9 i~iOS~12.
    \item System jest dostępny w~polskiej i~angielskiej wersji językowej.
    \item System ma czytelny i~minimalistyczny interfejs.
    \item Aplikacja webowa jest w~pełni responsywna i~wygodna do używania na ekranach od~5 do~30 cali.
\end{itemize}
\thispagestyle{normal}

