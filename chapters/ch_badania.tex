% !TeX spellcheck = pl_PL
\chapter{Optymalizacja procesu rekrutacji w branży IT z wykorzystaniem uczenia maszynowego}\label{ch:analysis}
\section{Metody badawcze}\label{sec:analysis-method}
\todo{co, kogo i dlaczego}
\todo{uzasadnienie wykorzystania uczenia maszynowego do analizy danych}
\section{Wybór źródła danych: Prezentacja ankiety StackOverflow}\label{sec:analysis:data-source-selection}
\todo{przedstawienie źródła danych, charakterystyka respondentów}
\section{Wstępna selekcja cech}\label{sec:analysis:feature-pre-selection}
\todo{z wyszczególnieniem cech stałych/środowiskowych (kraj, typ dewelopera, typ firmy, rozmiar firmy) i zmiennych (wynagrodzenie, benefity, metodologie, cechy profilu kandydata)}
\section{Wstępne przetworzenie danych}\label{sec:analysis:preprocessing}
\todo{kodowanie liczbowe, oczyszczanie}
\todo{opcjonalnie wzbogacenie o wybrane indeksy rozwoju społecznego}
\section{Selekcja cech z wykorzystaniem algorytmu XGB}\label{sec:analysis:feature-selection-xgb}
\todo{budowa modelu uczenia maszynowego w oparciu o cechy istotnie wpływające na predykcję}
\todo{\url{https://medium.com/@s.pranav.harathi/stack-overflow-survey-analysis-ed45127691b}}
\section{Prezentacja wyników}\label{sec:analysis:important-features}
\todo{}
\section{Wnioski i analiza możliwości praktycznego zastosowania zbudowanego modelu predykcji}\label{sec:analysis:model-fitness}
\todo{Analiza skuteczności (dopasowania) modelu}

\thispagestyle{normal}
