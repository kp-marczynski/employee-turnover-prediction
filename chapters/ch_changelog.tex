% !TeX spellcheck = pl_PL
\pagenumbering{gobble}
\addtocontents{toc}{\protect\setcounter{tocdepth}{-1}}
\chapter*{Changelog}\label{ch:changelog}

\textcolor{red}{
Jest to pomocniczy rozdział opisujący zmiany w~kolejnych wersjach pracy wysyłanej do promotora, żeby ułatwić współpracę z~promotorem.
Zostanie usunięty przed ostatecznym oddaniem pracy.
}
\subsubsection{v1}
\begin{itemize}
%    \item Uwzględniono sugestie z~komentarzy promotora:
%    \begin{itemize}
%    \end{itemize}
%    \item Zmiany:
%    \begin{itemize}
%    \end{itemize}
    \item Nowości:
    \begin{itemize}
        \item Demo szablonu pracy
        \item Propozycja rozdziałów i układu pracy
        \item Formatowanie tabel, obrazków
        \item Spisy literatury, tabel, rysunków i kodu
        \item strona tytułowa
        \item oświadczenia na końcu pracy
    \end{itemize}
%    \item Podsumowanie:
%    \begin{itemize}
%    \end{itemize}
    \item Do zrobienia:
    \begin{itemize}
        \item rozdział 1 do 21.01.2022
        \item rozdział 2 do 31.03.2022
        \item rozdział 3 do 15.05.2022
%        \item Podział bibliografii na kategorie:
%        \begin{itemize}
%            \item \url{https://tex.stackexchange.com/questions/340369/formatting-the-layout-of-splitbib-category-headings}
%            \item \url{https://tex.stackexchange.com/questions/54940/is-it-possible-to-split-the-bibliography-into-two-different-parts-using-bibtex-t}
%        \end{itemize}
%        \item Dodać oświadczenie na początku pracy
%        \item Stylowanie tabeli: pogróbione nagłówki, podpisy nad tabelą, źródła pod tabelą
%        \item cytaty blokowe
    \end{itemize}
\end{itemize}

\cleardoublepage
\pagenumbering{gobble}
%\thispagestyle{normal}
