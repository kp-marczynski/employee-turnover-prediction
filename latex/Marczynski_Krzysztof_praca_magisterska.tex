% !TEX encoding = UTF-8 Unicode 
% !TeX spellcheck = pl_PL
% http://wiki.languagetool.org/checking-la-tex-with-languagetool
\documentclass{dyplom}
\usepackage[utf8]{inputenc}
\usepackage{splitbib}

\makeatletter
\def\SBmisctitle{Wykaz innych źródeł}
\def\NMSB@titlestyle{simple}
\def\NMSB@stylesimple#1#2{\if@nobreak\else\bigskip\fi\textbf{\large #1#2}}
\makeatother
\begin{category}{Spis literatury}
    \SBentries{philips-edwards-2009,wozniak-2016,cron-2006,pocztowski-2009,pocztowski-2007,spychala-2019,
        ,krol-ludwiczynski-2006,taylor-2006,dalton-1982,wozniak-2012,steel-2009,kozlowski-2012}
\end{category}
\usepackage{bibentry}
\usepackage{hyperref}
%\usepackage{natbib}
%\setcitestyle{authoryear,open={[},close={]}}
%
\usepackage{lipsum}

% Dane o pracy
\author{Krzysztof Marczyński}
\title{Analiza~czynników~wpływających na~fluktuację~pracowników~IT z~wykorzystaniem~uczenia~maszynowego}
\titlen{Analysis of factors related to employee rotation in IT enterprises}
\promotor{dr hab. Joanna Radomska, prof. UEW}
\katedrapromotora{Zarządzania Strategicznego}
%\konsultant{dr hab. inż. Kazimerz Kabacki}
\wydzial{Wydział Zarządzania}
\kierunek{Zarządzanie}
\album{181154}
\krotkiestreszczenie{W pracy przedstawiono analizę czynników związanych z fluktuacją pracowników w przedsiębiorstwach IT}
\slowakluczowe{fluktuacja pracowników, selekcja pracowników}

\linespread{1.3}
\nobibliography*

%todo na koniec sprawdzić czy nie zostały sieroty/wdowy na końcach stron/wierszy
%todo na koniec zastąpić regexem: " ([a-z]) " na " $1~" i~" - " na "~- "
\begin{document}
%    \floatsetup[listing]{capposition=top}
    \sloppy %for words not to leak from right side of container
    % !TeX spellcheck = pl_PL
\pagenumbering{gobble}
\addtocontents{toc}{\protect\setcounter{tocdepth}{-1}}
\chapter*{Changelog}\label{ch:changelog}

Jest to pomocniczy rozdział opisujący zmiany w~kolejnych wersjach pracy wysyłanej do promotora, żeby ułatwić współpracę z~promotorem.
Zostanie usunięty przed ostatecznym oddaniem pracy.

\subsubsection{v1}
\begin{itemize}
%    \item Uwzględniono sugestie z~komentarzy promotora:
%    \begin{itemize}
%    \end{itemize}
%    \item Zmiany:
%    \begin{itemize}
%    \end{itemize}
    \item Nowości:
    \begin{itemize}
        \item Demo szablonu pracy
        \item Propozycja rozdziałów i układu pracy
    \end{itemize}
%    \item Podsumowanie:
%    \begin{itemize}
%    \end{itemize}
    \item Do zrobienia:
    \begin{itemize}
        \item rozdziały 1-3
%        \item Podział bibliografii na kategorie:
%        \begin{itemize}
%            \item \url{https://tex.stackexchange.com/questions/340369/formatting-the-layout-of-splitbib-category-headings}
%            \item \url{https://tex.stackexchange.com/questions/54940/is-it-possible-to-split-the-bibliography-into-two-different-parts-using-bibtex-t}
%        \end{itemize}
%        \item Dodać oświadczenie na początku pracy
%        \item Stylowanie tabeli: pogróbione nagłówki, podpisy nad tabelą, źródła pod tabelą
%        \item cytaty blokowe
    \end{itemize}
\end{itemize}

\cleardoublepage
\pagenumbering{gobble}
%\thispagestyle{normal}
 %todo usunąć przed oddaniem

    \maketitle
    \pagenumbering{gobble}
    % !TeX spellcheck = pl_PL
% --- Strona ze streszczeniem i~abstraktem ------------------------------------------------------------------
\addtocontents{toc}{\protect\setcounter{tocdepth}{-1}}
\chapter*{Streszczenie} % po polsku
Celem pracy było sprawdzenie, czy uczenie maszynowe pozwala określić z zadowalającym stopniem
pewności cechy wpływające na poziom zadowolenia i chęć zmiany pracy pracowników IT.
Aby osiągnąć ten cel przeprowadzono przegląd literatury w zakresie problematyki fluktuacji i zadowolenia pracowników
ze szczególnym zwróceniem uwagi na charakterystyki specyficzne dla branży informatycznej.
Następnie wybrano źródło danych pozwalające przeprowadzić badania ilościowe dla społeczności pracowników zatrudnionych w branży informatycznej
i przygotowano algorytm w języku Python pozwalający na zbudowanie modeli predykcji opartych o regresję i klasyfikację.
W celu oceny skuteczności modelu predykcji wybrano zestawy odpowiednich metryk zarówno dla regresji, jak i klasyfikacji.
Rezultaty badań mogą stanowić sugestię dla pracowników działów rekrutacyjnych i employer branding,
jakie kwestie organizacyjne w firmie mogą ulec usprawnieniu, aby zwiększyć retencję i zadowolenie pracowników.

% Kilka sztuczek, żeby:
%~- Abstract pojawił się na tej samej stronie co Streszczenie
%~- Abstract nie pojawił się w~spisie treści
\addtocontents{toc}{\protect\setcounter{tocdepth}{-1}}
\begingroup
\renewcommand{\cleardoublepage}{}
\renewcommand{\clearpage}{}
\chapter*{Abstract} % ...i to samo po angielsku

The aim of this work was to check whether machine learning allows to determine with a satisfactory degree of certainty the features influencing the level of satisfaction and willingness to change the job of IT employees.
To achieve this goal, a literature review was carried out in the field of employee fluctuation and satisfaction, paying particular attention to the characteristics specific to the IT industry.
Then, a data source was selected to conduct quantitative research for the community of employees working in the IT industry,
and an algorithm in the Python language was prepared to build prediction models based on regression and classification.
In order to assess the effectiveness of the prediction models, sets of appropriate metrics were selected for both regression and classification.
The results of the research may be a suggestion for employees of recruitment and employer branding departments
which organizational issues in the company can be improved in order to increase retention and satisfaction of the employees.

\endgroup
\addtocontents{toc}{\protect\setcounter{tocdepth}{2}}
% --- Koniec strony ze streszczeniem i~abstraktem -----------------------------------------------------------

    \cleardoublepage

    \pdfbookmark{\contentsname}{toc}
    \tableofcontents
    \cleardoublepage

    \pagenumbering{arabic}
    \setlength{\parskip}{6pt}%
    % !TeX spellcheck = pl_PL

\chapter*{Wstęp}\label{ch:admission}

\section*{Opis problemu}\label{sec:admission:problem-description}
\todo{Wprowadzenie do specyfiki rynku pracy IT}
\todo{Problem selekcji kandydatów}
\todo{Problem rotacji pracowników w firmie}

\section*{Cel pracy}\label{sec:admission:thesis-goal}

\todo{}

\section*{Zakres pracy}\label{sec:admission:scope-of-work}

\todo{}

\section*{Struktura pracy}\label{sec:admission:thesis-structure}

\thispagestyle{normal}


%    % !TeX~program = latexmk
% !TeX spellcheck = pl_PL
% !TeX~root = example.tex

\chapter{LaTeX demo}

\textcolor{red}{
    Niniejszy rozdział zawiera prezentację możliwości przygotowanego szablonu pracy
    i zostanie usunięty we właściwej pracy.
}

ĄĆĘŁŃÓŚŹŻ ąćęłńóśźż\footnote{Przykład użycia polskich znaków diakrytycznych oraz przypisu w~miejscu}.

Reszta dokumentacji znajduje się w\nolink{docker_compose_reference}.

Jak pisze Harel w\nolink{harel_rzecz_2008}: \lipsum[1]

Natomiast Kaleta uważał\nolink{kaleta_experimental_2005} \lipsum[1]

Jak widać na rys. \ref{fig:network} Docker ma wewnętrzną sieć.

\imagewidth[\bibentry{docker_compose_reference}\nolink{docker_compose_reference}]{0.6}{demo/img/swarm-network}{Docker ma sieć}{network}

Jak widać na rys.\ref{fig:kotek} Ala ma kota. \lipsum[1]

\imagewidth{0.4}{demo/img/kotek}{Ala ma kota}{kotek}

Co uwzględniono w~tabeli \ref{tabela:coktoma}. \lipsum[1]

\noindent\begin{minipage}{\textwidth}
    \begin{table}[H]
        \raggedright\caption{Co kto ma (patrz też dodatek~\ref{app:dod1}) \label{tabela:coktoma}}
        \begin{center}\begin{tabular}{|l|l|l|}% wyrównanie kolumn tabeli -> l~c r~- do lewej, środka, do prawej
                          \hline
                          Ala & ma & kota \\
                          \hline
                          Ola & ma & psa \\
                          \hline
                          Ula & ma & małpę\\
                          \hline
        \end{tabular}\end{center}
        \raggedright\source{\bibentry{harel_rzecz_2008}\nolink{harel_rzecz_2008}}
        \vspace{0.75cm}
    \end{table}
\end{minipage}

W~moim kodzie \ref{listing:liquibase} zrobiłem coś wspaniałego. \lipsum[1]
% lub {java} albo {bash} albo {text}
%\noindent\begin{minipage}{\textwidth}
%\begin{listing}[float=h!]
%    \caption{Przykładowy algorytm w~języku C} \label{listing:moj}
%    \begin{minted}{c}
%        int main()
%        {
%        int a=2*3;
%        printf("**Ala ma kota\n**");
%        while(!I2C_CheckEvent(I2C1, I2C_EVENT_MASTER_MODE_SELECT)); /* EV5 */
%        return 0;
%        }
%    \end{minted}
%    \raggedright\source{\ownwork}
%    \vspace{0.75cm}
%\end{listing}

%\end{minipage}
\noindent\begin{minipage}{\textwidth}
    \begin{lstlisting}[caption={Skrpyt ładujący dane z~pliku CSV}, label={listing:liquibase}]
    \end{lstlisting}
    \hspace{.075\textwidth}\begin{minipage}{.85\textwidth}
    \begin{minted}{xml}
<changeSet id="201902130001" author="kmarczynski" context="usda">
  <loadData file="config/liquibase/usda_sr_db/household_measure.csv"
    quotchar='"' separator=";" tableName="household_measure">
    <column header="product_id" name="product_id" type="numeric"/>
    <column header="measure_description" name="description" type="string"/>
    <column header="grams_weight" name="grams_weight" type="numeric"/>
    <column header="is_visible" name="is_visible" type="boolean"/>
  </loadData>
</changeSet>
    \end{minted}
    \end{minipage}

    \raggedright\source{\ownwork}
    \vspace{0.75cm}
\end{minipage}

\lipsum[1]

W tabelach \ref{tabela:rozwiazania-konkurencyjne-funkcjonalne}, \ref{tabela:sformulowanie-problemu}, \ref{tabela:uzytkownicy} przedstawiono przykładowe formatowanie tabel.

\noindent\begin{minipage}{\textwidth}
    \begin{table}[H]
        \raggedright\caption{Rozwiązania konkurencyjne~- cechy funkcjonalne\label{tabela:rozwiazania-konkurencyjne-funkcjonalne}}
        \begin{center}\begin{tabular}{|P{.22\textwidth}|P{.09\textwidth}|P{.09\textwidth}|P{.09\textwidth}|P{.09\textwidth}|P{.09\textwidth}|P{.09\textwidth}|}
            \hline
            & \cellgray{Rozw1}      & \cellgray{Rozw2}          & \cellgray{Rozw3}         & \cellgray{Rozw4}           & \cellgray{Rozw5}      & \cellgray{Rozw6}    \\ \hline
            \cellgray{Funkcjonalność 1}        & \cellgreen{TAK}       & \cellgreen{TAK}           & \cellgreen{TAK}          & \cellgreen{TAK}            & \cellgreen{TAK}       & \cellgreen{TAK}     \\ \hline
            \cellgray{Funkcjonalność 2}        & \cellgreen{TAK}       & \cellgreen{TAK}           & \cellgreen{TAK}          & \cellgreen{TAK}            & \cellred{NIE}         & \cellred{NIE}       \\ \hline
            \cellgray{Funkcjonalność 3}        & \cellgreen{TAK}       & \cellgreen{TAK}           & \cellgreen{TAK}          & \cellgreen{TAK}            & \cellgreen{TAK}       & \cellgreen{TAK}     \\ \hline
            \cellgray{Funkcjonalność 4}        & \cellgreen{TAK}       & \cellgreen{TAK}           & \cellgreen{TAK}          & \cellgreen{TAK}            & \cellred{NIE}         & \cellgreen{TAK}     \\ \hline
            \cellgray{Funkcjonalność 5}        & \cellgreen{TAK}       & \cellgreen{TAK}           & \cellgreen{TAK}          & \cellred{NIE}              & \cellred{NIE}         & \cellgreen{TAK}     \\ \hline
            \cellgray{Funkcjonalność 6}        & \cellgreen{TAK}       & \cellgreen{TAK}           & \cellgreen{TAK}          & \cellgreen{TAK}            & \cellgreen{TAK}       & \cellgreen{TAK}     \\ \hline
            \cellgray{Funkcjonalność 7}        & \cellgreen{TAK}       & \cellgreen{TAK}           & \cellgreen{TAK}          & \cellred{NIE}              & \cellgreen{TAK}       & \cellgreen{TAK}     \\ \hline
            \cellgray{Funkcjonalność 8}        & \cellgreen{TAK}       & \cellgreen{TAK}           & \cellgreen{TAK}          & \cellgreen{TAK}            & \cellgreen{TAK}       & \cellred{NIE}       \\ \hline
            \cellgray{Funkcjonalność 9}        & \cellgreen{TAK}       & \cellgreen{TAK}           & \cellgreen{TAK}          & \cellgreen{TAK}            & \cellgreen{TAK}       & \cellgreen{TAK}     \\ \hline
            \cellgray{Funkcjonalność 10}       & \cellgreen{TAK}       & \cellgreen{TAK}           & \cellgreen{TAK}          & \cellgreen{TAK}            & \cellgreen{TAK}       & \cellgreen{TAK}     \\ \hline
        \end{tabular}\end{center}
        \raggedright\source{\ownwork}
        \vspace{0.75cm}
    \end{table}
\end{minipage}

\noindent\begin{minipage}{\textwidth}
    \begin{table}[H]
        \raggedright\caption{Sformułowanie problemu\label{tabela:sformulowanie-problemu}}
        \begin{center}\begin{tabular}{|P{.2\textwidth}|p{.7\textwidth}|}

            \hline
            \cellgray{Problem} &
            \inlinetodo{todo} \\
            \hline

            \cellgray{Dotyczy} &
            \inlinetodo{todo} \\
            \hline

            \cellgray{Wpływ problemu} &
            \begin{itemize}
                \item \inlinetodo{todo}
                \item \inlinetodo{todo}
                \item \inlinetodo{todo}
            \end{itemize} \\
            \hline

            \cellgray{Pomyślne rozwiązanie} &
            \begin{itemize}
                \item \inlinetodo{todo}
                \item \inlinetodo{todo}
                \item \inlinetodo{todo}
            \end{itemize} \\
            \hline
        \end{tabular}\end{center}
        \raggedright\source{\ownwork}
        \vspace{0.75cm}
    \end{table}
\end{minipage}

\noindent\begin{minipage}{\textwidth}
    \begin{table}[H]
        \raggedright\caption{Użytkownicy\label{tabela:uzytkownicy}}
        \begin{center}\begin{tabular}{|P{.15\textwidth}|P{.25\textwidth}|P{.5\textwidth}|}

            \hline
            \cellgray{Nazwa} & \cellgray{Opis} & \cellgray{Odpowiedzialności}\\

            \hline
            Gość &
            Niezalogowany użytkownik &
            \begin{itemize}
                \item Zakłada konto użytkownika.
                \item Wyświetla stronę główną.
            \end{itemize} \\
            \hline
            Administrator &
            Osoba zarządzająca działaniem aplikacji &
            \begin{itemize}
                \item Przydzielanie i~odbieranie użytkownikom uprawnień.
                \item Zarządzanie definicjami wartości odżywczych, typami diet, typami posiłków, typami dań i~wyposażeniem kuchennym.
            \end{itemize} \\
            \hline
        \end{tabular}\end{center}
        \raggedright\source{\ownwork}
        \vspace{0.75cm}
    \end{table}
\end{minipage}

\thispagestyle{normal}


    % !TeX spellcheck = pl_PL
\chapter{Problem fluktuacji pracowników w literaturze}\label{ch:knowladge-state}
\section{Zjawisko fluktuacji}\label{sec:employee-turnover}
\todo{zjawisko fluktuacji + fluktuacja w zależności od branży}

\section{Zadowolenie pracowników}\label{sec:employee-happiness}
\todo{zadowolenie pracownikó}
\todo{wątek pandemii}
\todo{fluktuacja pracowników a praca zdalna}

\section{Rekrutacja pracowników a fluktuacja}\label{sec:employee-selection}
\todo{koszty fluktuacji w związku z rekrutacją}
\todo{dopasowanie osoby do stanowiska pracy a fluktuacja}

\thispagestyle{normal}

    % !TeX spellcheck = pl_PL
\chapter{Charakterystyka branży IT}\label{ch:it-sector}
\section{Wysoka fluktuacja}\label{sec:it-turnover}
\todo{niedobór programistów a wysoki popyt na ekspertów}

\section{Wpływ fluktuacji na utratę wiedzy w projektach informatycznych}\label{sec:it-knowledge-loss}
\todo{}

\section{Zarządzanie projektami informatycznymi a wysoka fluktuacja}\label{sec:it-project-management}
\todo{koszty fluktuacji w związku z rekrutacją}

\thispagestyle{normal}

    % !TeX spellcheck = pl_PL
\chapter{Badania}\label{ch:analysis}
\section{Wybór źródła danych: Prezentacja ankiety StackOverflow}\label{sec:analysis:data-source-selection}
\todo{}
\section{Porównanie zbioru pytań z kolejnych edycji ankiety}\label{sec:analysis:questions-comparision}
\todo{}
\section{Wstępna selekcja cech i edycji ankiety}\label{sec:analysis:feature-pre-selection}
\todo{z wyszczególnieniem cech stałych/środowiskowych (kraj, typ dewelopera, typ firmy, rozmiar firmy) i zmiennych (wynagrodzenie, benefity, metodologie, cechy profilu kandydata)}
\section{Wstępne przetworzenie danych}\label{sec:analysis:preprocessing}
\todo{kodowanie liczbowe, oczyszczanie, wzbogacenie}
\todo{\url{https://medium.com/@s.pranav.harathi/stack-overflow-survey-analysis-ed45127691b}}
\section{Selekcja cech z wykorzystaniem algorytmu XGB}\label{sec:analysis:feature-selection-xgb}
\todo{}
\section{Prezentacja cech istotnie wpływających na predykcję}\label{sec:analysis:important-features}
\todo{}
\section{Budowa modelu uczenia maszynowego}\label{sec:analysis:ml-model}
\todo{w oparciu o cechy istotnie wpływające na predykcję}
\section{Analiza dopasowania modelu}\label{sec:analysis:model-fitness}
\todo{Analiza skuteczności (dopasowania) modelu}

\thispagestyle{normal}


    % !TeX spellcheck = pl_PL
\chapter*{Zakończenie}\label{ch:ending}

\todo{zakończenie}

\thispagestyle{normal}


    % W~pracy pojawią się tylko prace naprawdę cytowane.
    % \nocite{*}
    \cleardoublepage
    \bibliography{literatura}
    \bibliographystyle{dyplom}

    \listnormal{\listoffigures}{\listfigurename}
    \listnormal{\listoftables}{\listtablename}

    \clearpage
%    \listof{listing}{Spis kodów źródłowych}
    \lstlistoflistings

    % !TeX spellcheck = pl_PL
% \appendixpage
% \addappheadtotoc

\appendix
\begin{appendices}
    \chapter{Lista pytań w ankiecie StackOverflow}\label{app:dod1}

    W sekcjach \ref{pytania-2017}, \ref{pytania-2018}, \ref{pytania-2019} przedstawiono przetłumaczoną listę pytań wykorzystanych w przeprowadzonych w ramach niniejszej pracy badaniach.
    Przedstawione pytania pochodzą z ankiety deweloperskiej StackOverflow \cite{so-survey-info}, pominięte jednak zostały pytania nieistotne z punktu widzenia prowadzonego badania.

    \section{Pytania z roku 2017}\label{pytania-2017}
    \begin{itemize}
        \item \textbf{CareerSatisfaction} - Czy jesteś zadowolony ze swojej kariery?
        \item \textbf{JobSatisfaction} - Czy jesteś zadowolony ze swojej pracy?
        \item \textbf{JobSeekingStatus} - Czy szukasz obecnie nowej pracy?
        \item \textbf{DeveloperType} - Jaki typ programisty najlepiej do ciebie pasuje?
        \item \textbf{YearsCodedJob} - Ile lat programujesz zawodowo?
        \item \textbf{Country} - W jakim kraju żyjesz?
        \item \textbf{FormalEducation} - Jaki jest twój najwyższy ukończony poziom edukacji formalnej?
        \item \textbf{EmploymentStatus} - Jaki jest twój status zatrudnienia?
        \item \textbf{MajorUndergrad} - W jakiej specjalizacji ukończyłeś edukację?
        \item \textbf{HaveWorkedLanguage} - W jakich językach programowania pisałeś kod w ciągu ostatniego roku?
        \item \textbf{Gender} - Z jaką płcią się identyfikujesz?
        \item \textbf{CompanySize} - Ilu pracowników zatrudnia firma w ktorej pracujesz?
        \item \textbf{HighestEducationParents} - Jaki jest najwyższy poziom edukacji osiągnięty przez któregokolwiek z twoich rodziców?
        \item \textbf{HaveWorkedFramework} - Z jakich narzędzi deweloperskich, bibliotek i frameworków korzystałeś w ciągu ostatniego roku?
        \item \textbf{HaveWorkedDatabase} - Z jakich baz danych korzystałeś w ciągu ostatniego roku?
        \item \textbf{HaveWorkedPlatform} - Na jakie platformy tworzyłeś oprogramowanie w ciągu ostatniego roku?
        \item \textbf{ProgramHobby} - Czy programujesz jako hobby lub kontrybuujesz kod do projektów typ Open Source?
        \item \textbf{University} - Czy jesteś obecnie studentem na wyższej uczelni?
        \item \textbf{YearsProgram} - Ile lat temu nauczyłeś się programować?
        \item \textbf{WebDeveloperType} - Jaki typ programisty aplikacji webowych najlepiej do ciebie pasuje?
        \item \textbf{MobileDeveloperType} - Jaki typ programisty aplikacji mobilnych najlepiej do ciebie pasuje?
        \item \textbf{ProblemSolving} - Czy lubisz rozwiązywać problemy?
        \item \textbf{BuildingThings} - Czy budowanie rzeczy daje ci satysfakcję?
        \item \textbf{LearningNewTech} - Czy nauka nowych rzeczy sprawia ci przyjemność?
        \item \textbf{BoringDetails} - Czy nudzą cię szczegóły implementacyjne?
        \item \textbf{JobSecurity} - Czy stabilność zatrudnienia jest dla ciebie ważna?
        \item \textbf{DiversityImportant} - Czy różnorodność (ang diversity) w miejscu pracy jest dla ciebie ważna?
        \item \textbf{AnnoyingUI} - Czy denerwuje cię jeśli aplikacja ma zły interfejs użytkownika?
        \item \textbf{FriendsDevelopers} - Czy większość twoich znajomych to programiści, naukowcy albo inżynierowie?
        \item \textbf{SeriousWork} - Czy poważnie podchodzisz do swojej pracy?
        \item \textbf{InvestTimeTools} - Czy inwestujesz dużo czasu w dostosowanie narzędzi których używasz?
        \item \textbf{WorkPayCare} - Czy nie przejmujesz się na czym polega twoja praca pod warunkiem, że dostajesz odpowiednie wynagrodzenie?
        \item \textbf{KinshipDevelopers} - Czy czujesz solidarność ze swoimi współpracownikami?
        \item \textbf{ChallengeMyself} - Czy lubisz stawiać sobie wyzwania?
        \item \textbf{CompetePeers} - Czy lubisz rywalizować ze swoimi współpracownikami?
        \item \textbf{ChangeWorld} - Czy chcesz zmienić świat?
        \item \textbf{LastNewJob} - Kiedy ostatni raz zmieniałeś pracę?
        \item \textbf{ImportantHiringAlgorithms} - Według ciebie jak ważna w procesie rekrutacji jest: znajomość algorytmów i struktur danych?
        \item \textbf{ImportantHiringTechExp} - "Congratulations! You've just been put in charge of technical recruiting at Globex, a multinational high- tech firm. This job comes with a corner office, and you have an experienced staff of recruiters at your disposal. They want to know what they should prioritize when recruiting software developers. How important should each of the following be in Globex's hiring process? Experience with specific tools (libraries, frameworks, etc.) used by the employer"
        \item \textbf{ImportantHiringCommunication} - "Congratulations! You've just been put in charge of technical recruiting at Globex, a multinational high- tech firm. This job comes with a corner office, and you have an experienced staff of recruiters at your disposal. They want to know what they should prioritize when recruiting software developers. How important should each of the following be in Globex's hiring process? Communication skills"
        \item \textbf{ImportantHiringOpenSource} - "Congratulations! You've just been put in charge of technical recruiting at Globex, a multinational high- tech firm. This job comes with a corner office, and you have an experienced staff of recruiters at your disposal. They want to know what they should prioritize when recruiting software developers. How important should each of the following be in Globex's hiring process? Contributions to open source projects"
        \item \textbf{ImportantHiringPMExp} - "Congratulations! You've just been put in charge of technical recruiting at Globex, a multinational high- tech firm. This job comes with a corner office, and you have an experienced staff of recruiters at your disposal. They want to know what they should prioritize when recruiting software developers. How important should each of the following be in Globex's hiring process? Experience with specific project management tools & techniques"
        \item \textbf{ImportantHiringCompanies} - "Congratulations! You've just been put in charge of technical recruiting at Globex, a multinational high- tech firm. This job comes with a corner office, and you have an experienced staff of recruiters at your disposal. They want to know what they should prioritize when recruiting software developers. How important should each of the following be in Globex's hiring process? Previous companies worked at"
        \item \textbf{ImportantHiringTitles} - "Congratulations! You've just been put in charge of technical recruiting at Globex, a multinational high- tech firm. This job comes with a corner office, and you have an experienced staff of recruiters at your disposal. They want to know what they should prioritize when recruiting software developers. How important should each of the following be in Globex's hiring process? Previous job titles held"
        \item \textbf{ImportantHiringEducation} - "Congratulations! You've just been put in charge of technical recruiting at Globex, a multinational high- tech firm. This job comes with a corner office, and you have an experienced staff of recruiters at your disposal. They want to know what they should prioritize when recruiting software developers. How important should each of the following be in Globex's hiring process? Educational credentials (e.g. schools attended, specific field of study, grades earned)"
        \item \textbf{ImportantHiringRep} - "Congratulations! You've just been put in charge of technical recruiting at Globex, a multinational high- tech firm. This job comes with a corner office, and you have an experienced staff of recruiters at your disposal. They want to know what they should prioritize when recruiting software developers. How important should each of the following be in Globex's hiring process? Stack Overflow reputation"
        \item \textbf{ImportantHiringGettingThingsDone} - "Congratulations! You've just been put in charge of technical recruiting at Globex, a multinational high- tech firm. This job comes with a corner office, and you have an experienced staff of recruiters at your disposal. They want to know what they should prioritize when recruiting software developers. How important should each of the following be in Globex's hiring process? Track record of getting things done"
        \item \textbf{Overpaid} - "Compared to your estimate of your own market value, do you think you are…?"
        \item \textbf{EducationImportant} - "Overall, how important has your formal schooling and education been to your career success?"
        \item \textbf{EducationTypes} - "Outside of your formal schooling and education, which of the following have you done?"
        \item \textbf{CousinEducation} - "Let's pretend you have a distant cousin. They are 24 years old, have a college degree in a field not related to computer programming, and have been working a non-coding job for the last two years. They want your advice on how to switch to a career as a software developer. Which of the following options would you most strongly recommend to your cousin?"
        \item \textbf{Methodology} - Which of the following methodologies do you have experience working in?
        \item \textbf{VersionControl} - "What version control system do you use? If you use several, please choose the one you use most often."
        \item \textbf{CheckInCode} - "Over the last year, how often have you checked-in or committed code?"
        \item \textbf{ShipIt} - It's better to ship now and optimize later
        \item \textbf{OtherPeoplesCode} - Maintaining other people's code is a form of torture
        \item \textbf{ProjectManagement} - Most project management techniques are useless
        \item \textbf{EnjoyDebugging} - I enjoy debugging code
        \item \textbf{InTheZone} - I often get “into the zone” when I'm coding
        \item \textbf{DifficultCommunication} - I have difficulty communicating my ideas to my peers
        \item \textbf{CollaborateRemote} - It's harder to collaborate with remote peers than those on site
        \item \textbf{AuditoryEnvironment} - "Suppose you're about to start a few hours of coding and have complete control over your auditory environment (music, background noise, etc.). What would you do?"
        \item \textbf{MetricAssess} - "Congratulations! The bosses at your new employer, E Corp, are allowing you to choose which metrics will be used to assess your individual performance in your role as a senior developer. Which metrics do you suggest to the E bosses?"
        \item \textbf{StackOverflowFoundAnswer} - "Over the last three months, approximately how often have you done each of the following on Stack Overflow? Found an answer that solved my coding problem"
        \item \textbf{StackOverflowCopiedCode} - "Over the last three months, approximately how often have you done each of the following on Stack Overflow? Copied a code example and pasted it into my codebase"
        \item \textbf{StackOverflowAnswer} - "Over the last three months, approximately how often have you done each of the following on Stack Overflow? Written a new answer to someone else's question"
        \item \textbf{StackOverflowMetaChat} - "Over the last three months, approximately how often have you done each of the following on Stack Overflow? Participated in community discussions on meta or in chat"
        \item \textbf{StackOverflowCommunity} - I feel like a member of the Stack Overflow community
        \item \textbf{StackOverflowHelpful} - The answers and code examples I get on Stack Overflow are helpful
        \item \textbf{StackOverflowBetter} - Stack Overflow makes the Internet a better place
        \item \textbf{StackOverflowWhatDo} - I don't know what I'd do without Stack Overflow
        \item \textbf{HomeRemote} - How often do you work from home or remotely?
        \item \textbf{CompanyType} - Which of the following best describes the type of company or organization you work for?
        \item \textbf{AssessJobIndustry} - "When you're assessing potential jobs to apply to, how important are each of the following to you? The industry that I'd be working in"
        \item \textbf{AssessJobRole} - "When you're assessing potential jobs to apply to, how important are each of the following to you? The specific role or job title I'd be applying for"
        \item \textbf{AssessJobExp} - "When you're assessing potential jobs to apply to, how important are each of the following to you? The experience level called for in the job description"
        \item \textbf{AssessJobDept} - "When you're assessing potential jobs to apply to, how important are each of the following to you? The specific department or team I'd be working on"
        \item \textbf{AssessJobTech} - "When you're assessing potential jobs to apply to, how important are each of the following to you? The languages, frameworks, and other technologies I'd be working with"
        \item \textbf{AssessJobProjects} - "When you're assessing potential jobs to apply to, how important are each of the following to you? How projects are managed at the company or organization"
        \item \textbf{AssessJobCompensation} - "When you're assessing potential jobs to apply to, how important are each of the following to you? The compensation and benefits offered"
        \item \textbf{AssessJobOffice} - "When you're assessing potential jobs to apply to, how important are each of the following to you? The office environment I'd be working in"
        \item \textbf{AssessJobCommute} - "When you're assessing potential jobs to apply to, how important are each of the following to you? The amount of time I'd have to spend commuting"
        \item \textbf{AssessJobRemote} - "When you're assessing potential jobs to apply to, how important are each of the following to you? The opportunity to work from home/remotely"
        \item \textbf{AssessJobLeaders} - "When you're assessing potential jobs to apply to, how important are each of the following to you? The reputations of the company's senior leaders"
        \item \textbf{AssessJobProfDevel} - "When you're assessing potential jobs to apply to, how important are each of the following to you? Opportunities for professional development"
        \item \textbf{AssessJobDiversity} - "When you're assessing potential jobs to apply to, how important are each of the following to you? The diversity of the company or organization"
        \item \textbf{AssessJobProduct} - "When you're assessing potential jobs to apply to, how important are each of the following to you? How widely used or impactful the product or service I'd be working on is"
        \item \textbf{AssessJobFinances} - "When you're assessing potential jobs to apply to, how important are each of the following to you? The financial performance or funding status of the company or organization"
        \item \textbf{ImportantBenefits} - "When it comes to compensation and benefits, other than base salary, which of the following are most important to you?"
        \item \textbf{WantWorkLanguage} - "Which of the following languages have you done extensive development work in over the past year, and which do you want to work in over the next year?"
        \item \textbf{WantWorkFramework} - "Which of the following libraries, frameworks, and tools have you done extensive development work in over the past year, and which do you want to work in over the next year?"
        \item \textbf{WantWorkDatabase} - "Which of the following database technologies have you done extensive development work in over the past year, and which do you want to work in over the next year?"
        \item \textbf{WantWorkPlatform} - "Which of the following platforms have you done extensive development work for over the past year, and which do you want to work on over the next year?"
        \item \textbf{EquipmentSatisfiedMonitors} - "Thinking about your main coding workstation, how satisfied are you with each of the following? Monitors/screens (number of, size, resolution)"
        \item \textbf{EquipmentSatisfiedCPU} - "Thinking about your main coding workstation, how satisfied are you with each of the following? Processing power (CPU and/or GPU)"
        \item \textbf{EquipmentSatisfiedRAM} - "Thinking about your main coding workstation, how satisfied are you with each of the following? Amount of RAM"
        \item \textbf{EquipmentSatisfiedStorage} - "Thinking about your main coding workstation, how satisfied are you with each of the following? Storage capacity"
        \item \textbf{EquipmentSatisfiedRW} - "Thinking about your main coding workstation, how satisfied are you with each of the following? Storage read/write speed"
        \item \textbf{InfluenceInternet} - How much influence do you have on purchasing decisions within your organization for each of the following? Internet bandwidth
        \item \textbf{InfluenceWorkstation} - How much influence do you have on purchasing decisions within your organization for each of the following? Your personal workstation hardware
        \item \textbf{InfluenceHardware} - How much influence do you have on purchasing decisions within your organization for each of the following? Personal workstation hardware for others in the company
        \item \textbf{InfluenceServers} - How much influence do you have on purchasing decisions within your organization for each of the following? Servers
        \item \textbf{InfluenceTechStack} - How much influence do you have on purchasing decisions within your organization for each of the following? Main technical stack of the company
        \item \textbf{InfluenceDeptTech} - How much influence do you have on purchasing decisions within your organization for each of the following? Technical stack used in your department
        \item \textbf{InfluenceVizTools} - "How much influence do you have on purchasing decisions within your organization for each of the following? Data analysis and visualization tools (e.g. Tableau, Looker)"
        \item \textbf{InfluenceDatabase} - How much influence do you have on purchasing decisions within your organization for each of the following? Database systems or solutions
        \item \textbf{InfluenceCloud} - How much influence do you have on purchasing decisions within your organization for each of the following? Cloud or serverless back-end solutions
        \item \textbf{InfluenceConsultants} - How much influence do you have on purchasing decisions within your organization for each of the following? Consultants
        \item \textbf{InfluenceRecruitment} - How much influence do you have on purchasing decisions within your organization for each of the following? Recruitment tools & platforms
        \item \textbf{InfluenceCommunication} - How much influence do you have on purchasing decisions within your organization for each of the following? Communication & collaboration tools
        \item \textbf{TimeAfterBootcamp} - You indicated previously that you went through a developer “bootcamp.” How long did it take you to get a full-time job as a developer after graduating?
        \item \textbf{StackOverflowSatisfaction} - Stack Overflow satisfaction
        \item \textbf{StackOverflowJobListing} - "Over the last three months, approximately how often have you done each of the following on Stack Overflow? Seen a job listing I was interested in"
        \item \textbf{StackOverflowCompanyPage} - "Over the last three months, approximately how often have you done each of the following on Stack Overflow? Researched a potential employer by visiting its company page"
        \item \textbf{StackOverflowNewQuestion} - "Over the last three months, approximately how often have you done each of the following on Stack Overflow? Asked a new question"
        \item \textbf{Race} - Which of the following do you identify as?
        \item \textbf{Salary} - "What is your current annual base salary, before taxes, and excluding bonuses, grants, or other compensation?"
        \item \textbf{Professional} - Which of the following best describes you?
    \end{itemize}


    \section{Pytania z roku 2018}\label{pytania-2018}

    \begin{itemize}
        \item \textbf{JobSatisfaction} - "How satisfied are you with your current job? If you work more than one job, please answer regarding the one you spend the most hours on."
        \item \textbf{CareerSatisfaction} - "Overall, how satisfied are you with your career thus far?"
        \item \textbf{JobSearchStatus} - Which of the following best describes your current job-seeking status?
        \item \textbf{ConvertedSalary} - "Salary converted to annual USD salaries using the exchange rate on 2018-01-18, assuming 12 working months and 50 working weeks."
        \item \textbf{DevType} - Which of the following describe you? Please select all that apply.
        \item \textbf{YearsCodingProf} - For how many years have you coded professionally (as a part of your work)?
        \item \textbf{Country} - In which country do you currently reside?
        \item \textbf{FormalEducation} - Which of the following best describes the highest level of formal education that you’ve completed?
        \item \textbf{Employment} - Which of the following best describes your current employment status?
        \item \textbf{UndergradMajor} - You previously indicated that you went to a college or university. Which of the following best describes your main field of study (aka 'major')
        \item \textbf{LanguageWorkedWith} - "Which of the following programming, scripting, and markup languages have you done extensive development work in over the past year, and which do you want to work in over the next year?  (If you both worked with the language and want to continue to do so, please check both boxes in that row.)"
        \item \textbf{Gender} - "Which of the following do you currently identify as? Please select all that apply. If you prefer not to answer, you may leave this question blank."
        \item \textbf{CompanySize} - Approximately how many people are employed by the company or organization you work for?
        \item \textbf{EducationParents} - "What is the highest level of education received by either of your parents? If you prefer not to answer, you may leave this question blank."
        \item \textbf{Age} - "What is your age? If you prefer not to answer, you may leave this question blank."
        \item \textbf{DatabaseWorkedWith} - "Which of the following database environments have you done extensive development work in over the past year, and which do you want to work in over the next year?   (If you both worked with the database and want to continue to do so, please check both boxes in that row.)"
        \item \textbf{PlatformWorkedWith} - "Which of the following platforms have you done extensive development work for over the past year?   (If you both developed for the platform and want to continue to do so, please check both boxes in that row.)"
        \item \textbf{FrameworkWorkedWith} - "Which of the following libraries, frameworks, and tools have you done extensive development work in over the past year, and which do you want to work in over the next year?"
        \item \textbf{Hobby} - Do you code as a hobby?
        \item \textbf{OpenSource} - Do you contribute to open source projects?
        \item \textbf{Student} - "Are you currently enrolled in a formal, degree-granting college or university program?"
        \item \textbf{YearsCoding} - "Including any education, for how many years have you been coding?"
        \item \textbf{HopeFiveYears} - Which of the following best describes what you hope to be doing in five years?
        \item \textbf{LastNewJob} - When was the last time that you took a job with a new employer?
        \item \textbf{EducationTypes} - Which of the following types of non-degree education have you used or participated in? Please select all that apply.
        \item \textbf{AgreeDisagree1} - To what extent do you agree or disagree with each of the following statements? I feel a sense of kinship or connection to other developers
        \item \textbf{AgreeDisagree2} - To what extent do you agree or disagree with each of the following statements? I think of myself as competing with my peers
        \item \textbf{AgreeDisagree3} - To what extent do you agree or disagree with each of the following statements? I'm not as good at programming as most of my peers
        \item \textbf{OperatingSystem} - What is the primary operating system in which you work?
        \item \textbf{NumberMonitors} - How many monitors are set up at your workstation?
        \item \textbf{Methodology} - Which of the following methodologies do you have experience working in?
        \item \textbf{VersionControl} - What version control systems do you use regularly? Please select all that apply.
        \item \textbf{CheckInCode} - "Over the last year, how often have you checked-in or committed code?"
        \item \textbf{EthicsChoice} - Imagine that you were asked to write code for a purpose or product that you consider extremely unethical. Do you write the code anyway?
        \item \textbf{EthicsReport} - Do you report or otherwise call out the unethical code in question?
        \item \textbf{EthicsResponsible} - Who do you believe is ultimately most responsible for code that accomplishes something unethical?
        \item \textbf{EthicalImplications} - Do you believe that you have an obligation to consider the ethical implications of the code that you write?
        \item \textbf{StackOverflowRecommend} - How likely is it that you would recommend Stack Overflow overall to a friend or colleague? Where 0 is not likely at all and 10 is very likely.
        \item \textbf{StackOverflowVisit} - How frequently would you say you visit Stack Overflow?
        \item \textbf{StackOverflowConsiderMember} - Do you consider yourself a member of the Stack Overflow community?
        \item \textbf{WakeTime} - "On days when you work, what time do you typically wake up?"
        \item \textbf{HoursComputer} - "On a typical day, how much time do you spend on a desktop or laptop computer?"
        \item \textbf{HoursOutside} - "On a typical day, how much time do you spend outside?"
        \item \textbf{SkipMeals} - "In a typical week, how many times do you skip a meal in order to be more productive?"
        \item \textbf{ErgonomicDevices} - What ergonomic furniture or devices do you use on a regular basis? Please select all that apply.
        \item \textbf{Exercise} - "In a typical week, how many times do you exercise?"
        \item \textbf{Dependents} - "Do you have any children or other dependents that you care for? If you prefer not to answer, you may leave this question blank."
        \item \textbf{LanguageDesireNextYear} - "Which of the following programming, scripting, and markup languages have you done extensive development work in over the past year, and which do you want to work in over the next year?  (If you both worked with the language and want to continue to do so, please check both boxes in that row.)"
        \item \textbf{DatabaseDesireNextYear} - "Which of the following database environments have you done extensive development work in over the past year, and which do you want to work in over the next year?   (If you both worked with the database and want to continue to do so, please check both boxes in that row.)"
        \item \textbf{PlatformDesireNextYear} - "Which of the following platforms have you done extensive development work for over the past year?   (If you both developed for the platform and want to continue to do so, please check both boxes in that row.)"
        \item \textbf{FrameworkDesireNextYear} - "Which of the following libraries, frameworks, and tools have you done extensive development work in over the past year, and which do you want to work in over the next year?"
        \item \textbf{AssessJob1} - "Imagine that you are assessing a potential job opportunity. Please rank the following aspects of the job opportunity in order of importance (by dragging the choices up and down), where 1 is the most important and 10 is the least important. The industry that I'd be working in"
        \item \textbf{AssessJob2} - "Imagine that you are assessing a potential job opportunity. Please rank the following aspects of the job opportunity in order of importance (by dragging the choices up and down), where 1 is the most important and 10 is the least important. The financial performance or funding status of the company or organization"
        \item \textbf{AssessJob3} - "Imagine that you are assessing a potential job opportunity. Please rank the following aspects of the job opportunity in order of importance (by dragging the choices up and down), where 1 is the most important and 10 is the least important. The specific department or team I'd be working on"
        \item \textbf{AssessJob4} - "Imagine that you are assessing a potential job opportunity. Please rank the following aspects of the job opportunity in order of importance (by dragging the choices up and down), where 1 is the most important and 10 is the least important. The languages, frameworks, and other technologies I'd be working with"
        \item \textbf{AssessJob5} - "Imagine that you are assessing a potential job opportunity. Please rank the following aspects of the job opportunity in order of importance (by dragging the choices up and down), where 1 is the most important and 10 is the least important. The compensation and benefits offered"
        \item \textbf{AssessJob6} - "Imagine that you are assessing a potential job opportunity. Please rank the following aspects of the job opportunity in order of importance (by dragging the choices up and down), where 1 is the most important and 10 is the least important. The office environment or company culture"
        \item \textbf{AssessJob7} - "Imagine that you are assessing a potential job opportunity. Please rank the following aspects of the job opportunity in order of importance (by dragging the choices up and down), where 1 is the most important and 10 is the least important. The opportunity to work from home/remotely"
        \item \textbf{AssessJob8} - "Imagine that you are assessing a potential job opportunity. Please rank the following aspects of the job opportunity in order of importance (by dragging the choices up and down), where 1 is the most important and 10 is the least important. Opportunities for professional development"
        \item \textbf{AssessJob9} - "Imagine that you are assessing a potential job opportunity. Please rank the following aspects of the job opportunity in order of importance (by dragging the choices up and down), where 1 is the most important and 10 is the least important. The diversity of the company or organization"
        \item \textbf{AssessJob10} - "Imagine that you are assessing a potential job opportunity. Please rank the following aspects of the job opportunity in order of importance (by dragging the choices up and down), where 1 is the most important and 10 is the least important. How widely used or impactful the product or service I'd be working on is"
        \item \textbf{AssessBenefits1} - "Now, imagine you are assessing a job's benefits package. Please rank the following aspects of a job's benefits package from most to least important to you (by dragging the choices up and down), where 1 is most important and 11 is least important. Salary and/or bonuses"
        \item \textbf{AssessBenefits2} - "Now, imagine you are assessing a job's benefits package. Please rank the following aspects of a job's benefits package from most to least important to you (by dragging the choices up and down), where 1 is most important and 11 is least important. Stock options or shares"
        \item \textbf{AssessBenefits3} - "Now, imagine you are assessing a job's benefits package. Please rank the following aspects of a job's benefits package from most to least important to you (by dragging the choices up and down), where 1 is most important and 11 is least important. Health insurance"
        \item \textbf{AssessBenefits4} - "Now, imagine you are assessing a job's benefits package. Please rank the following aspects of a job's benefits package from most to least important to you (by dragging the choices up and down), where 1 is most important and 11 is least important. Parental leave"
        \item \textbf{AssessBenefits5} - "Now, imagine you are assessing a job's benefits package. Please rank the following aspects of a job's benefits package from most to least important to you (by dragging the choices up and down), where 1 is most important and 11 is least important. Fitness or wellness benefit (ex. gym membership, nutritionist)"
        \item \textbf{AssessBenefits6} - "Now, imagine you are assessing a job's benefits package. Please rank the following aspects of a job's benefits package from most to least important to you (by dragging the choices up and down), where 1 is most important and 11 is least important. Retirement or pension savings matching"
        \item \textbf{AssessBenefits7} - "Now, imagine you are assessing a job's benefits package. Please rank the following aspects of a job's benefits package from most to least important to you (by dragging the choices up and down), where 1 is most important and 11 is least important. Company-provided meals or snacks"
        \item \textbf{AssessBenefits8} - "Now, imagine you are assessing a job's benefits package. Please rank the following aspects of a job's benefits package from most to least important to you (by dragging the choices up and down), where 1 is most important and 11 is least important. Computer/office equipment allowance"
        \item \textbf{AssessBenefits9} - "Now, imagine you are assessing a job's benefits package. Please rank the following aspects of a job's benefits package from most to least important to you (by dragging the choices up and down), where 1 is most important and 11 is least important. Childcare benefit"
        \item \textbf{AssessBenefits10} - "Now, imagine you are assessing a job's benefits package. Please rank the following aspects of a job's benefits package from most to least important to you (by dragging the choices up and down), where 1 is most important and 11 is least important. Transportation benefit (ex. company-provided transportation, public transit allowance)"
        \item \textbf{AssessBenefits11} - "Now, imagine you are assessing a job's benefits package. Please rank the following aspects of a job's benefits package from most to least important to you (by dragging the choices up and down), where 1 is most important and 11 is least important. Conference or education budget"
        \item \textbf{TimeFullyProductive} - "Suppose a new developer with four years of experience, including direct experience working with your company's main technical stack, joined your team tomorrow. All other things being equal, how long would you expect it to take before they were fully productive and contributing at a typical level to your main code base?"
        \item \textbf{Salary} - "What is your current gross salary (before taxes and deductions), in ${q://QID50/ChoiceGroup/SelectedChoicesTextEntry}? Please enter a whole number in the box below, without any punctuation. If you are paid hourly, please estimate an equivalent weekly, monthly, or yearly salary. If you prefer not to answer, please leave the box empty."
        \item \textbf{StackOverflowHasAccount} - Do you have a Stack Overflow account?
        \item \textbf{StackOverflowParticipate} - "How frequently would you say you participate in Q&A on Stack Overflow? By participate we mean ask, answer, vote for, or comment on questions."
        \item \textbf{StackOverflowJobs} - Have you ever used or visited Stack Overflow Jobs?
    \end{itemize}


    \section{Pytania z roku 2019}\label{pytania-2019}
    \begin{itemize}
        \item \textbf{CareerSat} - "Overall, how satisfied are you with your career thus far?"
        \item \textbf{JobSat} - "How satisfied are you with your current job? (If you work multiple jobs, answer for the one you spend the most hours on.)"
        \item \textbf{JobSeek} - Which of the following best describes your current job-seeking status?
        \item \textbf{ConvertedComp} - "Salary converted to annual USD salaries using the exchange rate on 2019-02-01, assuming 12 working months and 50 working weeks."
        \item \textbf{DevType} - Which of the following describe you? Please select all that apply.
        \item \textbf{YearsCodePro} - How many years have you coded professionally (as a part of your work)?
        \item \textbf{Country} - In which country do you currently reside?
        \item \textbf{EdLevel} - Which of the following best describes the highest level of formal education that you’ve completed?
        \item \textbf{Employment} - Which of the following best describes your current employment status?
        \item \textbf{UndergradMajor} - What was your main or most important field of study?
        \item \textbf{LanguageWorkedWith} - "Which of the following programming, scripting, and markup languages have you done extensive development work in over the past year, and which do you want to work in over the next year?  (If you both worked with the language and want to continue to do so, please check both boxes in that row.)"
        \item \textbf{Gender} - "Which of the following do you currently identify as? Please select all that apply. If you prefer not to answer, you may leave this question blank."
        \item \textbf{OrgSize} - Approximately how many people are employed by the company or organization you work for?
        \item \textbf{Age} - "What is your age (in years)? If you prefer not to answer, you may leave this question blank."
        \item \textbf{DatabaseWorkedWith} - "Which of the following database environments have you done extensive development work in over the past year, and which do you want to work in over the next year?   (If you both worked with the database and want to continue to do so, please check both boxes in that row.)"
        \item \textbf{PlatformWorkedWith} - "Which of the following platforms have you done extensive development work for over the past year?   (If you both developed for the platform and want to continue to do so, please check both boxes in that row.)"
        \item \textbf{WebFrameWorkedWith} - "Which of the following web frameworks have you done extensive development work in over the past year, and which do you want to work in over the next year? (If you both worked with the framework and want to continue to do so, please check both boxes in that row.)"
        \item \textbf{MiscTechWorkedWith} - "Which of the following other frameworks, libraries, and tools have you done extensive development work in over the past year, and which do you want to work in over the next year? (If you both worked with the technology and want to continue to do so, please check both boxes in that row.)"
        \item \textbf{DevEnviron} - Which development environment(s) do you use regularly? Please check all that apply.
        \item \textbf{MainBranch} - "Which of the following options best describes you today? Here, by ""developer"" we mean ""someone who writes code."""
        \item \textbf{Hobbyist} - Do you code as a hobby?
        \item \textbf{OpenSourcer} - How often do you contribute to open source?
        \item \textbf{OpenSource} - How do you feel about the quality of open source software (OSS)?
        \item \textbf{Student} - "Are you currently enrolled in a formal, degree-granting college or university program?"
        \item \textbf{EduOther} - Which of the following types of non-degree education have you used or participated in? Please select all that apply.
        \item \textbf{YearsCode} - "Including any education, how many years have you been coding?"
        \item \textbf{Age1stCode} - "At what age did you write your first line of code or program? (E.g., webpage, Hello World, Scratch project)"
        \item \textbf{MgrIdiot} - How confident are you that your manager knows what they’re doing?
        \item \textbf{MgrMoney} - Do you believe that you need to be a manager to make more money?
        \item \textbf{MgrWant} - Do you want to become a manager yourself in the future?
        \item \textbf{LastHireDate} - When was the last time that you took a job with a new employer?
        \item \textbf{ImpSyn} - "For the specific work you do, and the years of experience you have, how do you rate your own level of competence?"
        \item \textbf{OpSys} - What is the primary operating system in which you work?
        \item \textbf{BetterLife} - Do you think people born today will have a better life than their parents?
        \item \textbf{Extraversion} - Do you prefer online chat or IRL conversations?
        \item \textbf{SOVisitFreq} - How frequently would you say you visit Stack Overflow?
        \item \textbf{SOVisitTo} - I visit Stack Overflow to... (check all that apply)
        \item \textbf{SOFindAnswer} - "On average, how many times a week do you find (and use) an answer on Stack Overflow?"
        \item \textbf{SOTimeSaved} - "Think back to the last time you solved a coding problem using Stack Overflow, as well as the last time you solved a problem using a different resource. Which was faster?"
        \item \textbf{SOComm} - Do you consider yourself a member of the Stack Overflow community?
        \item \textbf{Trans} - Do you identify as transgender?
        \item \textbf{Dependents} - "Do you have any dependents (e.g., children, elders, or others) that you care for?"
        \item \textbf{JobFactors} - "Imagine that you are deciding between two job offers with the same compensation, benefits, and location. Of the following factors, which 3 are MOST important to you?"
        \item \textbf{WorkWeekHrs} - "On average, how many hours per week do you work?"
        \item \textbf{WorkPlan} - How structured or planned is your work?
        \item \textbf{WorkChallenge} - "Of these options, what are your greatest challenges to productivity as a developer? Select up to 3:"
        \item \textbf{WorkRemote} - How often do you work remotely?
        \item \textbf{WorkLoc} - Where would you prefer to work?
        \item \textbf{CodeRev} - Do you review code as part of your work?
        \item \textbf{CodeRevHrs} - "On average, how many hours per week do you spend on code review?"
        \item \textbf{UnitTests} - Does your company regularly employ unit tests in the development of their products?
        \item \textbf{PurchaseHow} - "How does your company make decisions about purchasing new technology (cloud, AI, IoT, databases)?"
        \item \textbf{PurchaseWhat} - "What level of influence do you, personally, have over new technology purchases at your organization?"
        \item \textbf{LanguageDesireNextYear} - "Which of the following programming, scripting, and markup languages have you done extensive development work in over the past year, and which do you want to work in over the next year?  (If you both worked with the language and want to continue to do so, please check both boxes in that row.)"
        \item \textbf{DatabaseDesireNextYear} - "Which of the following database environments have you done extensive development work in over the past year, and which do you want to work in over the next year?   (If you both worked with the database and want to continue to do so, please check both boxes in that row.)"
        \item \textbf{PlatformDesireNextYear} - "Which of the following platforms have you done extensive development work for over the past year?   (If you both developed for the platform and want to continue to do so, please check both boxes in that row.)"
        \item \textbf{WebFrameDesireNextYear} - "Which of the following web frameworks have you done extensive development work in over the past year, and which do you want to work in over the next year? (If you both worked with the framework and want to continue to do so, please check both boxes in that row.)"
        \item \textbf{MiscTechDesireNextYear} - "Which of the following other frameworks, libraries, and tools have you done extensive development work in over the past year, and which do you want to work in over the next year? (If you both worked with the technology and want to continue to do so, please check both boxes in that row.)"
        \item \textbf{SOHowMuchTime} - "About how much time did you save? If you're not sure, please use your best estimate."
        \item \textbf{SOPartFreq} - "How frequently would you say you participate in Q&A on Stack Overflow? By participate we mean ask, answer, vote for, or comment on questions."
    \end{itemize}
\end{appendices}
\thispagestyle{normal}


    \cleardoublepage
    \pagenumbering{gobble}
    % !TeX spellcheck = pl_PL
\begin{center}
    \large\textbf{\MakeUppercase{OŚWIADCZENIE AUTORA PRACY}}\\
    \vspace{1cm}
\end{center}

Świadom odpowiedzialności prawnej oświadczam, że niniejsza praca dyplomowa została napisana przeze mnie samodzielnie. Wszystkie dane, istotne myśli i~sformułowania pochodzące z~literatury (przytoczone dosłownie lub niedosłownie) są opatrzone odpowiednimi odsyłaczami.
\par
Praca ta w~całości ani w~części, która zawierałaby znaczne fragmenty przedstawione w~pracy jako oryginalne, nie była wcześniej przedmiotem procedur związanych z~uzyskaniem tytułu zawodowego w~wyższej uczelni.
\par
Oświadczam, że tekst pracy dyplomowej wgrany do systemu APD jest identyczny z~tekstem wydrukowanym złożonym w~dziekanacie, o~ile złożenie pracy w~dziekanacie jest wymagane aktualnymi regulacjami Uczelni.
\par
\vspace{1cm}
\noindent UWAGA: Oświadczenie składane w~wersji elektronicznej w~systemie APD

\begin{center}
    \vspace{1cm}
    \large\textbf{\MakeUppercase{OŚWIADCZENIE PROMOTORA}}\\
    \vspace{1cm}
\end{center}

Oświadczam, że niniejsza praca dyplomowa została przygotowana pod moim kierunkiem i~spełnia warunki do przedstawienia jej w~postępowaniu o~nadanie tytułu zawodowego.
\par
Jednocześnie oświadczam, że tematyka pracy jest zgodna z~efektami uczenia się określonymi dla kierunku Autora pracy.
\par
\vspace{1cm}
\noindent UWAGA: Oświadczenie składane w~wersji elektronicznej w~systemie APD

\thispagestyle{normal}


\end{document}
