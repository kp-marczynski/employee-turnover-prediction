% !TEX encoding = UTF-8 Unicode 
% !TeX spellcheck = pl_PL
% http://wiki.languagetool.org/checking-la-tex-with-languagetool
\documentclass{dyplom}
\usepackage[utf8]{inputenc}
\usepackage{splitbib}

\makeatletter
\def\SBmisctitle{Wykaz innych źródeł}
\def\NMSB@titlestyle{simple}
\def\NMSB@stylesimple#1#2{\if@nobreak\else\bigskip\fi\textbf{\large #1#2}}
\makeatother
\begin{category}{Spis literatury}
    \SBentries{philips-edwards-2009,wozniak-2016,cron-2006,pocztowski-2009,pocztowski-2007,spychala-2019,
        ,krol-ludwiczynski-2006,taylor-2006,dalton-1982,wozniak-2012,steel-2009,kozlowski-2012}
\end{category}
\usepackage{bibentry}
\usepackage{hyperref}
%\usepackage{natbib}
%\setcitestyle{authoryear,open={[},close={]}}
%
\usepackage{lipsum}

% Dane o pracy
\author{Krzysztof Marczyński}
\title{Analiza~czynników~wpływających na~fluktuację~pracowników~IT z~wykorzystaniem~uczenia~maszynowego}
\titlen{Analysis of factors related to employee rotation in IT enterprises}
\promotor{dr hab. Joanna Radomska, prof. UEW}
\katedrapromotora{Zarządzania Strategicznego}
%\konsultant{dr hab. inż. Kazimerz Kabacki}
\wydzial{Wydział Zarządzania}
\kierunek{Zarządzanie}
\album{181154}
\krotkiestreszczenie{W pracy przedstawiono analizę czynników związanych z fluktuacją pracowników w przedsiębiorstwach IT}
\slowakluczowe{fluktuacja pracowników, selekcja pracowników}

\linespread{1.3}
\nobibliography*

%todo na koniec sprawdzić czy nie zostały sieroty/wdowy na końcach stron/wierszy
%todo na koniec zastąpić regexem: " ([a-z]) " na " $1~" i~" - " na "~- "
\begin{document}
%    \floatsetup[listing]{capposition=top}
    \sloppy %for words not to leak from right side of container
    % !TeX spellcheck = pl_PL
\pagenumbering{gobble}
\addtocontents{toc}{\protect\setcounter{tocdepth}{-1}}
\chapter*{Changelog}\label{ch:changelog}

Jest to pomocniczy rozdział opisujący zmiany w~kolejnych wersjach pracy wysyłanej do promotora, żeby ułatwić współpracę z~promotorem.
Zostanie usunięty przed ostatecznym oddaniem pracy.

\subsubsection{v1}
\begin{itemize}
%    \item Uwzględniono sugestie z~komentarzy promotora:
%    \begin{itemize}
%    \end{itemize}
%    \item Zmiany:
%    \begin{itemize}
%    \end{itemize}
    \item Nowości:
    \begin{itemize}
        \item Demo szablonu pracy
        \item Propozycja rozdziałów i układu pracy
    \end{itemize}
%    \item Podsumowanie:
%    \begin{itemize}
%    \end{itemize}
    \item Do zrobienia:
    \begin{itemize}
        \item rozdziały 1-3
%        \item Podział bibliografii na kategorie:
%        \begin{itemize}
%            \item \url{https://tex.stackexchange.com/questions/340369/formatting-the-layout-of-splitbib-category-headings}
%            \item \url{https://tex.stackexchange.com/questions/54940/is-it-possible-to-split-the-bibliography-into-two-different-parts-using-bibtex-t}
%        \end{itemize}
%        \item Dodać oświadczenie na początku pracy
%        \item Stylowanie tabeli: pogróbione nagłówki, podpisy nad tabelą, źródła pod tabelą
%        \item cytaty blokowe
    \end{itemize}
\end{itemize}

\cleardoublepage
\pagenumbering{gobble}
%\thispagestyle{normal}
 %todo usunąć przed oddaniem

    \maketitle
    \pagenumbering{gobble}
    % !TeX spellcheck = pl_PL
% --- Strona ze streszczeniem i~abstraktem ------------------------------------------------------------------
\addtocontents{toc}{\protect\setcounter{tocdepth}{-1}}
\chapter*{Streszczenie} % po polsku
% Wprowadzenie
\todo{Wprowadzenie}
% Sposób rozwiązania problemu
\todo{Sposób rozwiązania problemu}
% Dodatkowe informacji o pracy
\todo{Dodatkowe informacje o pracy}
% Podsumowanie
\todo{Podumowanie}



% Kilka sztuczek, żeby:
%~- Abstract pojawił się na tej samej stronie co Streszczenie
%~- Abstract nie pojawił się w~spisie treści
\addtocontents{toc}{\protect\setcounter{tocdepth}{-1}}
\begingroup
\renewcommand{\cleardoublepage}{}
\renewcommand{\clearpage}{}
\chapter*{Abstract} % ...i to samo po angielsku

\todo{To samo co wyżej ale po angielsku}

\endgroup
\addtocontents{toc}{\protect\setcounter{tocdepth}{2}}
% --- Koniec strony ze streszczeniem i~abstraktem -----------------------------------------------------------

    \cleardoublepage

    \pdfbookmark{\contentsname}{toc}
    \tableofcontents
    \cleardoublepage

    \pagenumbering{arabic}
    \setlength{\parskip}{6pt}%
    % !TeX spellcheck = pl_PL

\chapter*{Wstęp}\label{ch:admission}

\section*{Opis problemu}\label{sec:admission:problem-description}
\todo{Wprowadzenie do specyfiki rynku pracy IT}
\todo{Problem selekcji kandydatów}
\todo{Problem rotacji pracowników w firmie}

\section*{Cel pracy}\label{sec:admission:thesis-goal}

\todo{}

\section*{Zakres pracy}\label{sec:admission:scope-of-work}

\todo{}

\section*{Struktura pracy}\label{sec:admission:thesis-structure}

\thispagestyle{normal}


%    % !TeX~program = latexmk
% !TeX spellcheck = pl_PL
% !TeX~root = example.tex

\chapter{LaTeX demo}

\textcolor{red}{
    Niniejszy rozdział zawiera prezentację możliwości przygotowanego szablonu pracy
    i zostanie usunięty we właściwej pracy.
}

ĄĆĘŁŃÓŚŹŻ ąćęłńóśźż\footnote{Przykład użycia polskich znaków diakrytycznych oraz przypisu w~miejscu}.

Reszta dokumentacji znajduje się w\nolink{docker_compose_reference}.

Jak pisze Harel w\nolink{harel_rzecz_2008}: \lipsum[1]

Natomiast Kaleta uważał\nolink{kaleta_experimental_2005} \lipsum[1]

Jak widać na rys. \ref{fig:network} Docker ma wewnętrzną sieć.

\imagewidth[\bibentry{docker_compose_reference}\nolink{docker_compose_reference}]{0.6}{demo/img/swarm-network}{Docker ma sieć}{network}

Jak widać na rys.\ref{fig:kotek} Ala ma kota. \lipsum[1]

\imagewidth{0.4}{demo/img/kotek}{Ala ma kota}{kotek}

Co uwzględniono w~tabeli \ref{tabela:coktoma}. \lipsum[1]

\noindent\begin{minipage}{\textwidth}
    \begin{table}[H]
        \raggedright\caption{Co kto ma (patrz też dodatek~\ref{app:dod1}) \label{tabela:coktoma}}
        \begin{center}\begin{tabular}{|l|l|l|}% wyrównanie kolumn tabeli -> l~c r~- do lewej, środka, do prawej
                          \hline
                          Ala & ma & kota \\
                          \hline
                          Ola & ma & psa \\
                          \hline
                          Ula & ma & małpę\\
                          \hline
        \end{tabular}\end{center}
        \raggedright\source{\bibentry{harel_rzecz_2008}\nolink{harel_rzecz_2008}}
        \vspace{0.75cm}
    \end{table}
\end{minipage}

W~moim kodzie \ref{listing:liquibase} zrobiłem coś wspaniałego. \lipsum[1]
% lub {java} albo {bash} albo {text}
%\noindent\begin{minipage}{\textwidth}
%\begin{listing}[float=h!]
%    \caption{Przykładowy algorytm w~języku C} \label{listing:moj}
%    \begin{minted}{c}
%        int main()
%        {
%        int a=2*3;
%        printf("**Ala ma kota\n**");
%        while(!I2C_CheckEvent(I2C1, I2C_EVENT_MASTER_MODE_SELECT)); /* EV5 */
%        return 0;
%        }
%    \end{minted}
%    \raggedright\source{\ownwork}
%    \vspace{0.75cm}
%\end{listing}

%\end{minipage}
\noindent\begin{minipage}{\textwidth}
    \begin{lstlisting}[caption={Skrpyt ładujący dane z~pliku CSV}, label={listing:liquibase}]
    \end{lstlisting}
    \hspace{.075\textwidth}\begin{minipage}{.85\textwidth}
    \begin{minted}{xml}
<changeSet id="201902130001" author="kmarczynski" context="usda">
  <loadData file="config/liquibase/usda_sr_db/household_measure.csv"
    quotchar='"' separator=";" tableName="household_measure">
    <column header="product_id" name="product_id" type="numeric"/>
    <column header="measure_description" name="description" type="string"/>
    <column header="grams_weight" name="grams_weight" type="numeric"/>
    <column header="is_visible" name="is_visible" type="boolean"/>
  </loadData>
</changeSet>
    \end{minted}
    \end{minipage}

    \raggedright\source{\ownwork}
    \vspace{0.75cm}
\end{minipage}

\lipsum[1]

W tabelach \ref{tabela:rozwiazania-konkurencyjne-funkcjonalne}, \ref{tabela:sformulowanie-problemu}, \ref{tabela:uzytkownicy} przedstawiono przykładowe formatowanie tabel.

\noindent\begin{minipage}{\textwidth}
    \begin{table}[H]
        \raggedright\caption{Rozwiązania konkurencyjne~- cechy funkcjonalne\label{tabela:rozwiazania-konkurencyjne-funkcjonalne}}
        \begin{center}\begin{tabular}{|P{.22\textwidth}|P{.09\textwidth}|P{.09\textwidth}|P{.09\textwidth}|P{.09\textwidth}|P{.09\textwidth}|P{.09\textwidth}|}
            \hline
            & \cellgray{Rozw1}      & \cellgray{Rozw2}          & \cellgray{Rozw3}         & \cellgray{Rozw4}           & \cellgray{Rozw5}      & \cellgray{Rozw6}    \\ \hline
            \cellgray{Funkcjonalność 1}        & \cellgreen{TAK}       & \cellgreen{TAK}           & \cellgreen{TAK}          & \cellgreen{TAK}            & \cellgreen{TAK}       & \cellgreen{TAK}     \\ \hline
            \cellgray{Funkcjonalność 2}        & \cellgreen{TAK}       & \cellgreen{TAK}           & \cellgreen{TAK}          & \cellgreen{TAK}            & \cellred{NIE}         & \cellred{NIE}       \\ \hline
            \cellgray{Funkcjonalność 3}        & \cellgreen{TAK}       & \cellgreen{TAK}           & \cellgreen{TAK}          & \cellgreen{TAK}            & \cellgreen{TAK}       & \cellgreen{TAK}     \\ \hline
            \cellgray{Funkcjonalność 4}        & \cellgreen{TAK}       & \cellgreen{TAK}           & \cellgreen{TAK}          & \cellgreen{TAK}            & \cellred{NIE}         & \cellgreen{TAK}     \\ \hline
            \cellgray{Funkcjonalność 5}        & \cellgreen{TAK}       & \cellgreen{TAK}           & \cellgreen{TAK}          & \cellred{NIE}              & \cellred{NIE}         & \cellgreen{TAK}     \\ \hline
            \cellgray{Funkcjonalność 6}        & \cellgreen{TAK}       & \cellgreen{TAK}           & \cellgreen{TAK}          & \cellgreen{TAK}            & \cellgreen{TAK}       & \cellgreen{TAK}     \\ \hline
            \cellgray{Funkcjonalność 7}        & \cellgreen{TAK}       & \cellgreen{TAK}           & \cellgreen{TAK}          & \cellred{NIE}              & \cellgreen{TAK}       & \cellgreen{TAK}     \\ \hline
            \cellgray{Funkcjonalność 8}        & \cellgreen{TAK}       & \cellgreen{TAK}           & \cellgreen{TAK}          & \cellgreen{TAK}            & \cellgreen{TAK}       & \cellred{NIE}       \\ \hline
            \cellgray{Funkcjonalność 9}        & \cellgreen{TAK}       & \cellgreen{TAK}           & \cellgreen{TAK}          & \cellgreen{TAK}            & \cellgreen{TAK}       & \cellgreen{TAK}     \\ \hline
            \cellgray{Funkcjonalność 10}       & \cellgreen{TAK}       & \cellgreen{TAK}           & \cellgreen{TAK}          & \cellgreen{TAK}            & \cellgreen{TAK}       & \cellgreen{TAK}     \\ \hline
        \end{tabular}\end{center}
        \raggedright\source{\ownwork}
        \vspace{0.75cm}
    \end{table}
\end{minipage}

\noindent\begin{minipage}{\textwidth}
    \begin{table}[H]
        \raggedright\caption{Sformułowanie problemu\label{tabela:sformulowanie-problemu}}
        \begin{center}\begin{tabular}{|P{.2\textwidth}|p{.7\textwidth}|}

            \hline
            \cellgray{Problem} &
            \inlinetodo{todo} \\
            \hline

            \cellgray{Dotyczy} &
            \inlinetodo{todo} \\
            \hline

            \cellgray{Wpływ problemu} &
            \begin{itemize}
                \item \inlinetodo{todo}
                \item \inlinetodo{todo}
                \item \inlinetodo{todo}
            \end{itemize} \\
            \hline

            \cellgray{Pomyślne rozwiązanie} &
            \begin{itemize}
                \item \inlinetodo{todo}
                \item \inlinetodo{todo}
                \item \inlinetodo{todo}
            \end{itemize} \\
            \hline
        \end{tabular}\end{center}
        \raggedright\source{\ownwork}
        \vspace{0.75cm}
    \end{table}
\end{minipage}

\noindent\begin{minipage}{\textwidth}
    \begin{table}[H]
        \raggedright\caption{Użytkownicy\label{tabela:uzytkownicy}}
        \begin{center}\begin{tabular}{|P{.15\textwidth}|P{.25\textwidth}|P{.5\textwidth}|}

            \hline
            \cellgray{Nazwa} & \cellgray{Opis} & \cellgray{Odpowiedzialności}\\

            \hline
            Gość &
            Niezalogowany użytkownik &
            \begin{itemize}
                \item Zakłada konto użytkownika.
                \item Wyświetla stronę główną.
            \end{itemize} \\
            \hline
            Administrator &
            Osoba zarządzająca działaniem aplikacji &
            \begin{itemize}
                \item Przydzielanie i~odbieranie użytkownikom uprawnień.
                \item Zarządzanie definicjami wartości odżywczych, typami diet, typami posiłków, typami dań i~wyposażeniem kuchennym.
            \end{itemize} \\
            \hline
        \end{tabular}\end{center}
        \raggedright\source{\ownwork}
        \vspace{0.75cm}
    \end{table}
\end{minipage}

\thispagestyle{normal}


    % !TeX spellcheck = pl_PL
\chapter{Stan wiedzy i~techniki w~zakresie tematyki pracy}\label{ch:knowladge-state}
\section{Metody selekcji kandydatów}\label{sec:employee-selection}
\todo{}

\section{Strategie zmniejszenia rotacji pracowników}\label{sec:employee-turnover}
\todo{}

\section{Predykcja zadowolenia pracowników i chęci zmiany pracy z wykorzystaniem uczenia maszynowego}\label{sec:employee-turnover-machine-learning}
\todo{\url{https://pdfs.semanticscholar.org/fa49/19810eaee67e851ad13775b78c94217a7908.pdf}}

\thispagestyle{normal}

    % !TeX spellcheck = pl_PL
\chapter{Charakterystyka branży IT}\label{ch:it-sector}
\section{Wysoka fluktuacja}\label{sec:it-turnover}
\todo{niedobór programistów a wysoki popyt na ekspertów}

\section{Wpływ fluktuacji na utratę wiedzy w projektach informatycznych}\label{sec:it-knowledge-loss}
\todo{}

\section{Zarządzanie projektami informatycznymi a wysoka fluktuacja}\label{sec:it-project-management}
\todo{koszty fluktuacji w związku z rekrutacją}

\thispagestyle{normal}

    % !TeX spellcheck = pl_PL
\chapter{Badania}\label{ch:analysis}
\section{Wybór źródła danych: Prezentacja ankiety StackOverflow}\label{sec:analysis:data-source-selection}
\todo{}
\section{Porównanie zbioru pytań z kolejnych edycji ankiety}\label{sec:analysis:questions-comparision}
\todo{}
\section{Wstępna selekcja cech i edycji ankiety}\label{sec:analysis:feature-pre-selection}
\todo{z wyszczególnieniem cech stałych/środowiskowych (kraj, typ dewelopera, typ firmy, rozmiar firmy) i zmiennych (wynagrodzenie, benefity, metodologie, cechy profilu kandydata)}
\section{Wstępne przetworzenie danych}\label{sec:analysis:preprocessing}
\todo{kodowanie liczbowe, oczyszczanie, wzbogacenie}
\todo{\url{https://medium.com/@s.pranav.harathi/stack-overflow-survey-analysis-ed45127691b}}
\section{Selekcja cech z wykorzystaniem algorytmu XGB}\label{sec:analysis:feature-selection-xgb}
\todo{}
\section{Prezentacja cech istotnie wpływających na predykcję}\label{sec:analysis:important-features}
\todo{}
\section{Budowa modelu uczenia maszynowego}\label{sec:analysis:ml-model}
\todo{w oparciu o cechy istotnie wpływające na predykcję}
\section{Analiza dopasowania modelu}\label{sec:analysis:model-fitness}
\todo{Analiza skuteczności (dopasowania) modelu}

\thispagestyle{normal}


    % !TeX spellcheck = pl_PL
\chapter*{Zakończenie}\label{ch:ending}

\todo{zakończenie}

\thispagestyle{normal}


    % W~pracy pojawią się tylko prace naprawdę cytowane.
    % \nocite{*}
    \cleardoublepage
    \bibliography{literatura}
    \bibliographystyle{dyplom}

    \listnormal{\listoffigures}{\listfigurename}
    \listnormal{\listoftables}{\listtablename}

    \clearpage
%    \listof{listing}{Spis kodów źródłowych}
    \lstlistoflistings

    % !TeX spellcheck = pl_PL
% \appendixpage
% \addappheadtotoc

\appendix
\begin{appendices}
    \chapter{To powinien być dodatek}\label{app:dod1}

    \lipsum[5]

\end{appendices}
\thispagestyle{normal}


    \cleardoublepage
    \pagenumbering{gobble}
    % !TeX spellcheck = pl_PL
\begin{center}
    \large\textbf{\MakeUppercase{OŚWIADCZENIE AUTORA PRACY}}\\
    \vspace{1cm}
\end{center}

Świadom odpowiedzialności prawnej oświadczam, że niniejsza praca dyplomowa została napisana przeze mnie samodzielnie. Wszystkie dane, istotne myśli i sformułowania pochodzące z literatury (przytoczone dosłownie lub niedosłownie) są opatrzone odpowiednimi odsyłaczami.
\par
Praca ta w całości ani w części, która zawierałaby znaczne fragmenty przedstawione w pracy jako oryginalne, nie była wcześniej przedmiotem procedur związanych z uzyskaniem tytułu zawodowego w wyższej uczelni.
\par
Oświadczam, że tekst pracy dyplomowej wgrany do systemu APD jest identyczny z tekstem wydrukowanym złożonym w dziekanacie, o ile złożenie pracy w dziekanacie jest wymagane aktualnymi regulacjami Uczelni.
\par
\vspace{1cm}
\noindent UWAGA: Oświadczenie składane w wersji elektronicznej w systemie APD

\begin{center}
    \vspace{1cm}
    \large\textbf{\MakeUppercase{OŚWIADCZENIE PROMOTORA}}\\
    \vspace{1cm}
\end{center}

Oświadczam, że niniejsza praca dyplomowa została przygotowana pod moim kierunkiem i spełnia warunki do przedstawienia jej w postępowaniu o nadanie tytułu zawodowego.
\par
Jednocześnie oświadczam, że tematyka pracy jest zgodna z efektami uczenia się określonymi dla kierunku Autora pracy.
\par
\vspace{1cm}
\noindent UWAGA: Oświadczenie składane w wersji elektronicznej w systemie APD

\thispagestyle{normal}


\end{document}
