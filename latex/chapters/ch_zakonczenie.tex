% !TeX spellcheck = pl_PL
\chapter*{Zakończenie}\label{ch:ending}

Celem pracy było sprawdzenie, czy uczenie maszynowe pozwala określić z zadowalającym stopniem
pewności cechy wpływające na poziom zadowolenia i chęć zmiany pracy pracowników IT.

Aby osiągnąć ten cel dla danych z ankiet deweloperskich StackOverflow \cite{so-survey-info} z lat 2017, 2018 i 2019
opracowano algorytm w języku Python pozwalający na zbudowanie modeli predykcji opartych o regresję i klasyfikację.

W wyniku badań wyłoniono zestaw cech silnie wpływających na zadowolenie pracowników:

\begin{itemize}
    \item Jak często respondent obecnie pracuje zdalnie,
    \item Czy respondent uważa, że możliwość rozwoju osobistego w firmie jest ważna,
    \item Indeks rozwoju społecznego kraju, w którym mieszka respondent,
    \item W jakim stopniu respondent czuje solidarność ze swoimi współpracownikami,
    \item Czy respondent uważa, że możliwość pracy zdalnej jest ważna.
\end{itemize}

oraz na deklarowaną przez pracowników chęć szukania nowej pracy:

\begin{itemize}
    \item Czy respondent uważa, że możliwość rozwoju osobistego w firmie jest ważna,
    \item Czy respondent uważa, że wymagane na stanowisku doświadczenie jest ważne,
    \item Czy respondent uważa, że technologie wykorzystywane w pracy są ważne,
    \item Czy respondent uważa, że czas dojazdu do pracy jest ważny,
    \item Czy respondent uważa, że nazwa roli lub stanowiska pracy jest ważna,
    \item Czy respondent uważa, że reputacja seniorów pracujących w firmie jest ważna.
\end{itemize}

Na podstawie analizy wyników badań można podjąć się odpowiedzi na postawione w pracy pytania badawcze:
\begin{enumerate}
    \item Można wyłonić cechy (osoby lub przedsiębiorstwa) pozwalające oszacować zadowolenie i~chęć zmiany pracy pracownika z~branży IT.
    \item Wyłonione cechy są raczej związane z preferencjami dotyczącymi stanowiska, które firma może dostosować, aby zwiększyć atrakcyjność miejsca pracy dla swoich pracowników. W wyniku badania nie wyłoniono cech, które silnie związane byłyby z profilem kandydata, więc nie można stwierdzić, że wyłonione cechy można użyć do poprawy procesu wstępnej selekcji kandydatów.
    \item Wyłonione cechy można użyć do zwiększenia atrakcyjności przedsiębiorstwa dla pracowników i~kandydatów, na przykład poprzez:
    \begin{itemize}
        \item oferowanie pracownikom możliwości rozwoju osobistego
        \item oferowanie pracownikom możliwości pracy zdalnej
        \item oferowanie pracownikom atrakcyjnej nazwy stanowiska
        \end{itemize}
\end{enumerate}

Przeprowadzone badania mogą być potraktowane jako wstęp do omówionej problematyki.
%Dzięki temu, że badanie przeprowadzono na 3 niezależnych zbiorach danych (pomimo tego, że wykorzystano kolejne edycje ankiety StackOverflow, to ze względu na to, że co roku zadawano inne pytania, a odpowiedzi udzieliła inna grupa użytkowników, można wykorzystane zbiory danych traktować jako niezależne),
%można stwierdzić, że dla odpowiednio postawionych pytań możliwe jest wyłonienie zbio
Uzyskana skuteczność predykcji pokazuje, że przy odpowiednio postawionych pytaniach możliwe jest poprzez przeprowadzenie ankiety wyłonienie cech, które odpowiadają za zadowolenie i chęć zmiany pracy pracowników.
Badania przeprowadzone w ramach niniejszej pracy traktowały jednorodnie całą branżę informatyczną.
Przyszłe badania mogłyby skupić się na podziale pracowników (np. ze względu na specjalizacje) i firm (np. ze względu na organizację) oraz na przygotowaniu modeli predykcji dla innych zbiorów danych.
Uzyskane specjalistyczne modele dla wielu zbiorów danych mogą następnie być połączone, aby dalej zwiększać skuteczność predykcji.
Następnie modele predykcji mogą zostać wykorzystane w formie aplikacji komputerowej, w której po wprowadzeniu danych firmy i kandydata, możliwe mogłoby być określenie dopasowania kandydata do firmy.


\thispagestyle{normal}
