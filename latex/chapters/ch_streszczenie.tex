% !TeX spellcheck = pl_PL
% --- Strona ze streszczeniem i~abstraktem ------------------------------------------------------------------
\addtocontents{toc}{\protect\setcounter{tocdepth}{-1}}
\chapter*{Streszczenie} % po polsku
Celem pracy było sprawdzenie, czy uczenie maszynowe pozwala określić z~zadowalającym stopniem
pewności cechy wpływające na poziom zadowolenia i~chęć zmiany pracy pracowników IT.
Aby osiągnąć ten cel przeprowadzono przegląd literatury w~zakresie problematyki fluktuacji i~zadowolenia pracowników
ze szczególnym zwróceniem uwagi na charakterystyki specyficzne dla branży informatycznej.
Następnie wybrano źródło danych pozwalające przeprowadzić badania ilościowe dla społeczności pracowników zatrudnionych w~branży informatycznej
i przygotowano algorytm w~języku Python pozwalający na zbudowanie modeli predykcji opartych o~regresję i~klasyfikację.
W celu oceny skuteczności modelu predykcji wybrano zestawy odpowiednich metryk zarówno dla regresji, jak i~klasyfikacji.
Rezultaty badań mogą stanowić sugestię dla pracowników działów rekrutacyjnych i~employer branding,
jakie kwestie organizacyjne w~firmie mogą ulec usprawnieniu, aby zwiększyć retencję i~zadowolenie pracowników.

% Kilka sztuczek, żeby:
%~- Abstract pojawił się na tej samej stronie co Streszczenie
%~- Abstract nie pojawił się w~spisie treści
\addtocontents{toc}{\protect\setcounter{tocdepth}{-1}}
\begingroup
\renewcommand{\cleardoublepage}{}
\renewcommand{\clearpage}{}
\chapter*{Abstract} % ...i to samo po angielsku

The aim of this work was to check whether machine learning allows to determine with a~satisfactory degree of certainty the features influencing the level of satisfaction and willingness to change the job of IT employees.
To achieve this goal, a~literature review was carried out in the field of employee fluctuation and satisfaction, paying particular attention to the characteristics specific to the IT industry.
Then, a~data source was selected to conduct quantitative research for the community of employees working in the IT industry,
and an algorithm in the Python language was prepared to build prediction models based on regression and classification.
In order to assess the effectiveness of the prediction models, sets of appropriate metrics were selected for both regression and classification.
The results of the research may be a~suggestion for employees of recruitment and employer branding departments
which organizational issues in the company can be improved in order to increase retention and satisfaction of the employees.

\endgroup
\addtocontents{toc}{\protect\setcounter{tocdepth}{2}}
% --- Koniec strony ze streszczeniem i~abstraktem -----------------------------------------------------------
