% !TeX spellcheck = pl_PL

\chapter*{Wstęp}\label{ch:admission}

\section*{Opis problemu}\label{sec:admission:problem-description}

W ciągu ostatniego stulecia na świecie można zauwążyć bezprecedensowe tempo rozwoju nowych technologii,
w związku z~czym nieustannie rośnie zapotrzebowanie na odpowiednio wykwalifikowanych pracowników technicznych.
Na uwagę szczególnie zasługuje rozwój informatyzacji i~zapotrzebowanie na pracowników zajmującyh się wytwarzaniem oprogramowania.
Według raportu Daxx w~2021 roku na świecie jest zatrudnionych ok 27 milionów programistów, natomiast globalne zapotrzebowanie jest o~40 milionów większe.
Oszacowali też, że do 2030 globalnie może być zatrudnionych ok 45 milionów programistów, a~deficyt sięgać może nawet 85 milionów programistów \cite{daxx-2021}.

Na rysunku \ref{fig:talent-shortage} przedstawiono problem niedoboru wykfalifikowanych pracowników w~branży IT z~podziałem na kraje.

\imagewide[\cite{daxx-nordic-2020}]{img/talent-shortage}{Niedobór wykwalifikowanych pracowników IT na świecie}{talent-shortage}

Mając na uwadze również, że całkowite koszty związane z~odejściem pracownika na stanowisku programisty i~zatrudnieniem na jego miejsce nowego
wynoszą średnio ok 50 tys dolarów amerykańskich \cite{hairing-dev-2021} (daje to kwotę ok 200 tys polskich złotych), wniosek nasuwa się prosty:
firmom działającym w~branży informatycznej zależy na zatrzymaniu (retencji) zatrudnionych programistów.
Dzięki wysokim staraniom pracodawców związanych z~retencją programistów,
zawód programisty na początku 2022 roku znalazł się na 5 miejscu rankingu najlepszych zawodów według U.S.News \cite{us-news-2022}.

W związku z~omówionymi problemami związanymi z~branżą informatyczną można sformułować następujące pytania badawcze:
\begin{enumerate}
    \item Czy można wyłonić cechy (osoby lub przedsiębiorstwa) pozwalające oszacować zadowolenie i~chęć zmiany pracy pracownika z~branży IT?
    \item Czy wyłonione cechy można użyć do poprawy procesu wstępnej selekcji kandydatów?
    \item Czy wyłonione cechy można użyć do zwiększenia atrakcyjności przedsiębiorstwa dla pracowników i~kandydatów?
\end{enumerate}

\section*{Cel i~zakres pracy}\label{sec:admission:thesis-goal}

Cele pracy można podzielić na 3 kategorie:
\begin{enumerate}
    \item Teoriopoznawcze
    \begin{itemize}
        \item przedstawienie zjawiska fluktuacji pracowników,
        \item czynniki wpływające na fluktuację i~zadowolenie pracowników w~literaturze,
        \item koszty związane z~fluktuacją pracowników,
        \item przedstawienie charaktystyki branży informatycznej i~konsekwencji fluktuacji w~tej branży,
    \end{itemize}
    \item Metodologiczne
    \begin{itemize}
        \item sprawdzenie czy uczenie maszynowe pozwala określić z~zadowalającym stopniem pewności (relatywnie niski znormalizowany błąd średniokwadratowy (ang. NRMSE)) cechy wpływające na poziom zadowolenia i~chęć zmiany pracy pracowników IT,
        \item sprawdzenie czy wytypowane cechy będą miały odzwierciedlenie w~cechach opisywanych w~literaturze,
    \end{itemize}
    \item Utylitarne
    \begin{itemize}
        \item próba przygotowania użytecznego narzędzia dla pracowników działów rekrutacyjnych i~employer branding.
    \end{itemize}
\end{enumerate}

\section*{Struktura pracy}\label{sec:admission:thesis-structure}

W pierwszym rozdziale pracy omawiane jest zjawisko fluktuacji pracowników z~uwzględnieniem występującego w~literaturze podziału na jej rodzaje,
pokazania czym fluktuacja różni się od rotacji pracowników oraz w~jaki sposób z~fluktuacją związana jest retencja.
Następnie przedstawione są czynniki prowadzące do wzrostu fluktucji, pokazane jest czy wysoki wskaźnik fluktuacji zawsze oznacza problemy dla przedsiębiorstwa
oraz jakie koszty ponosi firma w~związku z~fluktuacją.

W rozdziale drugim nakreślona zostaje charakterystyka zatrudnienia w~branży informatycznej ze szczególnym zwróceniem uwagi na wskaźnik fluktuacji pracowników w~tej branży.
Szczegółowej analizie zostaje poddany problem utraty wiedzy w~projektach informatycznych oraz jakości tworzonego oprogramowania w~związku z~odejściami pracowników.

W rozdziale trzecim przedstawiono źródło danych wybrane do przeprowadzenia badań ilościowych
oraz uzasadniono zasadność przeprowadzenia badań z~wykorzystaniem uczenia maszynowego dla omawianego w~pracy problemu.

\thispagestyle{normal}
