% !TeX spellcheck = pl_PL

\chapter*{Wstęp}\label{ch:admission}

\section*{Opis problemu}\label{sec:admission:problem-description}
\todo{do uzupełnienia po napisaniu pracy}
\todo{zarysowanie tła badanego zagadnienia}
\todo{przesłanki wyboru tematu pracy}
\todo{problem badawczy}

\section*{Cel pracy}\label{sec:admission:thesis-goal}

\todo{do uzupełnienia po napisaniu pracy}

\section*{Zakres pracy}\label{sec:admission:scope-of-work}

\todo{zakres do uzupełnienia po napisaniu pracy}

\section*{Struktura pracy}\label{sec:admission:thesis-structure}

W pierwszym rozdziale pracy omawiane jest zjawisko fluktuacji pracowników z uwzględnieniem występującego w literaturze podziału na jej rodzaje,
pokazania czym fluktuacji różni się od rotacji pracowników oraz w jaki sposób z fluktuacją związana jest retencja.
Następnie przedstawione są czynniki prowadzące do wzrostu fluktucji, pokazane jest czy wysoki wskaźnik fluktuacji zawsze oznacza problemy dla przedsiębiorstwa
oraz jakie koszty ponosi firma w związku z fluktuacją.

W rozdziale drugim nakreślona zostaje charakterystyka zatrudnienia w branży informatycznej ze szczególnym zwróceniem uwagi na wskaźnik fluktuacji pracowników w tej branży.
Szczegółowej analizie zostaje poddany problem utraty wiedzy w projektach informatycznych oraz jakości tworzonego oprogramowania w związku z odejściami pracowników.

W rozdziale trzecim przedstawiono źródło danych wybrane do przeprowadzenia badań ilościowych
oraz uzasadniono zasadność przeprowadzenia badań z wykorzystaniem uczenia maszynowego dla omawianego w pracy problemu.

\thispagestyle{normal}
