% !TeX spellcheck = pl_PL


\chapter{Charakterystyka branży IT}\label{ch:it-sector}


\section{Ogólna charakterystyka}\label{sec:it-industry-summary}

Branża IT jest stosunkowo młoda.
Konepcja stanowiąca podstawę architektury wszystkich współczesnych komputerów~- a~ogólniej w~zasadzie niemal wszystkich urządzeń korzystających z~procesora~- została opracowana w~1945 roku przez Johna von Neumanna, Johna Mauchly’ego i~Johna Eckerta \cite{von-neumann-2020}.
Model komputera implementującego tę architekturę nazwali "przykładową maszyną cyfrową".
Na przestrzeni lat wraz z~postępem technologicznym następował stopniowy wzrot znaczenia technologii informatycznych,
jednak prawdziwy rozkwit branży nastąpił po uruchomieniu w~1993 roku ogólnoświatowej sieci komputerowej (ang. World Wide Web) \cite{rangarajan-2014}, którą potocznie nazywamy "Internetem".

Rok 1993 można uznać za symboliczną datę kiedy rynek pracy IT zaczął przybierać znaną obecnie postać, jednak warto wyszczególnić jeszcze kilka ważnych dat z~historii rozwoju branży:
\begin{itemize}
    \item 2002~- Amazon Web Services~- pierwsza implementacja chmury obliczeniowej \cite{aws-2017},
    \item 2004~- Web 2.0~- ewolucja sposobu kreowania treści w~internecie w~sposób bardziej angażujący użytkowników \cite{web-2.0-2007},
    \item 2007~- iPhone~- początek ery urządzeń mobilnych \cite{iphone-2018}.
\end{itemize}

W Polsce branża IT stanowi bardzo istotny sektor gospodarki.
W 2021 roku działalność w~branży IT stanowiła w~przybliżeniu 8\% PKB Polski \cite{piih-2021}.
Według raportu HackerRank Polscy programiści znajdują się w~światowej czołówce pod względem umiejętności, plasując się na trzecim miejscu, zaraz za Chinami i~Rosją \cite{hackerrank-2021}.
Wysokie kwalifikacje polskich programistów są jedną z~przyczyn dlaczego wiele zagranicznych firm decyduje się na outsourcing projektów do polskich firm informatycznych.
Wśród innych powodów wybierania Polski do procesu outsourcingu wyzczególnić można \cite{softwaremind-2021}:
\begin{itemize}
    \item wysoki stopień znajomości języka angielskiego,
    \item przynależność Polski do Unii Europejskiej zapewnia zgodność z~dyrektywami RODO dotyczącymi nie przetwarzania danych użytkowników poza terenem Unii Europejskiej,
    \item Polska jest zbliżona kulturowo i~lokalizacyjnie do zachodnich zagranicznych inwestorów z~krajów takich jak Niemcy czy Wielka Brytania, a~nawet Stany Zjednoczone ()gdyż różnica czasu między Polską, a~wschodnim wybrzeżem Stanów Zjednoczonych wynosi zaledwie 6 godzin).
\end{itemize}

Duże zainteresowanie zagranicznych inwestorów polskim rynkiem IT oraz ogólny światowy niedobór wykwalifikowanych programistów \cite{daxx-2021}
kreują sytuację, w~której pracodawcy prześcigają się w~przygotowywaniu coraz atrakcyjniejszych ofert pracy zawierających szereg benefitów pracowniczych \cite{it-benefits-2020}.
Tak powstała subiektywnie postrzegana łatwość zmiany pracy może skutkować zwiększeniem fluktuacji pracowników.
Według raportu LinkedIn \cite{linkedin-2018} w~2017 roku branża IT odnotowała najwyższy wskaźnik fluktuacji wynoszący 13.2\%, gdzie globalna średnia dla innych zawodów wynosiła zaledwie 10.9\%.
Szczególnie wysokie wskaźniki fluktuacji odnotowano dla następujących specjalizacji w~branży IT:
\begin{itemize}
    \item 23.3\%~- projektant doświadczeń użytkownika,
    \item 21.7\%~- analityk danych,
    \item 21.7\%~- programista systemów wbudowanych.
\end{itemize}

Jak wspomniano w~rozdziale \ref{sec:wplyw-fluktuacji-na-firme}, wysoki poziom fluktuacji niekoniecznie musi oznaczać problemy dla firmy, jeśli dostępnych jest wielu potencjalnych pracowników, a~koszty rekrutacji są niskie.
Te warunki nie są jednak spełnione w~przypadku branży IT, gdzie wydatki związane z~zatrudnieniem nowego pracownika po odejściu starego mogą sięgać nawet rzędu 200 tys. polskich złotych \cite{hairing-dev-2021}.

%\section{Wysoka fluktuacja}\label{sec:it-turnover}
%\todo{niedobór programistów a~wysoki popyt na ekspertów}
%
%Przyczyny odejść specjalistów na stanowiskach technicznych \cite{kapor-2017}.
%
%Przyczyny odejść programistów \cite{hannon-2008}.


\section{Zadowolenie pracowników}\label{sec:it-motivation}
Jak pokazano w~rozdziale \ref{sec:czynniki-wplywajace-na-fluktuacje} wiele z~klasycznych teorii badających czynniki wpływające na fluktuację wyróżnia zadowolenie z~pracy jako jeden z~kluczowych czynników.
W przypadku branży IT zadowolenie z~pracy wpływa również znacząco na produktywność programistów i~przede wszystkim na jakość tworzonego oprogramowania \cite{graziotin-2018}.
Według Roberta C. Martina~- autorytetu w~dziedzinie stosowania dobrych wzorców w~programowaniu~- jakość kodu ma kluczowe znaczenie dla przyszłości tworzonego oprogramowania, a~także dla przyszłości firmy.
Martin wyszczególnił następujące skutki zaniechania dbałości o~jakość oprogramowania \cite{martin-2014}:
\begin{itemize}
    \item wydłużenie czasu dostarczania nowych funkcjonalności oprogramowania,
    \item częstsze pojawianie się awarii oprogramowania,
    \item spadek produktywności programistów,
    \item spadek motywacji programistów,
    \item konieczność poświęcania większej ilości czasu na analizowanie istniejącego kodu,
    \item zwiększenie kosztów tworzenia oprogramowania,
    \item wydłużenie czasu trwania projektu,
    \item możliwa konieczność napisania całego kodu od nowa.
\end{itemize}

Często stosowanym sposobem zwiększenia zadowolenia pracowników jest ograniczanie niezadowolenia.
Czynniki wpływające na niezadowlenie pracowników w~branży IT można podzielić na 3 podstawowe kategorie \cite{graziotin-2017}:
\begin{enumerate}
    \item wewnętrzne związane z~osobą pracownika
    \item zewnętrzne związane z~organizacją firmy
    \item zewnętrzne związane z~oprogramowaniem
\end{enumerate}

W tabeli \ref{tabela:niezadowolenie-IT} przedstawiono częste przyczyny niezadowlenia pracowników z~uwzględnieniem powyższych kategorii.

\noindent\begin{minipage}{\textwidth}
             \begin{table}[H]
                 \raggedright\caption{Przyczyny niezadowolenia pracowników w~branży IT\label{tabela:niezadowolenie-IT}}
                 \begin{center}
                     \begin{tabular}{|P{.2\textwidth}|P{.7\textwidth}|}

                         \hline
                         \cellgray{Kategoria} &
                         \cellgray{Przyczyny} \\
                         \hline

                         czynniki wewnętrzne związane z~osobą pracownika &
                         \begin{itemize}
                             \item fiksacja na punkcie rozwiązywania problemów,
                             \item syndrom impostora~- poczucie nie posiadania wystarczających kompetencji do pełnionej roli,
                             \item problemy osobiste;
                         \end{itemize} \\

                         \hline

                         czynniki zewnętrzne związane z~organizacją firmy &
                         \begin{itemize}
                             \item presja czasu,
                             \item monotonne lub powtarzające się zadania,
                             \item podejmowanie złych decyzji (zwłaszcza technicznych) na wyższych szczeblach w~organizacji;
                         \end{itemize} \\

                         \hline

                         czynniki zewnętrzne związane z~oprogramowaniem &
                         \begin{itemize}
                             \item niska jakość kodu i~złe praktyki programistyczne,
                             \item praca z~kodem, który nie działa z~niewiadomych przyczyn,
                             \item narzucone ograniczenia dotyczące narzędzi wykorzystywanych do programowania (takiej jak język programowania czy środowisko);
                         \end{itemize} \\

                         \hline

                     \end{tabular}
                 \end{center}
                 \raggedright\source{\workbasedon{\cite{graziotin-2017}}}
                 \vspace{0.75cm}
             \end{table}
\end{minipage}
%Pracownicy zadowoleni a~zmotywowani \cite{sharp-2014}.
%
%Satysfakcja i~produktywność \cite{storey-2021}.
%
%Wpływ środowiska pracy na produktywność i~satysfakcję \cite{johnson-2021}.
%
%Czynniki wpływające na retencję \cite{bass-2018}.
%
%Konsekwencje zadowolenia i~niezadowolenia pracowników \cite{graziotin-2018}.
%
%O niezadowoleniu pracowników \cite{graziotin-2017}.


\section{Wpływ fluktuacji na utratę wiedzy w~projektach informatycznych}\label{sec:it-knowledge-loss}

Jednym z~aspektów związanych z~fluktuacją pracowników w~branży IT jest utrata wiedzy w~projektach informatycznych.
Można wyszczególnić kilka podstawowych przyczyn utraty wiedzy \cite{nesh-2021}:
\begin{itemize}
    \item nieudokumentowane dane~- wiele decyzji projektowych podejmowanych jest w~oparciu o~doświadczenie pracownika, testowanie wielu rozwiązań i~wyciąganie wniosków z~sukcesów i~porażek~- powstaje w~ten sposób rodzaj "wiedzy plemiennej", która może być w~pełni zrozumiała dla obecnych członków zespołu, ale nie jest w~całości przekazywana nowym pracownikom,
    \item udokumentowane, ale nieuporządkowane dane~- wiedza projektowa często jest spisywana w~formie różnorakich dokumentów, które funkcjonują w~obiegu między pracownikami, jednak w~przypadku braku uporządkowanej struktury bazy wiedzy projektowej dokumenty takie mogą łatwo zaginąć w~przypadku odejścia pracownika,
    \item relacje międzyludzkie~- decyzje projektowe mogą być podejmowane na podstawie wymagań klienta; w~wielu projektach informatycznych wymagania klienta nie są udokumentowane w~formalnych dokumentach przygotowanych przez analityków biznesowych, ale są przekazywane w~formie ustnej lub mailowej, z~tego powodu w~przypadku odejścia pracownika wiedza wynikająca z~relacji z~klientem jest często trudna do przekazania.
\end{itemize}

Donadelli \cite{donadelli-2015} wyszczególnia dodatkowe czynniki związane z~wysoką fluktuacją w~projektach informatycznych:
\begin{itemize}
    \item spadek produktywności firmy,
    \item obniżenie jakości wytwarzanego kodu,
    \item zwiększenie liczby "porzuconych" plików (tj. takich, które od dawna nie były edytowane przez żadnego z~członków zespołu),
    \item ograniczenie rozprzestrzeniania wiedzy w~projekcie.
\end{itemize}

Na szczególną uwagę zasługuje ostatni wymieniony punkt.
W zwinnych metodykach zarządzania projektami stosuje się "wskaźnik autobusu", który dosłownie oznacza wskazanie ilu członków zespołu musiałoby zostać potrąconych przez autobus, aby w~znaczący sposób wpłynęło to negatywnie na przyszłość prowadzonego projektu \cite{truck-factor-2005}.
Wskaźnik autobusu zwykle jest szczególnie istotny w~projektach prowadzonych przez małe zespoły, gdzie każdy z~członków zespołu zajmuje się pracą nad odrebną częścią systemu.
W takich projektach odejście pojedynczego pracownika może doprowadzić do utraty znaczącej części wiedzy.
Sytuacja zwykle inaczej się prezentuje w~przypadku dużych projektów, gdzie wiedza jest znacząco bardziej rozproszona i~odejście pracownika nie wpływa istotnie na dalsze prowadzenie projektu.

%Przekazywanie kodu a~produktywność programistów \cite{mockus-2009}.
%
%Negatywny wpływ rotacji na jakość kodu \cite{donadelli-2015}.
%
%Unikanie negatywnego wpływu rotacji na utratę wiedzy \cite{rigby-2016}.


%\section{Zarządzanie zasobami ludzkimi}\label{sec:it-project-management}
%\todo{koszty fluktuacji w~związku z~rekrutacją}
%
%Zaangażowanie organizacyjne pracowników \cite{rosinski-2012}.
%
%Zarządzanie kapitałem ludzkim \cite{brylka-2019}.


\section{Charakterystyka pracowników w~branży IT}

W niniejszym podrozdziale przedstawiona zostanie demografia pracowników z~branży informatycznej.
Opracowanie zostało przygotowane na podstawie danych z~ankiety deweloperskiej przeprowadzonej przez StackOverflow w~2020 roku, w~której wzięło udział około 65 tysięcy programistów z~całego świata \cite{so-survey-2020}.

Na rysunku \ref{fig:firstLineOfCode} pokazano w~jakim wieku respondenci napisali swoją pierwszą linijkę kodu.
Na wykresie widać, że przeważająca większość respondentów~- aż 54\%~- miała pierwszy kontakt z~programowaniem już przed 16 rokiem życia.
Jednakże, wczesny kontakt z~programowaniem nie przekłada się wśród respondentów na długi staż zawwodowy, gdyż jak widać na rysunku \ref{fig:yearsCodingProf} około 65\% badanych osób ma mniej niż 10 lat doświadczenai zawodowego.
Można domniemywać, że w~związku ze wzrostem znaczenia branży IT w~ciągu ostatnich kilkkudziesięciu lat oraz wzrostem zapotrzebowania na specjalistów IT na rynku pracy wzrasta zainteresowanie branżą IT wśród młodszych pokoleń.
Obrazować może to fakt, że aż 70\% respondentów to osoby w~wieku poniżej 35 lat, co pokazuje rysunek \ref{fig:age}.


\bargraph[\workbasedon{\cite{so-survey-2020}}]{{Poniżej 10 lat}, {10-11 lat}, {12-13 lat}, {14-15 lat}, {16-17 lat}, {18-19 lat}, {20-21 lat}, {22-23 lat}, {24-25 lat}, {26-27 lat}, {28-29 lat}, {Powyżej 30 lat}}{(8.9,{Poniżej 10 lat}) (10,{10-11 lat}) (16,{12-13 lat}) (19.2,{14-15 lat}) (16.3,{16-17 lat}) (14.7,{18-19 lat}) (6.3,{20-21 lat}) (3,{22-23 lat}) (2.1,{24-25 lat}) (1,{26-27 lat}) (0.7,{28-29 lat}) (1.7,{Powyżej 30 lat})}{Rozkład wieku, w~którym respondenci ankiety StackOverflow napisali pierwszą linię kodu}{firstLineOfCode}{9cm}


\bargraph[\workbasedon{\cite{so-survey-2020}}]{{Poniżej 5 lat}, {5-9 lat}, {10-14 lat}, {15-19 lat}, {20-24 lat}, {25-29 lat}, {30-34 lat}, {35-39 lat}, {40-44 lat}, {45-49 lat}, {Powyżej 50 lat}}{(39.6,{Poniżej 5 lat}) (26.8,{5-9 lat}) (14.7,{10-14 lat}) (7.6,{15-19 lat}) (6,{20-24 lat}) (2.4,{25-29 lat}) (1.6,{30-34 lat}) (0.8,{35-39 lat}) (0.4,{40-44 lat}) (0.1,{45-49 lat}) (0.1,{Powyżej 50 lat})}{Rozkład stażu pracy respondentów ankiety StackOverflow w~latach}{yearsCodingProf}{9cm}


\bargraph[\workbasedon{\cite{so-survey-2020}}]{{15-19 lat}, {20-24 lat}, {25-29 lat}, {30-34 lat}, {35-39 lat}, {40-44 lat}, {45-49 lat}, {50-54 lat}, {55-59 lat}, {Powyżej 60 lat}}{(1.2,{15-19 lat}) (16.6,{20-24 lat}) (29.5,{25-29 lat}) (21.9,{30-34 lat}) (14,{35-39 lat}) (7.5,{40-44 lat}) (4.2,{45-49 lat}) (2.5,{50-54 lat}) (1.5,{55-59 lat}) (1.1,{Powyżej 60 lat})}{Rozkład wieku respondentów ankiety StackOverflow}{age}{9cm}

Na rysunku \ref{fig:education} pokazano jakie najwyższe wykształcenie uzyskali respondenci ankiety. Ciekawe jest, że mimo iż firmy z~branży IT zwykle nie wymagają od pracowników formalnego wykształcenia, aż 75\% badanych osób ma ukończone co najmniej studia I~stopnia.


\bargraph[\workbasedon{\cite{so-survey-2020}}]{{Brak formalnego wykształcenia}, {Szkoła podstawowa}, {Szkoła gimnazjalna}, {Szkoła licealna}, {Szkoła policealna}, {Szkoła zawodowa}, {Studia I~stopnia}, {Studia II stopnia}, {Studia doktorskie}}{(0.7,{Brak formalnego wykształcenia}) (0.5,{Szkoła podstawowa}) (4.5,{Szkoła gimnazjalna}) (11.5,{Szkoła licealna}) (3.2,{Szkoła policealna}) (1.4,{Szkoła zawodowa}) (49.3,{Studia I~stopnia}) (25.5,{Studia II stopnia}) (3.3,{Studia doktorskie})}{Rozkład wykształcenia respondentów ankiety StackOverflow}{education}{9cm}

Na rysunku \ref{fig:gender} przedstawiono rozkład płci respondentów i~jak widać branża IT jest silnie zdominowana przez mężczyzn (ponad 90\% respondentów).


\bargraph[\workbasedon{\cite{so-survey-2020}}]{Mężczyzna, Kobieta, Inna}{(91.2,Mężczyzna) (7.6,Kobieta) (1.2,Inna)}{Rozkład płci respondentów ankiety StackOverflow}{gender}{6cm}

% %work
%\item employment status (rodzaj zatrudnienia)
Na rysunku \ref{fig:employmentStatus} przedstawiono rozkład form zatrudnienia w~oparciu o~które pracują respondenci.
Niemal 80\% respondentów zdecydowało się na podjęcie stosunku pracy w~pełnym wymiarze godzin ze swoimi pracodawcami.


\bargraph[\workbasedon{\cite{so-survey-2020}}]{{Bezrobotny nie szukąjacy pracy}, {Bezrobotny szukający pracy}, Student, {Freelancer}, {Niepełen etat}, {Pełen etat} }{(82.8,{Pełen etat}) (3.1,{Niepełen etat}) (9.5,{Freelancer}) (2.1,Student) (2.1,{Bezrobotny szukający pracy}) (0.3,{Bezrobotny nie szukający pracy})}{Rozkład rodzaju zatrudnienia respondentów ankiety StackOverflow}{employmentStatus}{7cm}

Rozkład rozmiaru firm w~których pracują respondenci został przedstawiony na rysunku \ref{fig:companySize}. Najwięcej jest średnich firm (od 20 do 500 pracowników) i~dużych korporacji powyżej 10 tysięcy pracowników.


\bargraph[\workbasedon{\cite{so-survey-2020}}]{1, 2-9, 10-19, 20-99, 100-499, 500-999, 1000-4999, 5000-9999, +10000}{(4.9,1) (9.9,2-9) (9.3,10-19) (21.6,20-99) (18.7,100-499) (6.5,500-999) (11,1000-4999) (4.1,5000-9999) (13.9,+10000)}{Rozkład rozmiaru firm, w~których pracują respondenci ankiety StackOverflow}{companySize}{9cm}

Rysunek \ref{fig:jobSatisfaction} pokazuje, że około 63\% respondentów jest zadowolona lub bardzo zadowolona z~wykonywanej pracy, a~jedynie około 24\% respondentów jest niezadowolona lub bardzo niezadowolona z~pracy.
Porównując to z~rysunkiem \ref{fig:lookingForJob} obrazującym rozkład statusu szukania pracy widać korelację odsetka niezadowolonych respondentów z~odsetkiem respondentów aktywnie poszukujących pracy.
Ciekawe natomiast jest, że aż 57,6\% respondentów deklaruje, że mimo iż nie szuka aktywnie pracy, to jest otwarta na interesujące propozycje.
Pokazuje to, że nawet zadowoleni z~pracy pracownicy mogą być otwarci na zmianę pracy jeśli nowa oferta spotka się z~ich zainteresowaniem.


\bargraph[\workbasedon{\cite{so-survey-2020}}]{{Bardzo niezadowolony},{Niezadowolony}, {Neutralny}, {Zadowolony}, {Bardzo zadowolony},}{(32.3,{Bardzo zadowolony}) (30.8,{Zadowolony}) (12.8,{Neutralny}) (15.8,{Niezadowolony}) (8.3,{Bardzo niezadowolony})}{Rozkład zadowolenia z~pracy respondentów ankiety StackOverflow}{jobSatisfaction}{6cm}


\bargraph[\workbasedon{\cite{so-survey-2020}}]{{Nie szuka pracy i~nie jest otwarty na propozycje}, {Nie szuka pracy, ale jest otwarty na propozycje}, {Aktywnie szuka pracy}}{(17.3,{Aktywnie szuka pracy}) (57.6,{Nie szuka pracy, ale jest otwarty na propozycje}) (25.1,{Nie szuka pracy i~nie jest otwarty na propozycje})}{Rozkład statusu szukania pracy wśród respondentów ankiety StackOverflow}{lookingForJob}{6cm}


\section{Trendy na rynku pracy}

Branża informatyczna jest bardzo dynamicznie rozwijającą się branżą.
Wiele trendów obserwowanych na rynku pracy związanych jest z~rozwojem technologii, w~tym przede wszystkim sztucznej inteligencji, urządzeń mobilnych czy wirtualnej rzeczywistości.
Wraz z~nowymi technologiami pojawia się większa potrzeba specjalizacji, w~efekcie czego stanowisko "inżyniera oprogramowania" jest zastępowane bardziej konkretnymi stanowiskami takimi jak "programista aplikacji mobilnych na system iOS", czy "programista aplikacji webowych w~technologii Java" \cite{it-polyglots-2015}.

Od 2020 roku sytuacja związana z~pandemią koronawirusa miała istotny wpływ na wyłonienie się nowych trendów w~branży IT.
Na początku pandemii wiele firm zdecydowało się ograniczyć lub wstrzymać rekrutację, a~czasem nawet zwolnić niektórych pracowników, jednak takie działania okazały się nierozsądne, gdyż wbrew obawom zapotrzebowanie na usługi informatyczne nie zmalało, a~wręcz wzrosło w~trakcie pandemii.
Przez nałożone przez rząd ograniczenia firmy zostały zmuszone do oferowania w~większym zakresie usług zdalnie, co przyczyniło się do przyspieszenia transformacji cyfrowej \cite{it-covid-2021}.

Sytuacja pandemiczna pokazała również, że branża IT jest w~stanie bardzo dobrze dostosować się do pracy zdalnej i~nawet przeprowadzanie rekrutacji czy wdrażania nowych pracowników w~formie zdalnej stało się branżową normą.
Mniejsze znaczenie zaczęła mieć lokalizacja siedziby firmy, stąd też pracownicy coraz chętniej decydują się na pracę dla zagranicznego pracodawcy, gdzie często mogą liczyć na korzystniejsze warunki finansowe niż w~przypadku pracy dla polskiej firmy \cite{it-covid-2021}.

Na rysunku \ref{fig:pracaZdalna} pokazano udział ofert pracy zdalnej w~stosunku do wszystkich ofert pracy w~kolejnych kwartałach od 2020 do 2022 roku.
W ciągu zaledwie 2 lat liczba ta wzrosła od 14\% aż do 73\%.
Jednak liczby te nie są jedynie odzwierciedleniem sytuacji pandemicznej. Według badań przeprowadzonych we wrześniu 2021 roku przez firmę DataArt, aż 64\% ankietowanych programistów byłoby gotowych złożyć wypowiedzenie, jeśli pracodawca postanowi narzucić im powrót do pracy stacjonarnej z~biura \cite{pasterczyk-2021}.
% wykres na podstawie https://antyweb.pl/praca-zdalna-w-it


\bargraph[\workbasedon{\cite{ulan-2022}}]{{Q1 2020}, {Q2 2020}, {Q3 2020}, {Q4 2020}, {Q1 2021}, {Q2 2021}, {Q3 2021}, {Q4 2021}, {Q1 2022}}{(14,{Q1 2020}) (23,{Q2 2020}) (43,{Q3 2020}) (49,{Q4 2020}) (46,{Q1 2021}) (57,{Q2 2021}) (64,{Q3 2021}) (70,{Q4 2021}) (73,{Q1 2022})}{Udział ofert pracy zdalnej w~stosunku do wszystkich ofert pracy}{pracaZdalna}{9cm}

Pandemia koronawirusa nie jest jedynym istotnym w~skali świata wydarzeniem mającym wpływ na sytuację rynku pracy branży IT.
Z powodu rosyjskiej inwazji na Ukrainę w~2022 roku wiele firm postanowiło lub zostało zmuszonych do zawieszenia działalności swoich oddziałów w~Ukrainie, Rosji i~Białorusi \cite{blaszczak-2022}/
Kraje te były w~ostatnich latach popularnym kierunkiem outsourcingu usług ze względu na stosunkowo niski koszt zatrudnienia pracowników w~porównaniu do krajów zachodnich oraz dużą liczbę wysoko wykwalifikowanych specjalistów.
Eksperci z~firmy Everest Group oceniają, że ograniczenia działalności we wspomnianych krajach może spowodować wzrost zainteresowania Polską, Węgrami i~Rumunią jako kierunkiem outsourcingu \cite{overby-2022}.
W Polsce częściowo może to znaleźć odzwierciedlenie w~liczbie ofert pracy.
Według analizy Adecco Poland w~lutym 2022 roku liczba ogłoszeń była prawie o~połowę większa w~stosunku do grudnia 2021 roku \cite{blaszczak-2022}.

\thispagestyle{normal}
