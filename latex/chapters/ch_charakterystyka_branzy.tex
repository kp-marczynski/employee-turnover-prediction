% !TeX spellcheck = pl_PL


\chapter{Charakterystyka branży IT}\label{ch:it-sector}


\section{Ogólna charakterystyka}\label{sec:it-industry-summary}

Branża IT jest stosunkowo młoda.
Konepcja stanowiąca podstawę architektury wszystkich współczesnych komputerów - a ogólniej w zasadzie niemal wszystkich urządzeń korzystających z procesora - została opracowana w 1945 roku przez Johna von Neumanna, Johna Mauchly’ego i Johna Eckerta \cite{von-neumann-2020}.
Model komputera implementującego tę architekturę nazwali "przykładową maszyną cyfrową".
Na przestrzeni lat wraz z postępem technologicznym następował stopniowy wzrot znaczenia technologii informatycznych,
jednak prawdziwy rozkwit branży nastąpił po uruchomieniu w 1993 roku ogólnoświatowej sieci komputerowej (ang. World Wide Web) \cite{rangarajan-2014}, którą potocznie nazywamy "Internetem".

Rok 1993 można uznać za symboliczną datę kiedy rynek pracy IT zaczął przybierać znaną obecnie postać, jednak warto wyszczególnić jeszcze kilka ważnych dat z historii rozwoju branży:
\begin{itemize}
    \item 2002 - Amazon Web Services - pierwsza implementacja chmury obliczeniowej \cite{aws-2017},
    \item 2004 - Web 2.0 - ewolucja sposobu kreowania treści w internecie w sposób bardziej angażujący użytkowników \cite{web-2.0-2007},
    \item 2007 - iPhone - początek ery urządzeń mobilnych \cite{iphone-2018}.
\end{itemize}

W Polsce branża IT stanowi bardzo istotny sektor gospodarki.
W 2021 roku działalność w branży IT stanowiła w przybliżeniu 8\% PKB Polski \cite{piih-2021}.
Według raportu HackerRank Polscy programiści znajdują się w światowej czołówce pod względem umiejętności, plasując się na trzecim miejscu, zaraz za Chinami i Rosją \cite{hackerrank-2021}.
Wysokie kwalifikacje polskich programistów są jedną z przyczyn dlaczego wiele zagranicznych firm decyduje się na outsourcing projektów do polskich firm informatycznych.
Wśród innych powodów wybierania Polski do procesu outsourcingu wyzczególnić można \cite{softwaremind-2021}:
\begin{itemize}
    \item wysoki stopień znajomości języka angielskiego,
    \item przynależność Polski do Unii Europejskiej zapewnia zgodność z dyrektywami RODO dotyczącymi nie przetwarzania danych użytkowników poza terenem Unii Europejskiej,
    \item Polska jest zbliżona kulturowo i lokalizacyjnie do zachodnich zagranicznych inwestorów z krajów takich jak Niemcy czy Wielka Brytania, a nawet Stany Zjednoczone ()gdyż różnica czasu między Polską, a wschodnim wybrzeżem Stanów Zjednoczonych wynosi zaledwie 6 godzin).
\end{itemize}

Duże zainteresowanie zagranicznych inwestorów polskim rynkiem IT oraz ogólny światowy niedobór wykwalifikowanych programistów \cite{daxx-2021}
kreują sytuację, w której pracodawcy prześcigają się w przygotowywaniu coraz atrakcyjniejszych ofert pracy zawierających szereg benefitów pracowniczych \cite{it-benefits-2020}.
Tak powstała subiektywnie postrzegana łatwość zmiany pracy może skutkować zwiększeniem fluktuacji pracowników.
Według raportu LinkedIn \cite{linkedin-2018} w 2017 roku branża IT odnotowała najwyższy wskaźnik fluktuacji wynoszący 13.2\%, gdzie globalna średnia dla innych zawodów wynosiła zaledwie 10.9\%.
Szczególnie wysokie wskaźniki fluktuacji odnotowano dla następujących specjalizacji w branży IT:
\begin{itemize}
    \item 23.3\% - projektant doświadczeń użytkownika,
    \item 21.7\% - analityk danych,
    \item 21.7\% - programista systemów wbudowanych.
\end{itemize}

Jak wspomniano w rozdziale \ref{sec:wplyw-fluktuacji-na-firme}, wysoki poziom fluktuacji niekoniecznie musi oznaczać problemy dla firmy, jeśli dostępnych jest wielu potencjalnych pracowników, a koszty rekrutacji są niskie.
Te warunki nie są jednak spełnione w przypadku branży IT, gdzie wydatki związane z zatrudnieniem nowego pracownika po odejściu starego mogą sięgać nawet rzędu 200 tys. polskich złotych \cite{hairing-dev-2021}.

%\section{Wysoka fluktuacja}\label{sec:it-turnover}
%\todo{niedobór programistów a~wysoki popyt na ekspertów}
%
%Przyczyny odejść specjalistów na stanowiskach technicznych \cite{kapor-2017}.
%
%Przyczyny odejść programistów \cite{hannon-2008}.


\section{Zadowolenie pracowników}\label{sec:it-motivation}
Jak pokazano w rozdziale \ref{sec:czynniki-wplywajace-na-fluktuacje} wiele z klasycznych teorii badających czynniki wpływające na fluktuację wyróżnia zadowolenie z pracy jako jeden z kluczowych czynników.
W przypadku branży IT zadowolenie z pracy wpływa również znacząco na produktywność programistów i przede wszystkim na jakość tworzonego oprogramowania \cite{graziotin-2018}.
Według Roberta C. Martina - autorytetu w dziedzinie stosowania dobrych wzorców w programowaniu - jakość kodu ma kluczowe znaczenie dla przyszłości tworzonego oprogramowania, a także dla przyszłości firmy.
Martin wyszczególnił następujące skutki zaniechania dbałości o jakość oprogramowania \cite{martin-2014}:
\begin{itemize}
    \item wydłużenie czasu dostarczania nowych funkcjonalności oprogramowania,
    \item częstsze pojawianie się awarii oprogramowania,
    \item spadek produktywności programistów,
    \item spadek motywacji programistów,
    \item konieczność poświęcania większej ilości czasu na analizowanie istniejącego kodu,
    \item zwiększenie kosztów tworzenia oprogramowania,
    \item wydłużenie czasu trwania projektu,
    \item możliwa konieczność napisania całego kodu od nowa.
\end{itemize}

Często stosowanym sposobem zwiększenia zadowolenia pracowników jest ograniczanie niezadowolenia.
Czynniki wpływające na niezadowlenie pracowników w branży IT można podzielić na 3 podstawowe kategorie \cite{graziotin-2017}:
\begin{enumerate}
    \item wewnętrzne związane z osobą pracownika
    \item zewnętrzne związane z organizacją firmy
    \item zewnętrzne związane z oprogramowaniem
\end{enumerate}

W tabeli \ref{tabela:niezadowolenie-IT} przedstawiono częste przyczyny niezadowlenia pracowników z uwzględnieniem powyższych kategorii.

\noindent\begin{minipage}{\textwidth}
             \begin{table}[H]
                 \raggedright\caption{Przyczyny niezadowolenia pracowników w branży IT\label{tabela:niezadowolenie-IT}}
                 \begin{center}
                     \begin{tabular}{|P{.2\textwidth}|P{.7\textwidth}|}

                         \hline
                         \cellgray{Kategoria} &
                         \cellgray{Przyczyny} \\
                         \hline

                         czynniki wewnętrzne związane z osobą pracownika &
                         \begin{itemize}
                             \item fiksacja na punkcie rozwiązywania problemów,
                             \item syndrom impostora - poczucie nie posiadania wystarczających kompetencji do pełnionej roli,
                             \item problemy osobiste;
                         \end{itemize} \\

                         \hline

                         czynniki zewnętrzne związane z organizacją firmy &
                         \begin{itemize}
                             \item presja czasu,
                             \item monotonne lub powtarzające się zadania,
                             \item podejmowanie złych decyzji (zwłaszcza technicznych) na wyższych szczeblach w organizacji;
                         \end{itemize} \\

                         \hline

                         czynniki zewnętrzne związane z oprogramowaniem &
                         \begin{itemize}
                             \item niska jakość kodu i złe praktyki programistyczne,
                             \item praca z kodem, który nie działa z niewiadomych przyczyn,
                             \item narzucone ograniczenia dotyczące narzędzi wykorzystywanych do programowania (takiej jak język programowania czy środowisko);
                         \end{itemize} \\

                         \hline

                     \end{tabular}
                 \end{center}
                 \raggedright\source{\workbasedon{\cite{graziotin-2017}}}
                 \vspace{0.75cm}
             \end{table}
\end{minipage}
%Pracownicy zadowoleni a zmotywowani \cite{sharp-2014}.
%
%Satysfakcja i produktywność \cite{storey-2021}.
%
%Wpływ środowiska pracy na produktywność i satysfakcję \cite{johnson-2021}.
%
%Czynniki wpływające na retencję \cite{bass-2018}.
%
%Konsekwencje zadowolenia i niezadowolenia pracowników \cite{graziotin-2018}.
%
%O niezadowoleniu pracowników \cite{graziotin-2017}.


\section{Wpływ fluktuacji na utratę wiedzy w~projektach informatycznych}\label{sec:it-knowledge-loss}

Jednym z aspektów związanych z fluktuacją pracowników w branży IT jest utrata wiedzy w projektach informatycznych.
Można wyszczególnić kilka podstawowych przyczyn utraty wiedzy \cite{https://hellonesh.io/employee-turnover-knowledge-loss-lets-change-the-equation/}:
\begin{itemize}
    \item nieudokumentowane dane - wiele decyzji projektowych podejmowanych jest w oparciu o doświadczenie pracownika, testowanie wielu rozwiązań i wyciąganie wniosków z sukcesów i porażek - powstaje w ten sposób rodzaj "wiedzy plemiennej", która może być w pełni zrozumiała dla obecnych członków zespołu, ale nie jest w całości przekazywana nowym pracownikom,
    \item udokumentowane, ale nieuporządkowane dane - wiedza projektowa często jest spisywana w formie różnorakich dokumentów, które funkcjonują w obiegu między pracownikami, jednak w przypadku braku uporządkowanej struktury bazy wiedzy projektowej dokumenty takie mogą łatwo zaginąć w przypadku odejścia pracownika,
    \item relacje międzyludzkie - decyzje projektowe mogą być podejmowane na podstawie wymagań klienta; w wielu projektach informatycznych wymagania klienta nie są udokumentowane w formalnych dokumentach przygotowanych przez analityków biznesowych, ale są przekazywane w formie ustnej lub mailowej, z tego powodu w przypadku odejścia pracownika wiedza wynikająca z relacji z klientem jest często trudna do przekazania.
\end{itemize}

Donadelli \cite{donadelli-2015} wyszczególnia dodatkowe czynniki związane z wysoką fluktuacją w projektach informatycznych:
\begin{itemize}
    \item spadek produktywności firmy,
    \item obniżenie jakości wytwarzanego kodu,
    \item zwiększenie liczby "porzuconych" plików (tj. takich, które od dawna nie były edytowane przez żadnego z członków zespołu),
    \item ograniczenie rozprzestrzeniania wiedzy w projekcie.
\end{itemize}

Na szczególną uwagę zasługuje ostatni wymieniony punkt.
W zwinnych metodykach zarządzania projektami stosuje się "wskaźnik autobusu", który dosłownie oznacza wskazanie ilu członków zespołu musiałoby zostać potrąconych przez autobus, aby w znaczący sposób wpłynęło to negatywnie na przyszłość prowadzonego projektu \cite{https://www.agileadvice.com/2005/05/15/agilemanagement/truck-factor/}.
Wskaźnik autobusu zwykle jest szczególnie istotny w projektach prowadzonych przez małe zespoły, gdzie każdy z członków zespołu zajmuje się pracą nad odrebną częścią systemu.
W takich projektach odejście pojedynczego pracownika może doprowadzić do utraty znaczącej części wiedzy.
Sytuacja zwykle inaczej się prezentuje w przypadku dużych projektów, gdzie wiedza jest znacząco bardziej rozproszona i odejście pracownika nie wpływa istotnie na dalsze prowadzenie projektu.

%Przekazywanie kodu a produktywność programistów \cite{mockus-2009}.
%
%Negatywny wpływ rotacji na jakość kodu \cite{donadelli-2015}.
%
%Unikanie negatywnego wpływu rotacji na utratę wiedzy \cite{rigby-2016}.


%\section{Zarządzanie zasobami ludzkimi}\label{sec:it-project-management}
%\todo{koszty fluktuacji w~związku z~rekrutacją}
%
%Zaangażowanie organizacyjne pracowników \cite{rosinski-2012}.
%
%Zarządzanie kapitałem ludzkim \cite{brylka-2019}.


\section{Trendy na rynku pracy}

\begin{itemize}
    \item Wzrost liczby firm IT dla których niedobór pracowników jest barierą w biznesie \cite{https://www.rp.pl/rynek-pracy/art35856021-wojna-nasili-niedobor-kadr-w-it-naplyw-nowych-zlecen-do-polski}
    \item 50\% więcej ogłoszeń o pracę niż w 2021\cite{https://www.rp.pl/rynek-pracy/art35856021-wojna-nasili-niedobor-kadr-w-it-naplyw-nowych-zlecen-do-polski}
    \item problemy outsourcingu w związku z wojną na Ukrainie \cite{https://www.computerworld.pl/news/Wojna-w-Ukrainie-stawia-globalny-outsourcing-IT-w-trudnej-sytuacji,437151.html}
    \item przyspieszenie transformacji cyfrowej,  wzrost zapotrzebowania na systemy informatyczne przez pandemię \cite{https://itwiz.pl/sytuacja-na-rynku-pracy-it-w-dobie-koronawirusa/}
    \item zwiększenie znaczenia pracy zdalnej \cite{https://antyweb.pl/praca-zdalna-w-it}
    \item więcej hajsu na pracy zdalnej \cite{https://cyberdefence24.pl/biznes-i-finanse/polscy-pracodawcy-nie-chca-pracy-zdalnej-branza-it-zachowuje-elastycznosc}
    \item powrót do biur => wypowiedzenie \cite{https://geek.justjoin.it/przymusowy-powrot-do-biura-z-home-office-w-branzy-it-64-proc-wrociloby-z-wypowiedzeniem}
    \item
\end{itemize}
\todo{praca zdalna}
\todo{praca hybrydowa}
\todo{wojna}


\section{Charakterystyka pracowników w branży IT}

W niniejszym podrozdziale przedstawiona zostanie demografia pracowników z branży informatycznej.
Opracowanie zostało przygotowane na podstawie danych z ankiety deweloperskiej przeprowadzonej przez StackOverflow w 2020 roku, w której wzięło udział około 65 tysięcy programistów z całego świata \cite{so-survey-2020}.

Na rysunku \ref{fig:firstLineOfCode} pokazano w jakim wieku respondenci napisali swoją pierwszą linijkę kodu.
Na wykresie widać, że przeważająca większość respondentów - aż 54\% - miała pierwszy kontakt z programowaniem już przed 16 rokiem życia.
Jednakże, wczesny kontakt z programowaniem nie przekłada się wśród respondentów na długi staż zawwodowy, gdyż jak widać na rysunku \ref{fig:yearsCodingProf} około 65\% badanych osób ma mniej niż 10 lat doświadczenai zawodowego.
Można domniemywać, że w związku ze wzrostem znaczenia branży IT w ciągu ostatnich kilkkudziesięciu lat oraz wzrostem zapotrzebowania na specjalistów IT na rynku pracy wzrasta zainteresowanie branżą IT wśród młodszych pokoleń.
Obrazować może to fakt, że aż 70\% respondentów to osoby w wieku poniżej 35 lat, co pokazuje rysunek \ref{fig:age}.

\bargraph[\workbasedon{\cite{so-survey-2020}}]{{Poniżej 10 lat}, {10-11 lat}, {12-13 lat}, {14-15 lat}, {16-17 lat}, {18-19 lat}, {20-21 lat}, {22-23 lat}, {24-25 lat}, {26-27 lat}, {28-29 lat}, {Powyżej 30 lat}}{(8.9,{Poniżej 10 lat}) (10,{10-11 lat}) (16,{12-13 lat}) (19.2,{14-15 lat}) (16.3,{16-17 lat}) (14.7,{18-19 lat}) (6.3,{20-21 lat}) (3,{22-23 lat}) (2.1,{24-25 lat}) (1,{26-27 lat}) (0.7,{28-29 lat}) (1.7,{Powyżej 30 lat})}{Wiek, w którym respondenci napisali pierwszą linię kodu}{firstLineOfCode}{9cm}


\bargraph[\workbasedon{\cite{so-survey-2020}}]{{Poniżej 5 lat}, {5-9 lat}, {10-14 lat}, {15-19 lat}, {20-24 lat}, {25-29 lat}, {30-34 lat}, {35-39 lat}, {40-44 lat}, {45-49 lat}, {Powyżej 50 lat}}{(39.6,{Poniżej 5 lat}) (26.8,{5-9 lat}) (14.7,{10-14 lat}) (7.6,{15-19 lat}) (6,{20-24 lat}) (2.4,{25-29 lat}) (1.6,{30-34 lat}) (0.8,{35-39 lat}) (0.4,{40-44 lat}) (0.1,{45-49 lat}) (0.1,{Powyżej 50 lat})}{Staż pracy w latach}{yearsCodingProf}{9cm}


\bargraph[\workbasedon{\cite{so-survey-2020}}]{{15-19 lat}, {20-24 lat}, {25-29 lat}, {30-34 lat}, {35-39 lat}, {40-44 lat}, {45-49 lat}, {50-54 lat}, {55-59 lat}, {Powyżej 60 lat}}{(1.2,{15-19 lat}) (16.6,{20-24 lat}) (29.5,{25-29 lat}) (21.9,{30-34 lat}) (14,{35-39 lat}) (7.5,{40-44 lat}) (4.2,{45-49 lat}) (2.5,{50-54 lat}) (1.5,{55-59 lat}) (1.1,{Powyżej 60 lat})}{Wiek}{age}{9cm}

Na rysunku \ref{fig:education} pokazano jakie najwyższe wykształcenie uzyskali respondenci ankiety. Ciekawe jest, że mimo iż firmy z branży IT zwykle nie wymagają od pracowników formalnego wykształcenia, aż 75\% badanych osób ma ukończone co najmniej studia I stopnia.

\bargraph[\workbasedon{\cite{so-survey-2020}}]{{Brak formalnego wykształcenia}, {Szkoła podstawowa}, {Szkoła gimnazjalna}, {Szkoła licealna}, {Szkoła policealna}, {Szkoła zawodowa}, {Studia I stopnia}, {Studia II stopnia}, {Studia doktorskie}}{(0.7,{Brak formalnego wykształcenia}) (0.5,{Szkoła podstawowa}) (4.5,{Szkoła gimnazjalna}) (11.5,{Szkoła licealna}) (3.2,{Szkoła policealna}) (1.4,{Szkoła zawodowa}) (49.3,{Studia I stopnia}) (25.5,{Studia II stopnia}) (3.3,{Studia doktorskie})}{Wykształcenie}{education}{9cm}

Na rysunku \ref{fig:gender} przedstawiono rozkład płci respondentów i jak widać branża IT jest silnie zdominowana przez mężczyzn (ponad 90\% respondentów).
\bargraph[\workbasedon{\cite{so-survey-2020}}]{Meżczyzna, Kobieta, Inna}{(91.2,Meżczyzna) (7.6,Kobieta) (1.2,Inna)}{Płeć}{gender}{6cm}

% %work
%\item employment status (rodzaj zatrudnienia)
\bargraph[\workbasedon{\cite{so-survey-2020}}]{{Pełen etat}, {Niepełen etat}, {Freelancer}, Student, {Bezrobotny szukajacy pracy}, {Bezrobotny nie szukajacy pracy}}{(82.8,{Pełen etat}) (3.1,{Niepełen etat}) (9.5,{Freelancer}) (2.1,Student) (2.1,{Bezrobotny szukajacy pracy}) (0.3,{Bezrobotny nie szukajacy pracy})}{Rodzaj zatrudnienia}{employmentStatus}{7cm}

%\item rozmiar firmy
\bargraph[\workbasedon{\cite{so-survey-2020}}]{1, 2-9, 10-19, 20-99, 100-499, 500-999, 1000-4999, 5000-9999, +10000}{(4.9,1) (9.9,2-9) (9.3,10-19) (21.6,20-99) (18.7,100-499) (6.5,500-999) (11,1000-4999) (4.1,5000-9999) (13.9,+10000)}{Liczba pracowników}{companySize}{9cm}

%\item zadowolenie z pracy
\bargraph[\workbasedon{\cite{so-survey-2020}}]{{Bardzo zadowolony}, {Zadowolony}, {Neutralny}, {Niezadowolony}, {Bardzo niezadowolony}}{(32.3,{Bardzo zadowolony}) (30.8,{Zadowolony}) (12.8,{Neutralny}) (15.8,{Niezadowolony}) (8.3,{Bardzo niezadowolony})}{Zadowolenie z pracy}{jobSatisfaction}{6cm}

%\item chęć zmiany pracy
\bargraph[\workbasedon{\cite{so-survey-2020}}]{{Aktywnie szuka pracy}, {Nie szuka pracy, ale jest otwarty na propozycje}, {Nie szuka pracy i nie jest otwarty na propozycje}}{(17.3,{Aktywnie szuka pracy}) (57.6,{Nie szuka pracy, ale jest otwarty na propozycje}) (25.1,{Nie szuka pracy i nie jest otwarty na propozycje})}{Status szukania pracy}{lookingForJob}{6cm}

% \item najważniejsze rzeczy w pracy
% \bargraph[\workbasedon{\cite{so-survey-2020}}]{XXX}{Salaries, Capital, Loans, Taxes}{(15,Salaries) (16,Capital) (52,Loans) (16,Taxes)}{tytuł}{figLabel}

\thispagestyle{normal}
