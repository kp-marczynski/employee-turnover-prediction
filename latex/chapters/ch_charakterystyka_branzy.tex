% !TeX spellcheck = pl_PL


\chapter{Charakterystyka branży IT}\label{ch:it-sector}


\section{Ogólna charakterystyka}\label{sec:it-industry-summary}

Branża IT jest stosunkowo młoda.
Konepcja stanowiąca podstawę architektury wszystkich współczesnych komputerów - a ogólniej w zasadzie niemal wszystkich urządzeń korzystających z procesora - została opracowana w 1945 roku przez Johna von Neumanna, Johna Mauchly’ego i Johna Eckerta \cite{von-neumann-2020}.
Model komputera implementującego tę architekturę nazwali "przykładową maszyną cyfrową".
Na przestrzeni lat wraz z postępem technologicznym następował stopniowy wzrot znaczenia technologii informatycznych,
jednak prawdziwy rozkwit branży nastąpił po uruchomieniu w 1993 roku ogólnoświatowej sieci komputerowej (ang. World Wide Web) \cite{rangarajan-2014}, którą potocznie nazywamy "Internetem".

Rok 1993 można uznać za symboliczną datę kiedy rynek pracy IT zaczął przybierać znaną obecnie postać, jednak warto wyszczególnić jeszcze kilka ważnych dat z historii rozwoju branży:
\begin{itemize}
    \item 2002 - Amazon Web Services - pierwsza implementacja chmury obliczeniowej \cite{aws-2017},
    \item 2004 - Web 2.0 - ewolucja sposobu kreowania treści w internecie w sposób bardziej angażujący użytkowników \cite{web-2.0-2007},
    \item 2007 - iPhone - początek ery urządzeń mobilnych \cite{iphone-2018}.
\end{itemize}

W Polsce branża IT stanowi bardzo istotny sektor gospodarki.
W 2021 roku działalność w branży IT stanowiła w przybliżeniu 8\% PKB Polski \cite{piih-2021}.
Według raportu HackerRank Polscy programiści znajdują się w światowej czołówce pod względem umiejętności, plasując się na trzecim miejscu, zaraz za Chinami i Rosją \cite{hackerrank-2021}.
Wysokie kwalifikacje polskich programistów są jedną z przyczyn dlaczego wiele zagranicznych firm decyduje się na outsourcing projektów do polskich firm informatycznych.
Wśród innych powodów wybierania Polski do procesu outsourcingu wyzczególnić można \cite{softwaremind-2021}:
\begin{itemize}
    \item wysoki stopień znajomości języka angielskiego,
    \item przynależność Polski do Unii Europejskiej zapewnia zgodność z dyrektywami RODO dotyczącymi nie przetwarzania danych użytkowników poza terenem Unii Europejskiej,
    \item Polska jest zbliżona kulturowo i lokalizacyjnie do zachodnich zagranicznych inwestorów z krajów takich jak Niemcy czy Wielka Brytania, a nawet Stany Zjednoczone ()gdyż różnica czasu między Polską, a wschodnim wybrzeżem Stanów Zjednoczonych wynosi zaledwie 6 godzin).
\end{itemize}

Duże zainteresowanie zagranicznych inwestorów polskim rynkiem IT oraz ogólny światowy niedobór wykwalifikowanych programistów \cite{daxx-2021}
kreują sytuację, w której pracodawcy prześcigają się w przygotowywaniu coraz atrakcyjniejszych ofert pracy zawierających szereg benefitów pracowniczych \cite{it-benefits-2020}.
Tak powstała subiektywnie postrzegana łatwość zmiany pracy może skutkować zwiększeniem fluktuacji pracowników.
Według raportu LinkedIn \cite{linkedin-2018} w 2017 roku branża IT odnotowała najwyższy wskaźnik fluktuacji wynoszący 13.2\%, gdzie globalna średnia dla innych zawodów wynosiła zaledwie 10.9\%.
Szczególnie wysokie wskaźniki fluktuacji odnotowano dla następujących specjalizacji w branży IT:
\begin{itemize}
    \item 23.3\% - projektant doświadczeń użytkownika,
    \item 21.7\% - analityk danych,
    \item 21.7\% - programista systemów wbudowanych.
\end{itemize}

Jak wspomniano w rozdziale \ref{sec:wplyw-fluktuacji-na-firme}, wysoki poziom fluktuacji niekoniecznie musi oznaczać problemy dla firmy, jeśli dostępnych jest wielu potencjalnych pracowników, a koszty rekrutacji są niskie.
Te warunki nie są jednak spełnione w przypadku branży IT, gdzie wydatki związane z zatrudnieniem nowego pracownika po odejściu starego mogą sięgać nawet rzędu 200 tys. polskich złotych \cite{hairing-dev-2021}.

\section{Wysoka fluktuacja}\label{sec:it-turnover}
\todo{niedobór programistów a~wysoki popyt na ekspertów}

Przyczyny odejść specjalistów na stanowiskach technicznych \cite{kapor-2017}.

Przyczyny odejść programistów \cite{hannon-2008}.


\section{Zadowolenie pracowników}\label{sec:it-motivation}
Jak pokazano w rozdziale \ref{sec:czynniki-wplywajace-na-fluktuacje} wiele z klasycznych teorii badających czynniki wpływające na fluktuację wyróżnia zadowolenie z pracy jako jeden z kluczowych czynników.
W przypadku branży IT zadowolenie z pracy wpływa również znacząco na produktywność programistów i przede wszystkim na jakość tworzonego oprogramowania \cite{graziotin-2018}.
Według Roberta C. Martina - autorytetu w dziedzinie stosowania dobrych wzorców w programowaniu - jakość kodu ma kluczowe znaczenie dla przyszłości tworzonego opragrmowania, a także dla przyszłości firmy.
Martin wyszczególnił następujące skutki zaniechania dbałości o jakość oprogramowania \cite{clean-code}:
\begin{itemize}
    \item wydłużenie czasu dostarczania nowych funkcjonalności oprogramowania,
    \item częstsze pojawianie się awarii oprogramowania,
    \item spadek produktywności programistów,
    \item spadek motywacji programistów,
    \item konieczność poświęcania większej ilości czasu na analizowanie istniejącego kodu,
    \item zwiększenie kosztów tworzenia oprogramowania,
    \item wydłużenie czasu trwania projektu,
    \item możliwa konieczność napisania całego kodu od nowa.
\end{itemize}

Pracownicy zadowoleni a zmotywowani \cite{sharp-2014}.

Satysfakcja i produktywność \cite{storey-2021}.

Wpływ środowiska pracy na produktywność i satysfakcję \cite{johnson-2021}.

Czynniki wpływające na retencję \cite{bass-2018}.

Konsekwencje zadowolenia i niezadowolenia pracowników \cite{graziotin-2018}.

O niezadowoleniu pracowników \cite{graziotin-2017}.

%Retencja pracowników w Indiach \cite {james-2012}.


\section{Wpływ fluktuacji na utratę wiedzy w~projektach informatycznych}\label{sec:it-knowledge-loss}

Przekazywanie kodu a produktywność programistów \cite{mockus-2009}.

Negatywny wpływ rotacji na jakość kodu \cite{donadelli-2015}.

Unikanie negatywnego wpływu rotacji na utratę wiedzy \cite{rigby-2016}.


\section{Zarządzanie zasobami ludzkimi}\label{sec:it-project-management}
\todo{koszty fluktuacji w~związku z~rekrutacją}

Zaangażowanie organizacyjne pracowników \cite{rosinski-2012}.

Zarządzanie kapitałem ludzkim \cite{brylka-2019}.

\thispagestyle{normal}
