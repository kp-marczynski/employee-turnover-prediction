% !TeX spellcheck = pl_PL
\chapter{Charakterystyka branży IT}\label{ch:it-sector}
\section{Ogólna charakterystyka}\label{sec:it-industry-summary}
Wymagania na stanowiska entry-level \cite{aasheim-2009}.

Dostępność specjalistów IT w Polsce \cite{prodata-2020}.

Top 3 najwyższej rotacji \cite{linkedin-2018}.

\section{Wysoka fluktuacja}\label{sec:it-turnover}
\todo{niedobór programistów a~wysoki popyt na ekspertów}

Przyczyny odejść specjalistów na stanowiskach technicznych \cite{kapor-2017}.

Przyczyny odejść programistów \cite{hannon-2008}.

\section{Zadowolenie pracowników}\label{sec:it-motivation}
Pracownicy zadowoleni a zmotywowani \cite{sharp-2014}.

Satysfakcja i produktywność \cite{storey-2021}.

Wpływ środowiska pracy na produktywność i satysfakcję \cite{johnson-2021}.

Czynniki wpływające na retencję \cite{bass-2018}.

Konsekwencje zadowolenia i niezadowolenia pracowników \cite{graziotin-2018}.

O niezadowoleniu pracowników \cite{graziotin-2017}.

%Retencja pracowników w Indiach \cite {james-2012}.

\section{Wpływ fluktuacji na utratę wiedzy w~projektach informatycznych}\label{sec:it-knowledge-loss}

Przekazywanie kodu a produktywność programistów \cite{mockus-2009}.

Negatywny wpływ rotacji na jakość kodu \cite{donadelli-2015}.

Unikanie negatywnego wpływu rotacji na utratę wiedzy \cite{rigby-2016}.

\section{Zarządzanie zasobami ludzkimi}\label{sec:it-project-management}
\todo{koszty fluktuacji w~związku z~rekrutacją}

Zaangażowanie organizacyjne pracowników \cite{rosinski-2012}.

Zarządzanie kapitałem ludzkim \cite{brylka-2019}.

\thispagestyle{normal}
