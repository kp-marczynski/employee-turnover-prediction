% !TeX spellcheck = pl_PL
\chapter{Optymalizacja procesu rekrutacji w~branży IT z~wykorzystaniem uczenia maszynowego}\label{ch:analysis}
\section{Metody badawcze}\label{sec:analysis-method}
Biorąc pod uwagę rosnące zapotrzebowanie na usługi informatyczne, deficyt wykwalifikowanych pracowników \cite{daxx-2021} oraz wysoki koszt rekrutacji nowych pracowników \cite{hairing-dev-2021} istotne jest poszukiwanie sposobów na zwiększenie retencji pracowników IT.
Badania będące przedmiotem niniejszej pracy mają na celu analizę cech skorelowanych z zadowoleniem pracowników branży IT i z ich chęcią do zmiany pracy.
Badane cechy zostaną podzielone na 4 kategorie:
\begin{enumerate}
    \item profil firmy (rozmiar firmy)
    \item organizacja firmy (rozmiar zespołów, metodyka zarządzania projektem)
    \item profil pracownika (wiek, płeć, staż pracy, wykształcenie)
    \item preferencje pracownika (technologia, benefity pracownicze)
    \end{enumerate}

Powyższe kategorie dzielą cechy firmy i pracownika na 2 zasadnicze typy:
\begin{enumerate}
    \item cechy profilowe - profil firmy i profil pracownika
    \item cechy zmienne - organizacja firmy i preferencje pracownika
    \end{enumerate}

Cechy profilowe mogą być uznawane za niezmienne w kontekście prowadzonego badania, gdyż ich zmiana może całkowicie zmienić istotne cechy podmiotu
(na przykład w przypadku liczby pracowników zmiana może dotyczyć przekształcenia małej firmy w międzynarodową korporację, a w przypadku wieku pracownika zmiany z młodej osoby w osobę w wieku przedemerytalnym - w obu przypadkach oczekiwania firmy wobec pracowników i vice versa będą inne).
Natomiast cechy zmienne mogą być zmieniane tak aby dostosować je do bieżących potrzeb.

Celem badania jest określenie czy i jeśli tak to jakie cechy profilu firmy i pracownika mogą mieć istotny wpływ na zadowolenie pracownika
oraz czy można wyszczególnić cechy organizacyjne firmy, których modyfikacja mogłaby pozwolić na zwiększenie zadowolenia pracowników.
Określenie takich cech miałoby szczególne znaczenie już na etapie rekrutacji, gdyż mogłoby umożliwić selekcjonowanie kandydatów o profilu "kompatybilnym" z profilem firmy.


Badania zostaną przeprowadzone w oparciu o ankietę developerską StackOverflow \cite{so-survey-info}.
Ankieta ta jest przeprowadzana corocznie od 2011 roku, jednak popularność zyskała dopiero w 2016 roku, kiedy to liczba respondentów biorących udział w badaniu przekroczyła 50 tysięcy.
W latach 2016-2022 liczba respondentów utrzymywała się na poziomie od 50 do 100 tysięcy.
Odpowiedzi w ankiecie udzielali specjaliści z całego świata reprezentując różne grupy wiekowe i kulturowe.
Tak wysoka liczba odpowiedzi i różnorodność respondentów może pozwolić na przeprowadzenie badań ilościowych i umożliwić próbę generalizacji wniosków na całą branżę IT.


Do analizy danych zostaną wykorzystane metody uczenia maszynowego, a w szczególności regresja z wykorzystaniem klasyfikatora XGBoost opartego na algorytmie wzmocnienia gradientowego.
Punnoose i Ajit \cite{punnoose-2016} przeprowadzili badania związane ze skutecznością przewidywania retencji pracowników w firmie z wykorzystaniem uczenia maszynowego.
Klasyfikator XGBoost osiągnął aż 86\% skuteczności klasyfikacji co stanowiło ponad 50\% lepszy wynik niż pozostałe testowane metody klasyfikacji tradycyjnie używane podczas badania problemu fluktuacji i retencji pracowników.

%\section{Wybór źródła danych: Prezentacja ankiety StackOverflow}\label{sec:analysis:data-source-selection}
%\todo{przedstawienie źródła danych, charakterystyka respondentów}
%\section{Wstępna selekcja cech}\label{sec:analysis:feature-pre-selection}
%\todo{z wyszczególnieniem cech stałych/środowiskowych (kraj, typ dewelopera, typ firmy, rozmiar firmy) i~zmiennych (wynagrodzenie, benefity, metodologie, cechy profilu kandydata)}
\section{Wstępne przetworzenie danych}\label{sec:analysis:preprocessing}
Do przeprowadzenia badania wybrano ankiety deweloperskie StackOverflow z lat 2017, 2018, 2019 ze względu na to, że są to 3 najnowsze edycje ankiety w których zadano pytania o następujące 2 aspekty:
\begin{enumerate}
    \item ocena zadowolenia z obecnie wykonywanej pracy w 10 stopniowej skali
    \item status poszukiwania pracy w 3 stopniowej skali
    \end{enumerate}

W celu przygotowania danych do badania konieczne było oczyszczenie i znormalizowanie danych. Proces ten został przeprowadzony w następujący sposób:
\begin{enumerate}
    \item Usunięto pytania, które nie mają istotnego wpływu na badany problem (na przykład pytanie dotyczące preferencji związanej z wymawianiem słowa "GIF").
    \item Uwspólniono nazwy etykiet dotyczących tych samych pytań w kolejnych edycjach ankiety.
    \item Usunięto odpowiedzi respondentów, którzy nie mają doświadczenia zawodowego lub nie udzielili odpowiedzi na pytania dotyczące zadowolenia z pracy i statusu poszukiwania pracy.
    \item Dla pytań z odpowiedzami w skali Likerta (typu od "bardzo się nie zgadzam" do "bardzo się zgadzam") zastosowano przekształcenie na skalę liczbową.
    \item Dla pytań z odpowiedziami reprezentowanymi przez kategorie, które można przedstawić w skali (na przykład edukacja) zastosowano przekształcenie na skalę liczbową.
    \item Dla pytań z odpowiedziami reprezentowanymi przez kategorię, których nie można przedstawić w skali (na przykład technologie) potraktowano każdą kategorię jak osobne pytanie i zastosowano kodowanie binarne (0 - niewystępuje i 1 - występuje).
    \item
    \end{enumerate}
\todo{kodowanie liczbowe, oczyszczanie}
\todo{opcjonalnie wzbogacenie o~wybrane indeksy rozwoju społecznego}
\section{Selekcja cech z~wykorzystaniem algorytmu XGB}\label{sec:analysis:feature-selection-xgb}
\todo{budowa modelu uczenia maszynowego w~oparciu o~cechy istotnie wpływające na predykcję}
\todo{\url{https://medium.com/@s.pranav.harathi/stack-overflow-survey-analysis-ed45127691b}}
\section{Prezentacja wyników}\label{sec:analysis:important-features}
\todo{}
\section{Wnioski i~analiza możliwości praktycznego zastosowania zbudowanego modelu predykcji}\label{sec:analysis:model-fitness}
\todo{Analiza skuteczności (dopasowania) modelu}

\thispagestyle{normal}
