% !TeX spellcheck = pl_PL


\chapter{Optymalizacja procesu rekrutacji w~branży IT z~wykorzystaniem uczenia maszynowego}\label{ch:analysis}


\section{Metody badawcze}\label{sec:analysis-method}
Biorąc pod uwagę rosnące zapotrzebowanie na usługi informatyczne, deficyt wykwalifikowanych pracowników \cite{daxx-2021} oraz wysoki koszt rekrutacji nowych pracowników \cite{hairing-dev-2021} istotne jest poszukiwanie sposobów na zwiększenie retencji pracowników IT.
Badania będące przedmiotem niniejszej pracy mają na celu analizę cech skorelowanych z~zadowoleniem pracowników branży IT i~z ich chęcią do zmiany pracy.
% Badane cechy zostaną podzielone na 4 kategorie:
% \begin{enumerate}
%     \item profil firmy (rozmiar firmy),
%     \item organizacja firmy (rozmiar zespołów, metodyka zarządzania projektem),
%     \item profil pracownika (wiek, płeć, staż pracy, wykształcenie),
%     \item preferencje pracownika (technologia, benefity pracownicze).
%     \end{enumerate}
%
% Powyższe kategorie dzielą cechy firmy i~pracownika na 2 zasadnicze typy:
% \begin{enumerate}
%     \item cechy profilowe~- profil firmy i~profil pracownika,
%     \item cechy zmienne~- organizacja firmy i~preferencje pracownika.
%     \end{enumerate}
%
% Cechy profilowe mogą być uznawane za niezmienne w~kontekście prowadzonego badania, gdyż ich zmiana może całkowicie zmienić istotne cechy podmiotu
% (na przykład w~przypadku liczby pracowników zmiana może dotyczyć przekształcenia małej firmy w~międzynarodową korporację, a~w przypadku wieku pracownika zmiany z~młodej osoby w~osobę w~wieku przedemerytalnym~- w~obu przypadkach oczekiwania firmy wobec pracowników i~vice versa będą inne).
% Natomiast cechy zmienne mogą być zmieniane tak aby dostosować je do bieżących potrzeb.

Celem badania jest określenie czy i~jeśli tak to jakie cechy firmy i~pracownika mogą mieć istotny wpływ na zadowolenie pracownika
oraz czy można wyszczególnić cechy organizacyjne firmy, których modyfikacja mogłaby pozwolić na zwiększenie zadowolenia pracowników.
Określenie takich cech miałoby szczególne znaczenie już na etapie rekrutacji, gdyż mogłoby umożliwić selekcjonowanie kandydatów o~profilu "kompatybilnym" z~profilem firmy.


Badania zostaną przeprowadzone w~oparciu o~ankietę developerską StackOverflow \cite{so-survey-info}.
Ankieta ta jest przeprowadzana corocznie od 2011 roku, jednak popularność zyskała dopiero w~2016 roku, kiedy to liczba respondentów biorących udział w~badaniu przekroczyła 50 tysięcy.
W latach 2016-2022 liczba respondentów utrzymywała się na poziomie od 50 do 100 tysięcy.
Odpowiedzi w~ankiecie udzielali specjaliści z~całego świata reprezentując różne grupy wiekowe i~kulturowe.
Tak wysoka liczba odpowiedzi i~różnorodność respondentów może pozwolić na przeprowadzenie badań ilościowych i~umożliwić próbę generalizacji wniosków na całą branżę IT.


Do analizy danych zostaną wykorzystane metody uczenia maszynowego, a~w szczególności regresja z~wykorzystaniem regresora XGBoost opartego na algorytmie wzmocnienia gradientowego.
Zdecydowano się na modelowanie problemu jako regresji, a~nie klasyfikacji, gdyż zadowolenie pracowników oraz chęć zmiany pracy można przedstawić na skali liczbowej.
Punnoose i~Ajit \cite{punnoose-2016} przeprowadzili badania związane ze skutecznością przewidywania retencji pracowników w~firmie z~wykorzystaniem uczenia maszynowego.
Klasyfikator XGBoost osiągnął aż 86\% skuteczności klasyfikacji co stanowiło ponad 50\% lepszy wynik niż pozostałe testowane metody klasyfikacji tradycyjnie używane podczas badania problemu fluktuacji i~retencji pracowników.
Do celów badań będących przedmiotem niniejszej pracy implementacja algorytmu regresji i~klasyfikacji zostanie przygotowana w~języku Python.


\section{Wstępne przetworzenie danych}\label{sec:analysis:preprocessing}
Do przeprowadzenia badania wybrano ankiety deweloperskie StackOverflow z~lat 2017, 2018, 2019 (dostępne do pobrania pod adresem \url{https://insights.stackoverflow.com/survey}) ze względu na to, że są to 3 najnowsze edycje ankiety w~których zadano pytania o~następujące 2 aspekty:
\begin{enumerate}
    \item ocena zadowolenia z~obecnie wykonywanej pracy w~10 stopniowej skali,
    \item status poszukiwania pracy w~3 stopniowej skali.
\end{enumerate}

Powyższe aspekty zostaną wykorzystane w~badaniu jako zmienne zależne, czyli zmienne których wartość chcemy wyznaczać wykorzystując model predykcji.
Dla badań z~wykorzystaniem algorytmu regresji zmienne zależne pozostawiono w~oryginalnej skali,
natomiast dla algorytmu klasyfikacji odpowiedzi sprowadzono do klasyfikacji binarnej przy następujących założeniach:
\begin{enumerate}
    \item pracownik jest zadowolony z~pracy jeśli ocenił poziom zadowolenia na 5 lub więcej,
    \item pracownik ma status osoby poszukującej jeśli szuka pracy aktywnie lub jest otwarty na atrakcyjne oferty pracy.
\end{enumerate}

W celu przygotowania danych do badania konieczne było oczyszczenie i~znormalizowanie danych. Proces ten został przeprowadzony w~następujący sposób:
\begin{enumerate}
    \item Usunięto pytania, które nie mają istotnego wpływu na badany problem (na przykład pytanie dotyczące preferencji związanej z~wymawianiem słowa "GIF").
    \item Uwspólniono nazwy etykiet dotyczących tych samych pytań w~kolejnych edycjach ankiety.
    \item Usunięto odpowiedzi respondentów, którzy nie mają doświadczenia zawodowego lub nie udzielili odpowiedzi na pytania dotyczące zadowolenia z~pracy i~statusu poszukiwania pracy.
    \item Dla pytań z~odpowiedzami w~skali Likerta (typu od "bardzo się nie zgadzam" do "bardzo się zgadzam") zastosowano przekształcenie na skalę liczbową.
    \item Dla pytań z~odpowiedziami reprezentowanymi przez kategorie, które można przedstawić w~skali (na przykład edukacja) zastosowano przekształcenie na skalę liczbową.
    \item Dla pytań wielokrotnego wyboru oraz dla pytań z~odpowiedziami reprezentowanymi przez kategorie, których nie można przedstawić w~skali (na przykład technologie) potraktowano każdą kategorię jak osobne pytanie i~zastosowano kodowanie binarne (0~- niewystępuje i~1~- występuje).
    \item Kraj pochodzenia respondenta po zastąpieniu kodowaniem binarnym daje w~rezultacie macierz rzadką z~zaledwie 1 elementem niezerowym. Z~tego powodu kraj pochodzenia respondenta zastąpiono wartością wskaźnika rozwoju społecznego HDI (ang. Human Development Index) \cite{hdi} odpowiednią dla danego kraju i~roku badania.
    \item Wynagrodzenie respondentów przeliczono na wartość w~dolarach amerykańskich (USD) według kursu z~1 dnia roku w~którym przeprowadzono badanie.
    \item W~pytaniach na które respondent nie udzielił odpowiedzi, wartości uzupełniono średnią wartością z~pozostałych odpowiedzi, żeby wartości te nie wpływały negatywnie na działanie algorytmu regresji.
    \item Na koniec wszystkie wartości przeskalowano aby otrzymać wartości z~przedziału 0~- 1.
\end{enumerate}


\section{Selekcja cech z~wykorzystaniem uczenia maszynowego}\label{sec:analysis:feature-selection-xgb}

Do trenowania modelu uczenia maszynowego wykorzystano algorytm przedstawiony na listingu \ref{listing:xgb}.


\noindent\begin{minipage}{\textwidth}
             \begin{lstlisting}[caption={Algorytm uczenia modelu regresji}, label={listing:xgb}]
             \end{lstlisting}
             \hspace{.075\textwidth}\begin{minipage}{.85\textwidth}
                                        \begin{minted}{python}
import xgboost as xgb
import pandas as pd
import matplotlib.pyplot as plt
import math
from sklearn.model_selection import train_test_split
from sklearn.metrics import mean_squared_error, r2_score
from sklearn.feature_selection import SelectFromModel

def fit_model(data, dependent_variable, name):
    def fit_estimator(x_set, y_set):
        estimator = xgb.XGBRegressor(**estimator_params, eval_metric='rmsle')
        estimator.fit(x_set, y_set)
        return estimator

    x~= data.drop(regression_dependent_variables + class_dependent_variables, axis=1)
    y~= data[dependent_variable]

    x_train, x_test, y_train, y_test = train_test_split(x, y, test_size=0.3, stratify=y)

    selection = SelectFromModel(fit_estimator(x_train, y_train), threshold=.015, prefit=True)

    selection_estimator = fit_estimator(selection.transform(x_train), y_train)
    preds = selection_estimator.predict(selection.transform(x_test))
    print("RMSE: %.2f" % math.sqrt(abs(mean_squared_error(y_test, preds))))
    print("R2: %.2f" % r2_score(y_test, preds))

    xgb.plot_importance(selection_estimator, max_num_features=10)
    plt.tight_layout()
    plt.savefig(f'data/feat_importance_{name}.png')
                                        \end{minted}
             \end{minipage}

             \raggedright\source{\ownwork}
             \vspace{0.75cm}
\end{minipage}

Hiperparametry estymatora wyznaczono w~oparciu o~projekt Harathi \cite{harathi-2018} oraz z~wykorzystaniem narzędzia GridSearchCV.
Dodatkowo skorzystano z~metody histogramowej do budowania drzewa decyzyjnego w~celu skrócenia czasu uczenia modelu \cite{golarnyk-2021}.
Wyznaczone parametry przedstawiono na listingu \ref{listing:estimator_params}.


\noindent\begin{minipage}{\textwidth}
             \begin{lstlisting}[caption={Parametry estmatora}, label={listing:estimator_params}]
             \end{lstlisting}
             \hspace{.075\textwidth}\begin{minipage}{.85\textwidth}
                                        \begin{minted}{python}
estimator_params = {
    'tree_method': "hist",
    'single_precision_histogram': True,
    'n_jobs': -1,
    'n_estimators': 1500,
    'importance_type': 'weight',
    'use_label_encoder': False,
    'booster': 'gbtree',
    'scale_pos_weight': 1,
    'reg_alpha': 5,
    'colsample_bytree': .8,
    'learning_rate': .1,
    'min_child_weight': 7,
    'subsample': .5,
    'max_depth': 6,
    'gamma': 0.1,
    'reg_lambda': 1
}
                                        \end{minted}
             \end{minipage}

             \raggedright\source{\ownwork}
             \vspace{0.75cm}
\end{minipage}

Zbiór danych podzielono losowo w~stosunku 7:3 w~nastęujący sposób:
\begin{itemize}
    \item z~70\% danych utworzono zbiór uczący (ang. training set)
    \item z~30\% danych utworzono zbiór testowy (ang. test set)
\end{itemize}

Zbiór uczący został wykorzystany do trenowania modelu predykcji, natomiast zbiór testowy do sprawdzenia skuteczności predykcji przy pomocy odpowiednich miar.

Do sprawdzenia skuteczności predykcji modelu regresji wykorzystano następujące miary:
\begin{itemize}
    \item Pierwiastek logarytmu błędu średniokwadratowego RMSLE (ang. Root Mean Squared Logarytmic Error)~- pokazuje jak bardzo predykcja jest oddalona od oczekiwanej wartości. Im niższa wartość tym lepiej.
    \item Współczynnik determinacji R2~- pokazuje w~jakim stopniu predykcja jest dopasowana do oczekiwanej wartości. Im wyższa wartość tym lepiej.
\end{itemize}

Celem modelu jest maksymalizacja R2 i~minimalizacja RMSLE. Dobrze dopasowany model powinien mieć wartość R2 na poziomie powyżej 70\% \cite{r2-good-value}, jednak nawet przy niskiej wartości R2 model może być użyteczny jeżeli RMSLE ma wartość poniżej 10\% \cite{r2-vs-rmse}.
Natomiast przy akceptowalnym poziomie współczynnika R2, model może być użyteczny jeśli RMSLE nie przekracza 50\% \cite{rmse-good-value}.
Ma to szczególne znaczenie w~przypadku badań prowadzonych w~obszarze nauk o~społeczeństwie, gdyż nawet model nie oferujący bardzo wysokiej skuteczności predykcji może pozwolić na pogłębienie wglądu w~kwestię zachowań społecznych.

Do sprawdzenia skuteczności predykcji modelu klasyfikacji wykorzystano następujące miary:
\begin{itemize}
    \item Dokładność (ang. accuracy)~- pokazuje w~ilu przypadkach klasyfikacja oszacowała poprawną wartość,
    \item Precyzja (ang. precision)~- "proporcja prawidłowych pozytywnych klasyfikacji względem wszystkich pozytywnych klasyfikacji" \cite{recall-precision},
    \item Czułość (ang. recall)~- "proporcja poprawnych klasyfikacji przykładów pozytywnych względem wszystkich przykładów należących do klasy pozytywnej" \cite{recall-precision}.
\end{itemize}

Na podstawie opracowania Brownlee \cite{accuracy-level}, w~badaniach przyjęto wartości miar klasyfikacji za dobre, jeśli wynoszą powyżej 85\% i~za złe jeśli wynoszą poniżej 75\%.

Selekcję pytań najsilniej wpływających na predykcję wykonano z~wykorzystaniem metody SHAP (ang. SHapley Additive exPlanations).
Jest to metoda oparta o~kooperacyjną teorię gier i~jest wykorzystywana do zwiększenia możliwości interpretacyjnych modeli uczenia maszynowego.
Pozwala na selekcję cech silnie skorelowanych z~przewidzianym wynikiem, jednak nie pozwala na ocenę skuteczności samej predykcji \cite{shap}.


\section{Prezentacja wyników}\label{sec:analysis:important-features}

Algorytm przedstawiony na listingu \ref{listing:xgb} został uruchomiony dla znormalizowanych danych z~lat 2017-2019.
Lista atrybutów biorących udział w~badaniu i~odpowiadających im pytań została przedstawiona w~dodatku \ref{app:dod1}.
Wyniki zostały podzielone ze względu na modelowanie problemu jako regresję i~jako klasyfikację.

\subsection{Wyniki regresji}\label{sec:analysis:important-features:regression}

Na rysunkach \ref{fig:result1}, \ref{fig:result11} i~\ref{fig:result21} przedstawiono wyniki regresji dla zmiennej zależnej "JobSatisfaction", natomiast na rysunkach \ref{fig:result3}, \ref{fig:result13} i~\ref{fig:result23} przedstawiono wyniki regresji dla zmiennej zależnej "JobSeekingStatus".

\imagescale{0.6}{../code/feat_importance/feat_importance_2017_JobSatisfaction.png}{Średnia siła wpływu 10 pytań najsilniej wpływających na predykcję w~modelu regresji dla zmiennej zależnej JobSatisfaction dla ankiety StackOverflow z~roku 2017}{result1}

\imagescale{0.6}{../code/feat_importance/feat_importance_2018_JobSatisfaction.png}{Średnia siła wpływu 10 pytań najsilniej wpływających na predykcję w~modelu regresji dla zmiennej zależnej JobSatisfaction dla ankiety StackOverflow z~roku 2018}{result11}

\imagescale{0.6}{../code/feat_importance/feat_importance_2019_JobSatisfaction.png}{Średnia siła wpływu 10 pytań najsilniej wpływających na predykcję w~modelu regresji dla zmiennej zależnej JobSatisfaction dla ankiety StackOverflow z~roku 2019}{result21}

\imagescale{0.6}{../code/feat_importance/feat_importance_2017_JobSeekingStatus.png}{Średnia siła wpływu 10 pytań najsilniej wpływających na predykcję w~modelu regresji dla zmiennej zależnej JobSeekingStatus dla ankiety StackOverflow z~roku 2017}{result3}

\imagescale{0.6}{../code/feat_importance/feat_importance_2018_JobSeekingStatus.png}{Średnia siła wpływu 10 pytań najsilniej wpływających na predykcję w~modelu regresji dla zmiennej zależnej JobSeekingStatus dla ankiety StackOverflow z~roku 2018}{result13}

\imagescale{0.6}{../code/feat_importance/feat_importance_2019_JobSeekingStatus.png}{Średnia siła wpływu 10 pytań najsilniej wpływających na predykcję w~modelu regresji dla zmiennej zależnej JobSeekingStatus dla ankiety StackOverflow z~roku 2019}{result23}

W tabelach \ref{tabela:JobSatisfactionRegression} i~\ref{tabela:JobSeekingStatusRegression} przedstawiono podsumowanie skuteczności modeli regresji, odpowiednio dla zmiennych zależnych JobSatisfaction i~JobSeekingStatus.

\noindent\begin{minipage}{\textwidth}
             \begin{table}[H]
                 \raggedright\caption{Porównanie wyników regresji dla zmiennej zależnej JobSatisfaction\label{tabela:JobSatisfactionRegression}}
                 \begin{center}
                     \begin{tabular}{|P{.1\textwidth}|P{.26\textwidth}|P{.26\textwidth}|P{.26\textwidth}|}

                         \hline
                         Rok   & 2017             & 2018             & 2019             \\
                         \hline
                         RMSLE & \cellgreen{0.20} & \cellgreen{0.31} & \cellgreen{0.30} \\
                         \hline
                         R2    & \cellred{0.16}   & \cellred{0.04}   & \cellred{0.23}   \\
                         \hline
                         Pytania z~siłą wpływu większą od 0.2 &
                         \begin{itemize}
                             \item AssessJobProfDevel
                             \item HomeRemote
                             \item AssessJobRemote
                         \end{itemize} &
                         \begin{itemize}
                             \item LastNewJob
                             \item HopeFiveYears \_SameWork
                             \item Country\_HDI
                             \item AgreeDisagree1 \_KinshipToDevs
                             \item ConvertedSalary
                         \end{itemize} &
                         \begin{itemize}
                             \item MgrIncompetent
                             \item PurchaseWant
                             \item WorkChallenge \_ToxicEnvironment
                             \item WorkChallenge \_NoManagementSupport
                             \item LastNewJob
                         \end{itemize} \\
                         \hline
                     \end{tabular}
                 \end{center}
                 \raggedright\source{\ownwork}
                 \vspace{0.75cm}
             \end{table}
\end{minipage}

\noindent\begin{minipage}{\textwidth}
             \begin{table}[H]
                 \raggedright\caption{Porównanie wyników regresji dla zmiennej zależnej JobSeekingStatus\label{tabela:JobSeekingStatusRegression}}
                 \begin{center}
                     \begin{tabular}{|P{.1\textwidth}|P{.26\textwidth}|P{.26\textwidth}|P{.26\textwidth}|}

                         \hline
                         Rok   & 2017             & 2018             & 2019             \\
                         \hline
                         RMSLE & \cellgreen{0.18} & \cellgreen{0.43} & \cellgreen{0.38} \\
                         \hline
                         R2    & \cellgreen{0.63} & \cellred{0.18}   & \cellred{0.30}   \\
                         \hline
                         Pytania z~siłą wpływu większą od 0.2 &
                         \begin{itemize}
                             \item AssessJobProfDevel
                             \item AssessJobExp
                             \item AssessJobTech
                         \end{itemize} &
                         \begin{itemize}
                             \item Country\_HDI
                             \item LastNewJob
                             \item YearsCoding
                             \item HopeFiveYears \_SameWork
                             \item HopeFiveYears \_MoreSpecialized
                         \end{itemize} &
                         \begin{itemize}
                             \item MgrIncompetent
                             \item Country\_HDI
                             \item LastNewJob
                             \item WorkChallenge \_NoManagementSupport
                             \item ConvertedSalary
                         \end{itemize} \\
                         \hline
                     \end{tabular}
                 \end{center}
                 \raggedright\source{\ownwork}
                 \vspace{0.75cm}
             \end{table}
\end{minipage}

Mimo stosunkowo dobrych wartości RMSLE, w~większości badanych przypadków osiągnięto niską wartość R2, co oznacza, że wyselekcjonowane atrybuty są zasadniczo słabo skorelowane ze zmienną zależną.
Satysfakcjonujący rezultat osiągnięto dla danych z~2017 roku i~zmiennej zależnej JobSeekingStatus.
Na tej podstawie można wnioskować, że na chęć poszukiwania pracy mogą mieć wpływ następujące cechy:

\begin{itemize}
    \item Czy respondent uważa, że możliwość rozwoju osobistego w~firmie jest ważna,
    \item Czy respondent uważa, że wymagane na stanowisku doświadczenie jest ważne,
    \item Czy respondent uważa, że technologie wykorzystywane w~pracy są ważne.
\end{itemize}

\subsection{Wyniki klasyfikacji}\label{sec:analysis:important-features:classification}
%\bargraph{{AssessJobIndustry}, {AssessJobExp}, {AssessJobLeaders}, {AssessJobProfDevel}, {AssessJobDiversity}, {AssessJobFinances}, {StackOverflowCompanyPage}, {StackOverflowMetaChat}, {ImportantBenefits\_Vacation/day}, {ImportantBenefits\_Remote optio}}{(0.27,{AssessJobIndustry}) (2.60,{AssessJobExp}) (0.15,{AssessJobLeaders}) (1.34,{AssessJobProfDevel}) (3.01,{AssessJobDiversity}) (0.72,{AssessJobFinances}) (0.67,{StackOverflowCompanyPage}) (0.61,{StackOverflowMetaChat}) (0.47,{ImportantBenefits\_Vacation/day}) (0.33,{ImportantBenefits\_Remote optio})}{feat\_importance\_2017\_JobSeekingStatus}{feat\_importance\_2017\_JobSeekingStatus}{10cm}

Na rysunkach \ref{fig:result2}, \ref{fig:result12} i~\ref{fig:result22} przedstawiono wyniki klasyfikacji dla zmiennej zależnej "JobSatisfaction", natomiast na rysunkach \ref{fig:result4}, \ref{fig:result14} i~\ref{fig:result24} przedstawiono wyniki klasyfikacji dla zmiennej zależnej "JobSeekingStatus".

\imagescale{0.6}{../code/feat_importance/feat_importance_2017_JobSatisfaction_class.png}{Średnia siła wpływu 10 pytań najsilniej wpływających na predykcję w~modelu klasyfikacji dla zmiennej zależnej JobSatisfaction dla ankiety StackOverflow z~roku 2017}{result2}

\imagescale{0.6}{../code/feat_importance/feat_importance_2018_JobSatisfaction_class.png}{Średnia siła wpływu 10 pytań najsilniej wpływających na predykcję w~modelu klasyfikacji dla zmiennej zależnej JobSatisfaction dla ankiety StackOverflow z~roku 2018}{result12}

\imagescale{0.6}{../code/feat_importance/feat_importance_2019_JobSatisfaction_class.png}{Średnia siła wpływu 10 pytań najsilniej wpływających na predykcję w~modelu klasyfikacji dla zmiennej zależnej JobSatisfaction dla ankiety StackOverflow z~roku 2019}{result22}

\imagescale{0.6}{../code/feat_importance/feat_importance_2017_JobSeekingStatus_class.png}{Średnia siła wpływu 10 pytań najsilniej wpływających na predykcję w~modelu klasyfikacji dla zmiennej zależnej JobSeekingStatus dla ankiety StackOverflow z~roku 2017}{result4}

\imagescale{0.6}{../code/feat_importance/feat_importance_2018_JobSeekingStatus_class.png}{Średnia siła wpływu 10 pytań najsilniej wpływających na predykcję w~modelu klasyfikacji dla zmiennej zależnej JobSeekingStatus dla ankiety StackOverflow z~roku 2018}{result14}

\imagescale{0.6}{../code/feat_importance/feat_importance_2019_JobSeekingStatus_class.png}{Średnia siła wpływu 10 pytań najsilniej wpływających na predykcję w~modelu klasyfikacji dla zmiennej zależnej JobSeekingStatus dla ankiety StackOverflow z~roku 2019}{result24}


W tabelach \ref{tabela:JobSatisfactionClassification} i~\ref{tabela:JobSeekingStatusClassification} przedstawiono podsumowanie skuteczności modeli klasyfikacji, odpowiednio dla zmiennych zależnych JobSatisfaction i~JobSeekingStatus.

\noindent\begin{minipage}{\textwidth}
             \begin{table}[H]
                 \raggedright\caption{Porównanie wyników klasyfikacji dla zmiennej zależnej JobSatisfaction\label{tabela:JobSatisfactionClassification}}
                 \begin{center}
                     \begin{tabular}{|P{.1\textwidth}|P{.26\textwidth}|P{.26\textwidth}|P{.26\textwidth}|}

                         \hline
                         Rok        & 2017             & 2018             & 2019             \\
                         \hline
                         Dokładność & \cellgreen{0.87} & \cellgray{0.76}  & \cellgray{0.81}  \\
                         \hline
                         Precyzja   & \cellgreen{0.87} & \cellgray{0.77}  & \cellgray{0.83}  \\
                         \hline
                         Czułość    & \cellgreen{1.00} & \cellgreen{1.00} & \cellgreen{0.96} \\
                         \hline
                         Pytania z~siłą wpływu większą od 0.2 &
                         \begin{itemize}
                             \item HomeRemote
                             \item AssessJobProfDevel
                             \item Country\_HDI
                             \item KinshipDevelopers
                             \item AssessJobRemote
                         \end{itemize} &
                         \begin{itemize}
                             \item HopeFiveYears \_SameWork
                             \item LastNewJob
                             \item Country\_HDI
                             \item ConvertedSalary
                             \item YearsCoding
                         \end{itemize} &
                         \begin{itemize}
                             \item MgrIncompetent
                             \item WorkChallenge \_ToxicEnvironment
                             \item WorkChallenge \_NoManagementSupport
                             \item LastNewJob
                             \item PurchaseWhat
                         \end{itemize} \\
                         \hline
                     \end{tabular}
                 \end{center}
                 \raggedright\source{\ownwork}
                 \vspace{0.75cm}
             \end{table}
\end{minipage}

\noindent\begin{minipage}{\textwidth}
             \begin{table}[H]
                 \raggedright\caption{Porównanie wyników klasyfikacji dla zmiennej zależnej JobSeekingStatus\label{tabela:JobSeekingStatusClassification}}
                 \begin{center}
                     \begin{tabular}{|P{.1\textwidth}|P{.26\textwidth}|P{.26\textwidth}|P{.26\textwidth}|}

                         \hline
                         Rok        & 2017             & 2018           & 2019            \\
                         \hline
                         Dokładność & \cellgreen{0.95} & \cellred{0.74} & \cellgray{0.79} \\
                         \hline
                         Precyzja   & \cellgreen{0.95} & \cellred{0.68} & \cellred{0.69}  \\
                         \hline
                         Czułość    & \cellgreen{0.99} & \cellred{0.47} & \cellred{0.53}  \\
                         \hline
                         Pytania z~siłą wpływu większą od 0.2 &
                         \begin{itemize}
                             \item AssessJobExp
                             \item AssessJobCommute
                             \item AssessJobProfDevel
                             \item AssessJobRole
                             \item AssessJobLeaders
                         \end{itemize} &
                         \begin{itemize}
                             \item LastNewJob
                             \item Country\_HDI
                             \item HopeFiveYears \_SameWork
                             \item YearsCoding
                             \item ConvertedSalary
                         \end{itemize} &
                         \begin{itemize}
                             \item MgrIncompetent
                             \item LastNewJob
                             \item Country\_HDI
                             \item WorkChallenge \_NoManagementSupport
                             \item WorkChallenge \_ToxicEnvironment
                         \end{itemize} \\
                         \hline
                     \end{tabular}
                 \end{center}
                 \raggedright\source{\ownwork}
                 \vspace{0.75cm}
             \end{table}
\end{minipage}

Satysfakcjonujący rezultat osiągnięto dla danych z~2017 roku i~obu badanych zmiennych zależnych.
Na tej podstawie można wnioskować, że na zadowolenie z~pracy mogą mieć wpływ następujące cechy:

\begin{itemize}
    \item Jak często respondent obecnie pracuje zdalnie,
    \item Czy respondent uważa, że możliwość rozwoju osobistego w~firmie jest ważna,
    \item Indeks rozwoju społecznego kraju, w~którym mieszka respondent,
    \item W~jakim stopniu respondent czuje solidarność ze swoimi współpracownikami,
    \item Czy respondent uważa, że możliwość pracy zdalnej jest ważna.
\end{itemize}

Natomiast na chęć poszukiwania pracy mogą mieć wpływ następujące cechy:

\begin{itemize}
    \item Czy respondent uważa, że doświadczenie wymagane na stanowisku pracy jest ważne,
    \item Czy respondent uważa, że czas dojazdu do pracy jest ważny,
    \item Czy respondent uważa, że możliwość rozwoju osobistego w~firmie jest ważna,
    \item Czy respondent uważa, że nazwa roli lub stanowiska pracy jest ważna,
    \item Czy respondent uważa, że reputacja seniorów pracujących w~firmie jest ważna.
\end{itemize}


\section{Wnioski i~analiza możliwości praktycznego zastosowania zbudowanego modelu predykcji}\label{sec:analysis:model-fitness}

Przeprowadzone badania wykazały, że dla zbioru danych opartego o~wyniki ankiety deweloperskiej StackOverflow \cite{so-survey-info}
można zbudować skuteczny model predykcji zarówno modelując problem jako regresję, jak i~jako klasyfikację.

Opracowane modele pozwoliły na wytypowanie cech silnie wpływających na możliwość predykcji zadowolenia z~pracy i~chęci szukania nowej pracy.

Szczególnie wysoką jakość modelu predykcji osiągnięto dla zbioru danych z~2017 roku.

Osiągnięte rezultaty mogą być uwzględnione podczas rekrutacji nowych pracowników
oraz mogą być sugestią dla pracowników działów zasobów ludzkich, w~jakich aspektach można poprawić funkcjonowanie firmy, aby zwiększyć retencję pracowników.

W przyszłości badania można rozszerzyć uwzględniając podział pracowników ze względu na takie cechy jak:

\begin{itemize}
    \item wiek,
    \item płeć,
    \item kraj zamieszkania,
    \item obszar specjalizacji;
\end{itemize}

oraz uwzględniają podział firm ze względu na takie cechy jak:

\begin{itemize}
    \item rozmiar firmy,
    \item gałąź przemysłu, w~obrębie której działa firma.
\end{itemize}

Przypuszczać można, że stosując taki granularny podział możliwe będzie wyszczególnienie cech szczególnie pożądanych wśród kandydatów zatrudnianych do określonej firmy
lub cech przedsiębiorstwa, na które konkretny programista poszukujący pracy powinien zwracać uwagę.

Zarówno dla bieżących modeli predykcji, jak i~potencjalnie możliwych do opracowania modeli szczegółowych,
możliwe byłoby opracowanie kwestionariusza, w~którym po uzupełnieniu danych dotyczących kandydata i~przedsiębiorstwa
algorytm korzystający z~opracowanych modeli predykcji byłby w~stanie oszacować, w~jakim stopniu pracownik będzie zadowolony z~pracy.

\thispagestyle{normal}
