% !TeX spellcheck = pl_PL


\chapter{Problem fluktuacji pracowników w~literaturze}\label{ch:knowladge-state}


\section{Zjawisko fluktuacji}\label{sec:zjawisko-fluktuacji}
%\todo{zjawisko fluktuacji + fluktuacja w~zależności od branży}
Fluktuacja pracowników (ang. employee turnover) w~literaturze definiowana jest na wiele sposób, jednak najczęstsze to "dobrowolne odejście z~pracy pracowników dojrzałych" \cite{wozniak-2016}
lub szerzej jako "tempo w~jakim pracownicy opuszczają firmę" \cite{cron-2006}.
Czasami fluktuacja jest utożsamiana z~rotacją pracowników (ang. job rotation),
jednak częściej rotację definiuje się nie jako samo opuszczanie firmy przez pracowników, a~raczej jako proces wymiany pracowników,
który może służyć jako metoda rozwoju dla pracowników (np. poprzez przeniesienie pracownika na inne stanowisko wewnątrz organizacji lub awans) \cite{pocztowski-2009}.
Wysoka fluktuacja może nie być pożądana, choćby ze względu na wysokie koszty z~nią związane \cite{philips-edwards-2009},
więc odpowiedzią pracodawcy może być próba jej ograniczenia.
Proces taki określany jest mianem retencji pracowników i~jego głównym celem jest utrzymanie zatrudnienia na poziomie pozwalającym na "sprawną realizację procesów biznesowych" \cite{pocztowski-2007}.

Fluktuacja może być skategoryzowana na kilka różnych sposobów, co zostało przedstawione w~tabeli \ref{tabela:fluktuacja-rodzaje}.

\noindent\begin{minipage}{\textwidth}
             \begin{table}[H]
                 \raggedright\caption{Rodzaje fluktuacji\label{tabela:fluktuacja-rodzaje}}
                 \begin{center}
                     \begin{tabular}{|P{.15\textwidth}|P{.75\textwidth}|}

                         \hline
                         \cellgray{Cecha fluktuacji} &
                         \cellgray{Opis} \\
                         \hline

                         pożądana &
                         \begin{itemize}
                             \item pożądana~- opuszczenie pracownika pozytywnie wpływa na działanie firmy (np. pracownik o~niskich kwalifikacjach),
                             \item niepożądana~- pracownik jest trudny do zastąpienia i~jego pracy przynosi firmie korzyści;
                         \end{itemize} \\
                         \hline

                         dobrowolna &
                         \begin{itemize}
                             \item dobrowolna~- pracownik sam odchodzi z~firmy,
                             \item niedobrowolna~- pracodawca zwalnia pracownika lub pracownik w~inny sposób zostaje zmuszony do odejścia;
                         \end{itemize} \\
                         \hline

                         zapobiegalna &
                         \begin{itemize}
                             \item zapobiegalna~- firma jest jej w~stanie zapobiec,
                             \item niezapobiegalna~- zależy od czynników zewnętrznych, na które firma nie ma wpływu (np. wprowadzenie nowych regulacji prawnych);
                         \end{itemize} \\
                         \hline

                         nadmierna &
                         \begin{itemize}
                             \item związana ściśle z~wewnętrznym działaniem firmy~- cechy takie jak złe warunki pracy i~nieadekwatne wynagrodzenie mogą wpłynąć na zwiększenie fluktuacji pracowników w~danym przedsiębiorstwie;
                         \end{itemize} \\
                         \hline
                     \end{tabular}
                 \end{center}
                 \raggedright\source{\workbasedon{\cite{spychala-2019}}}
                 \vspace{0.75cm}
             \end{table}
\end{minipage}

Do mierzenia poziomu fluktuacji wykorzystuje się wskaźnik fluktuacji definiowany jako stosunek osób opuszczającej w~firmę w~danym roku do średniej liczby pracowników zatrudnionych w~danym roku.
Do wyliczenia fluktuacji można wykorzystać liczbę wszystkich pracowników, którzy opuścili organizację~- niezależnie od powodu opuszczenia tejże organizacji~- ale można też obliczyć ten wskaźnik uwzględniając jedynie fluktuację dobrowolną.
Badanie fluktuacji dobrowolnej jest o~tyle istotne, że może pozwolić na wykrycie problemów w~firmie i~opracowanie sposobu na ich przeciwdziałanie \cite{spychala-2019}.
Co więcej, badania pokazują, że większość odejść w~organizacjach stanowią właśnie odejścia dobrowolne \cite{dalton-1982}.
Z~tego względu, dalsze rozważania będą dotyczyły fluktuacji dobrowolnej, o~ile wprost nie będzie napisane inaczej.


\section{Wpływ fluktuacji na firmę}\label{sec:wplyw-fluktuacji-na-firme}
Fluktuacja pracowników niekoniecznie musi oznaczać problem dla przedsiębiorstwa.
Z~analizy badań Human Capital Index przeprowadzonych przez firmę Watson Wyatt w~2005 r. wynika,
że zarówno bardzo niska i~bardzo wysoka fluktuacja nie są korzystne dla przedsiębiorstw.
Organizacje z~umiarkowanym poziomem wskaźnika fluktuacji wynoszącym ok. 15\%
miały zwrot z~inwestycji akcjonariuszy (ang. total shareholders return~- TSR) na poziomie 43\%,
co stanowiło rezultat średnio o~9 punktów procentowych lepszy niż firmy o~niższym lub o~wyższym wskaźniku fluktuacji \cite{krol-ludwiczynski-2006}.

Nie zawsze jednak można jednoznacznie stwierdzić, że wysoki poziom fluktuacji będzie negatywny w~skutkach dla danego biznesu.
Decyzję o~tym należy poprzedzić analizą sytuacji w~przedsiębiorstwach konkurencyjnych oraz specyfiki rynku pracy \cite{taylor-2006}.
Heurystyka pozwalająca określić czy wysoki poziom fluktuacji jest akceptowalny w~przypadku danej branży została przedstawiona w~tabeli \ref{tabela:wysoka-fluktuacja-heurystyka}.

\noindent\begin{minipage}{\textwidth}
             \begin{table}[H]
                 \raggedright\caption{Heurystyka do oceny wysokiego poziomu fluktuacji\label{tabela:wysoka-fluktuacja-heurystyka}}
                 \begin{center}
                     \begin{tabular}{|P{.45\textwidth}|P{.45\textwidth}|}

                         \hline
                         \cellgray{Wysoki poziom fluktuacji nieakceptowalny} &
                         \cellgray{Wysoki poziom fluktuacji akceptowalny} \\
                         \hline

                         \begin{itemize}
                             \item niedobór kandydatów o~odpowiednich kompetencjach na rynku pracy,
                             \item fluktuacja jest wyższa niż u~bezpośredniej konkurencji,
                             \item wysokie koszty rekrutacji;
                         \end{itemize} &
                         \begin{itemize}
                             \item wielu kandydatów o~odpowiednich kompetencjach na rynku pracy,
                             \item niskie koszty rekrutacji,
                             \item niski koszt wdrożenia nowego pracownika,
                             \item niskie ryzyko utraty wiedzy w~wyniku odejścia pracownika,
                             \item przewidywanie redukcji etatów w~niedalekiej przyszłości;
                         \end{itemize} \\
                         \hline
                     \end{tabular}
                 \end{center}
                 \raggedright\source{\workbasedon{\cite{taylor-2006}}}
                 \vspace{0.75cm}
             \end{table}
\end{minipage}

\clearpage


\section{Przyczyny fluktuacji pracowników}\label{sec:czynniki-wplywajace-na-fluktuacje}

Najstarszy model opisujący dobrowolną fluktuację pracowników został opracowany przez Marcha i~Simona w~1958 roku.
Model ten wyróżnia 2 czynniki wpływające istotnie na poziom fluktuacji \cite{wozniak-2012}:
\begin{itemize}
    \item to jak pracownik ocenia swoją chęć zmiany pracy~- związane jest to głównie z~niską satysfakcją z~wykonywanej pracy oraz niskim zaangażowaniem organizacyjnym,
    \item to jak pracownik ocenia łatwość zmiany pracy~- związane jest to z~dostępnością na rynku pracy posad interesujących pracownika.
\end{itemize}

W~ciągu ostatnich 50 lat powstało wiele modeli prognozujących fluktuację i~większość z~nich korzysta mniej lub bardziej bezpośrednio z~cech wyróżnionych przez Marcha i~Simona.
Badania empiryczne prowadzone w~tym czasie potwierdziły, że te cechy mają znaczący wpływ na fluktuację~- jednak nie oddają w~pełni istoty fluktuacji.
Przy pracy nad tymi modelami wyszczególniono wiele drugorzędnych cech wpływających na fluktuację \cite{steel-2009}.
Cechy te zostały przedstawione w~tabelach~\ref{tabela:fluktuacja-cechy}~i~\ref{tabela:fluktuacja-cechy-2}.

\noindent\begin{minipage}{\textwidth}
             \begin{table}[H]
                 \raggedright\caption{Najczęściej pojawiające się cechy w~modelach fluktuacji\label{tabela:fluktuacja-cechy}}
                 \begin{center}
                     \begin{tabular}{|P{.2\textwidth}|P{.7\textwidth}|}

                         \hline
                         \cellgray{Kategoria} &
                         \cellgray{Cechy} \\
                         \hline

                         cechy osobiste &
                         \begin{itemize}
                             \item osobowość,
                             \item wyznawane wartości,
                             \item wiek,
                             \item staż pracy,
                             \item wiedza,
                             \item doświadczenie,
                             \item profesjonalizm,
                             \item odpowiedzialność wobec rodziny;
                         \end{itemize} \\

                         \hline

                     \end{tabular}
                 \end{center}
                 \raggedright\source{\workbasedon{\cite{steel-2009}}}
                 \vspace{0.75cm}
             \end{table}
\end{minipage}

\noindent\begin{minipage}{\textwidth}
             \begin{table}[H]
                 \raggedright\caption{Najczęściej pojawiające się cechy w~modelach fluktuacji (ciąg dalszy)\label{tabela:fluktuacja-cechy-2}}
                 \begin{center}
                     \begin{tabular}{|P{.2\textwidth}|P{.7\textwidth}|}

                         \hline
                         \cellgray{Kategoria} &
                         \cellgray{Cechy} \\
                         \hline

                         cechy stanowiska pracy &
                         \begin{itemize}
                             \item postrzeganie pracy,
                             \item skomplikowanie pracy,
                             \item oczekiwania wobec wykonywanej pracy,
                             \item wynagrodzenie i~benefity,
                             \item koszt zmiany pracy,
                             \item stres,
                             \item dopasowanie wykonywanej pracy do oczekiwań pracownika,
                             \item rozmiar firmy;
                         \end{itemize} \\
                         \hline

                         mechanizmy zmiany stanowiska &
                         \begin{itemize}
                             \item chęć zmiany,
                             \item oczekiwania względem przyszłej pracy,
                             \item wysiłek potrzebny do zmiany bierzącej sytuacji,
                             \item możliwość przejścia do firmy powiązanej lub innego oddziału firmy,
                             \item możliwość awansu lub degradacji,
                             \item alternatywne sposoby opuszczenia pracy;
                         \end{itemize} \\
                         \hline

                         konsekwencje opuszczenia lub pozostania w~firmie &
                         \begin{itemize}
                             \item konsekwencje pozapracowe,
                             \item wydajność pracy;
                         \end{itemize} \\
                         \hline

                         mechanizmy wpływające na proces decyzyjny &
                         \begin{itemize}
                             \item zdarzenia nieprzewidziane,
                             \item myśli o~odejściu z~firmy;
                         \end{itemize} \\
                         \hline
                     \end{tabular}
                 \end{center}
                 \raggedright\source{\workbasedon{\cite{steel-2009}}}
                 \vspace{0.75cm}
             \end{table}
\end{minipage}
%\section{Zadowolenie pracowników}\label{sec:employee-happiness}
%\todo{zadowolenie pracownikó}
%\todo{wątek pandemii}
%\todo{fluktuacja pracowników a~praca zdalna}

%Modele przbliżające zjawisko fluktuacji zgodnie stwierdzają, że zadowolenie i~zaangażowanie organizacyjne pracowników znacząco wpływają na fluktuację,
%w związku z~tym zjawiska zostały te przedstawione w~sekcjach \ref{sec:czynniki-wplywajace-na-fluktuacje:zadowolenie} i~\ref{sec:czynniki-wplywajace-na-fluktuacje:zaangazowanie-organizacyjne}.

\subsection{Zadowolenie i~satysfakcja pracowników}\label{sec:czynniki-wplywajace-na-fluktuacje:zadowolenie}
%Zadowolenie \cite{sowinska-2014}
%Satysfakcja \cite{robak-2013}
Satysfakcję z~pracy najprościej można zdefiniować jako "pozytywne i~negatywne uczucia oraz postawy wobec wykonywanej pracy"\cite{shultz-2002}.
Często w~literaturze zamiennie do określenia "satysfakcja z~pracy" występuje "zadowolenie z~pracy", jednak czasem wskazywane jest rozróżnienie ze względu na czas trwania \cite{sowinska-2014}:
\begin{itemize}
    \item zadowolenie jest uczuciem chwilowym,
    \item satysfakcja jest odczuwana po długim czasie odczuwania zadowolenia.
\end{itemize}

Na satysfakcję z~pracy wpływają głównie 3 grupy czynników \cite{shultz-2002}:
\begin{itemize}
    \item czynniki związane z~pracą~- np. zakres obowiązków, lokalizacja biura, relacje z~współpracownikami,
    \item czynniki indywidualne~- np. staż pracy, wiek, zdrowie, zależności rodzinne,
    \item motywacja i~aspiracje.
\end{itemize}

W~tabelach \ref{tabela:zadowolenie-pracownikow-organizacja} i~\ref{tabela:zadowolenie-pracownikow-osoba} omówiono poszczególne czynniki wpływające na zadowolenie z~pracy.

\noindent\begin{minipage}{\textwidth}
             \begin{table}[H]
                 \raggedright\caption{Czynniki organizacyjne wpływające na zadowolenie pracowników\label{tabela:zadowolenie-pracownikow-organizacja}}
                 \begin{center}
                     \begin{tabular}{|P{.2\textwidth}|P{.7\textwidth}|}

                         \hline
                         \cellgray{Czynnik} &
                         \cellgray{Opis} \\
                         \hline

                         doświadczenia zawodowe &
                         Satysfakcja spada u~pracowników z~kilkuletnim doświadczeniem, jeśli w~swojej pracy nie są odpowiednio nagradzani i~motywowani. \\
                         \hline

                         wykorzystanie umiejętności &
                         Pracownicy, którzy w~pracy mogą korzystać z~wcześniej zdobytych umiejętności są zwykle bardziej zadowoleni. \\
                         \hline

                         odpowiedniość pracy &
                         Pracownicy odczuwają większą satysfakcję jeśli wymagania stanowiska pracy pokrywają się z~ich zdolnościami i~aspiracjami. \\
                         \hline

                         status pracy &
                         Status społeczny stanowiska istotnie wpływa na poziom zadowolenia pracownika. \\
                         \hline
                     \end{tabular}
                 \end{center}
                 \raggedright\source{\workbasedon{\cite{shultz-2002}}}
                 \vspace{0.75cm}
             \end{table}
\end{minipage}

\noindent\begin{minipage}{\textwidth}
             \begin{table}[H]
                 \raggedright\caption{Czynniki osobowe wpływające na zadowolenie pracowników\label{tabela:zadowolenie-pracownikow-osoba}}
                 \begin{center}
                     \begin{tabular}{|P{.2\textwidth}|P{.7\textwidth}|}

                         \hline
                         \cellgray{Czynnik} &
                         \cellgray{Opis} \\
                         \hline

                         wiek &
                         Statystycznie wraz z~wiekiem rośnie zadowolenie z~pracy.\\
                         \hline

                         płeć &
                         Nie wykazano bezpośredniej korelacji płci z~zadowoleniem z~pracy,
                         natomiast pośrednio na zadowolenie przedstawicieli danej płci może wpływać różnica w~poziomach wynagrodzeń kobiet i~mężczyzn na takich samych stanowiskach. \\
                         \hline

                         rasa &
                         Statystycznie przedstawiciele mniejszości etnicznych częściej są przyjmowani na stanowiska gorsze lub gorzej płatne niż przedstawiciele lokalnej większości etnicznej. \\
                         \hline

                         zdolności poznawcze &
                         Osoby o~wysokich zdolnościach poznawczych częściej odczuwają brak satysfakcji jeśli praca nie angażuje ich intelektualnie. \\
                         \hline

                         cechy osobowości &
                         Badania wskazują na korelację pomiędzy stabilnością emocjonalną, a~zadowoleniem z~pracy. \\
                         \hline

                     \end{tabular}
                 \end{center}
                 \raggedright\source{\workbasedon{\cite{shultz-2002}}}
                 \vspace{0.75cm}
             \end{table}
\end{minipage}

W pierwszych próbach badań nad zagadnieniem zadowolenia z~pracy, zadowolenie definiowano jako przestrzeń jednowymiarową rozpinającą się od niezadowolenia do zadowolenia \cite{sowinska-2014}.
Oznaczało to, że występowanie określonego czynnika mogło skutkować zadowoleniem, a~jego brak~- niezadowoleniem.
Nowe spojrzenie na tę kwestię wprowadził F. Herzberg definiując teorię dwuczynnikową,
która zakłada, że czynniki wpływające na zadowolenie mogą być rozpatrywane niezależnie od czynników wpływających na niezadowolenie.
Teoria ta jednak została zdefiniowana w~sposób uznany za nieprecyzyjny, przez co spotkała się z~szeroką falą krytki \cite{sowinska-2014}.

Rozwinięciem teorii Herzberga jest teoria trychotomii czynników motywacji, którą zaproponował L. Kozioł \cite{koziol-2011}.
Według tej teorii można wyszczególnić 3 kategorie czynników wpływających na zadowolenie: motywatory, demotywatory i~czynniki higieny.
Czynniki te zostały omówione w~tabeli \ref{tabela:trychotomia}.


\noindent\begin{minipage}{\textwidth}
             \begin{table}[H]
                 \raggedright\caption{Wpływ czynników motywacji na zadowolenie pracowników\label{tabela:trychotomia}}
                 \begin{center}
                     \begin{tabular}{|P{.2\textwidth}|P{.2\textwidth}|P{.5\textwidth}|}

                         \hline
                         \cellgray{Kategoria} &
                         \cellgray{Wpływ} &
                         \cellgray{Przykłady} \\
                         \hline

                         motywatory &
                         ich występowanie wpływa na zadowolenie &
                         \begin{itemize}
                             \item odpowiedzialność,
                             \item uznanie,
                             \item możliwość rozwoju osobistego,
                             \item osiągnięcia,
                             \item awans;
                             \end{itemize}\\
                         \hline

                         demotywatory &
                         ich występowanie wpływa na niezadowolenie &
                         \begin{itemize}
                             \item niejasne oczekiwania,
                             \item zmuszanie pracowników do działań sprzecznych z~etyką zawodową,
                             \item praca ponad siły,
                             \item nierówny podział obowiązków,
                             \item brak szkoleń,
                             \item niedocenianie pracowników z~wieloletnim stażem,
                             \item nadmierna presja,
                             \item strach przed utratą pracy;
                             \end{itemize}\\
                         \hline

                         czynniki higieny &
                         ich niewystępowanie wpływa na niezadowolenie &
                         \begin{itemize}
                             \item wynagrodzenia,
                             \item zajmowane stanowisko,
                             \item warunki pracy,
                             \item nadzór techniczny,
                             \item bezpieczeństwo pracy,
                             \item polityka firmy,
                             \item świadczenia socjalne;
                             \end{itemize}\\
                         \hline
                     \end{tabular}
                 \end{center}
                 \raggedright\source{\workbasedon{\cite{koziol-2011}}}
                 \vspace{0.75cm}
             \end{table}
\end{minipage}

\subsection{Zaangażowanie organizacyjne pracowników}\label{sec:czynniki-wplywajace-na-fluktuacje:zaangazowanie-organizacyjne}
%Zaangażowanie \cite{buzowska-2017}
Zaangażowanie organizacyjne jest w~literaturze definiowane na wiele różnych sposobów.
Istniejące definicje można podzielić na 3 koncepcje \cite{juchnowicz-2010}:
\begin{enumerate}
    \item zaangażowanie jako postawa pracownika,
    \item zaangażowanie przejawiane poprzez zachowanie pracownika,
    \item zaangażowanie jako wymiana świadczeń na poziomie relacji firma -- pracownik.
\end{enumerate}

M.~Juchnowicz \cite{juchnowicz-2010} pokazuje, że pierwsza koncepcja traktująca zaangażowanie organizacyjne jako postawę pracownika najszerzej opisuje zjawisko zaangażowania,
gdyż zawiera 3 czynniki kształtujące zaangażowanie: poznawczy, emocjonalny i~behawioralny.
Czynniki te zostały omówione w~tabeli \ref{tabela:czynniki-zaangazowania}.
%\begin{enumerate}
%    \item poznawczy (myślenie)~- wiedza o~organizacji jest podstawą zaangażowania,
%    \item emocjonalny (odczuwanie)~- konieczny jest stosunek emocjonalny w~stosunku do sposobu działania organizacji oraz jej wartości i~celów,
%    \item behawioralny (działanie)~- do zaangażowania potrzebna jest chęć podjęcia działań względem organizacji.
%\end{enumerate}

\noindent\begin{minipage}{\textwidth}
             \begin{table}[H]
                 \raggedright\caption{Czynniki kształtujące zaangażowanie\label{tabela:czynniki-zaangazowania}}
                 \begin{center}
                     \begin{tabular}{|P{.2\textwidth}|P{.2\textwidth}|P{.5\textwidth}|}

                         \hline
                         \cellgray{Czynnik} &
                         \cellgray{Akcja} &
                         \multicolumn{1}{>{\centering\arraybackslash}m{.5\textwidth}|}{\cellgray{Opis}} \\
                         \hline

                         poznawczy &
                         myślenie &
                         Wiedza o~organizacji jest podstawą zaangażowania. \\
                         \hline

                         emocjonalny &
                         odczuwanie &
                         Konieczny jest stosunek emocjonalny w~stosunku do sposobu działania organizacji oraz jej wartości i~celów. \\
                         \hline

                         behawioralny &
                         działanie &
                         Do zaangażowania potrzebna jest chęć podjęcia działań względem organizacji. \\
                         \hline
                     \end{tabular}
                 \end{center}
                 \raggedright\source{\workbasedon{\cite{juchnowicz-2010}}}
                 \vspace{0.75cm}
             \end{table}
\end{minipage}

Na tej podstawie zaangażowanie można zdefiniować jako "intelektualne i~emocjonalne oddanie organizacji" \cite{juchnowicz-2010}.
Dla tak zdefiniowanego zaangażowania można wyszczególnić 4 podstawowe cechy postawy pracownika zaangażowanego \cite{juchnowicz-2010}:
\begin{itemize}
    \item stabilizacja~- pracownikowi zależy żeby należeć do firmy,
    \item identyfikacja~- pracownik wierzy w~misje, wartości i~cele organizacji i~chce uczestniczyć w~ich realizacji,
    \item pasja~- wykonywana praca jest zgodna z~cechami, aspiracjami i~zainteresowaniami pracownika,
    \item efektywne działanie na rzecz pracodawcy~- pracownik dąży do wykorzystania pełni swojego potencjału, aktywnie dzieli się wiedzą.
\end{itemize}

%Tak zdefiniowane zaangażowanie związane jest z~satysfakcją z~pracy, lojalnością wobec pracodawcy, a~także gotowością do poświęceń.


\section{Koszt fluktuacji pracowników}\label{sec:koszt-fluktuacji}

Z~perspektywy menadżerskiej wydawać by się mogło, że głównym kosztem związanym z~fluktuacją pracowników jest koszt prowadzenia rekrutacji przez dział HR.
Problem jest jednak zdecydowanie bardziej złożony.
Edwards i~Philips \cite{philips-edwards-2009} pokazują, że~- w~zależności od stanowiska i~wymaganych na nim kompetencji -
całkowity koszt związany z~odejściem pracownika i~zatrudnieniem w~jego miejsce nowego oscyluje od 30 do nawet 400 procent
rocznego wynagrodzenia na danym stanowisku. Wyszczególnione przez nich koszty przedstawiono w~tabeli~\ref{tabela:fluktuacja-koszty}.

\noindent\begin{minipage}{\textwidth}
             \begin{table}[H]
                 \raggedright\caption{Typy kosztów związanych z~fluktuacją\label{tabela:fluktuacja-koszty}}
                 \begin{center}
                     \begin{tabular}{|P{.15\textwidth}|P{.75\textwidth}|}

                         \hline
                         \cellgray{Typ} &
                         \cellgray{Cechy} \\
                         \hline

                         koszty związane z~odejściem starego pracownika &
                         \begin{itemize}
                             \item przekazanie wiedzy innym pracownikom,
                             \item po podjęciu decyzji o~odejściu z~pracy, odchodzący pracownik może być mniej zaangażowana w~wykonywane obowiązki;
                         \end{itemize} \\

                         \hline

                         koszt prowadzenia rekrutacji &
                         \begin{itemize}
                             \item koszt związany z~publikowaniem ogłoszeń o~pracę,
                             \item selekcja aplikantów,
                             \item prowadzenie rozmów rekrutacyjnych,
                             \item koszt operacyjny zakontraktowania nowego pracownika~- związany między innymi z~procesowaniem umowy czy skierowaniem na badania lekarskie;
                         \end{itemize} \\
                         \hline

                         koszty związane z~wdrożeniem nowego pracownika &
                         \begin{itemize}
                             \item czas nowego pracownika potrzebny na zapoznanie się z~obowiązkami i~wdrożenie na nowe stanowisko pracy (w zależności od stanowiska może to trwać nawet kilka miesięcy),
                             \item czas doświadczonych pracowników potrzebny na wdrażanie nowego pracownika,
                             \item w~zależności od specyfiki stanowiska zakup odpowiedniego sprzętu dla pracownika, np ubrań roboczych czy laptopa;
                         \end{itemize} \\
                         \hline

                         szacowane utracone korzyści &
                         \begin{itemize}
                             \item potencjalnie większe obciążenie pracowników którzy pozostali w~firmie (przy brakach kadrowych),
                             \item utrata części wiedzy odchodzącego pracownika;
                         \end{itemize} \\
                         \hline

                         inne koszty powiązane &
                         \begin{itemize}
                             \item możliwe pogorszenie relacji z~klientem, co może negatywnie wpłynąć na sprzedaż,
                             \item odejście pracownika może zachęcić innych do rozważenia zmiany pracy;
                         \end{itemize} \\
                         \hline
                     \end{tabular}
                 \end{center}
                 \raggedright\source{\workbasedon{\cite{philips-edwards-2009}}}
                 \vspace{0.75cm}
             \end{table}
\end{minipage}


\section{Sposoby na retencję pracowników}\label{sec:retencja}

Przy rozważaniach na temat zwiększenia retencji pracowników warto zwrócić uwagę czynniki fluktuacji z~modelu Marcha i~Simona,
które zostały opisane w~sekcji \ref{sec:czynniki-wplywajace-na-fluktuacje}.
Wynika z~nich, że przedsiębiorstwa powinny dążyć do zwiększenia zadowolenia pracowników oraz do zwiększenia ich zaangażowania organizacyjnego.

Postulaty te są ujęte w~założeniach koncepcji "employer branding" \cite{spychala-2019}.
W~literaturze koncepcja ta jest definiowana jako
całokształt działań danej organizacji ukierunkowanych do pracowników obecnych, byłych oraz potencjalnych,
które służą kreowaniu wizerunku firmy jako atrakcyjnego miejsca pracy,
jednocześnie tworząc odpowiednie środowisko do realizacji jej celów biznesowych \cite{kozlowski-2012}.

Akcje związane z~koncepcją employer branding są podejmowane na 2 poziomach relacji firmy \cite{spychala-2019}:
\begin{enumerate}
    \item z~obecnymi pracownikami,
    \item z~byłymi pracownikami oraz potencjalnymi przyszłymi pracownikami.
\end{enumerate}

W~celu zarządzania tymi relacjami firmy podejmują akcje marketingowe~- w~przypadku relacji z~obecnymi pracownikami jest to marketing wewnętrzny,
natomiast w~przypadku byłych i~potencjalnych pracowników marketing zewnętrzny.
Marketing wewnętrzny opiera się na relacjach pracowników z~współpracownikami i~przełożonymi.
Aby zwiększyć przywiązanie organizacyjne i~utożsamianie się pracowników z~celami przedsiębiorstwa przełożeni powinni podejmować następujące akcje \cite{spychala-2019}:
\begin{itemize}
    \item transparentne przedstawianie pracownikom powodów kryjących się za działaniami firmy,
    \item wykazywanie empatii wobec pracowników,
    \item promowanie indywidualnego rozwoju pracowników,
    \item umożliwianie pracownikom pracy dającej poczucie dawania istotnego wkładu w~działanie firmy.
\end{itemize}

Marketing zewnętrzny nakierowany jest na zwiększenie rozpoznowalności przedsiębiorstwa i~budowaniu pozytywnych skojarzeń z~nim związanych.
Z~tego powodu wiele firm angażuje się w~akcje związane z~tzw. "społeczną odpowiedzialnością biznesu".
Takie akcje mogą być związane np. z~przekazywaniem środków na cele charytatywne czy promowaniem ograniczania emisji spalin.
Inną formą akcji marketingowych zwiększających zasięg firmy wśród potencjalnych pracowników jest prowadzenie prelekcji dla studentów kierunków związanych z~działalnością firmy.
Należy podkreślić, że marketing zewnętrzny opiera się w~dużej mierze na opiniach pracowników, dlatego ważne żeby wpierw dopracowany był marketing wewnętrzny \cite{spychala-2019}.

Skuteczny employer branding opiera się na następujących założeniach \cite{spychala-2019}:
\begin{itemize}
    \item pracownicy traktowani są jako klienci przedsiębiorstwa na równi z~konsumentami, czy~ogólnie~-~klientami~zewnętrznymi,
    \item pracownik i~pracodawca są równie istotni żeby możliwe było zrealizowanie celów organizacji,
    \item pracownicy tworzą wewnętrzny rynek pracy~- firma powinna umożliwiać pracownikom objęcie nowych wakatów w~celu umożliwienia pracownikom rozwoju i~zaspokojenia ich ambicji,
    \item postrzeganie firmy zależy przede wszystkim od pracowników.
\end{itemize}
%W zakresie relacji firma~- potencjalny przyszły pracownik istotne jest budowanie zewnętrznego wizerunku firmy.
%W tabeli \ref{tabela:employer-branding-potencjalny-pracownik} przedstawiono źródła z~których potencjalni pracownicy mogą
%uzyskiwać informacje o~firmie oraz po krótce scharakteryzowano ich cechy.

%\noindent\begin{minipage}{\textwidth}
%             \begin{table}[H]
%                 \raggedright\caption{Źródła informacji o~firmie dla potencjalnych pracowników\label{tabela:employer-branding-potencjalny-pracownik}}
%                 \begin{center}
%                     \begin{tabular}{|P{.15\textwidth}|p{.75\textwidth}|}
%
%                         \hline
%                         \cellgray{Źródło} &
%                         \multicolumn{1}{>{\centering\arraybackslash}m{.75\textwidth}|}{\cellgray{Cechy}} \\
%                         \hline
%
%                         obecni pracownicy &
%                         Pracownicy zadowoleni z~pracy budują pozytywną opinię firmy i~mogą przekazywać tę opinię potencjalnym pracownikom.\\
%
%                         \hline
%
%                         byli pracownicy &
%                         Jeśli organizajca dba o~zaspokajanie potrzeb pracowników i~tworzenie przyjaznego środowiska pracy
%                         to jest szansa, że pracownicy po dobrowolnym opuszczeniu firmy będą mieć o~niej pozytywną opinię.
%                         Wtedy nie tylko mogą polecać firmę innym potencjalnym przyszłym pracownikom,
%                         ale sami również stają się potencjalnymi przyszłymi pracownikami.\\
%                         \hline
%
%                         marka firmy &
%                         Rozpoznawalne marki mogą łatwiej przyciągać potencjalnych pracowników. \\
%                         \hline
%
%                         pozaorganizacyjna działalność firmy &
%                         Aby zwiększyć rozpoznawalność firmy wśród potencjalnych pracowników firmy angażują się w~wiele działalności pozaorganizacyjnych takich
%                         \begin{itemize}
%                             \item możliwe pogorszenie relacji z~klientem, co może negatywnie wpłynąć na sprzedaż
%                             \item odejście pracownika może zachęcić innych do rozważenia zmiany pracy
%                         \end{itemize} \\
%                         \hline
%                     \end{tabular}
%                 \end{center}
%                 \raggedright\source{\workbasedon{\longcite{philips-edwards-2009}}}
%                 \vspace{0.75cm}
%             \end{table}
%\end{minipage}
%
%Potencjalni przyszli pracownicy mogą uzyskiwać informacje o~firmie z~kilku źródeł:
%
%\begin{itemize}
%    \item od obecnych i~byłych pracowników firmy
%    \item w~oparciu o~efekty działania firmy
%    \item dzięki pozaorganizacyjnym działalnościom firmy
%    \end{itemize}
%Działania związane z~realizacją koncepcji employer branding powinny być r


\thispagestyle{normal}
