% !TeX spellcheck = pl_PL


\chapter{Problem fluktuacji pracowników w~literaturze}\label{ch:knowladge-state}


\section{Zjawisko fluktuacji}\label{sec:zjawisko-fluktuacji}
%\todo{zjawisko fluktuacji + fluktuacja w~zależności od branży}
Fluktuacja pracowników (ang. employee turnover) w~literaturze definiowana jest na wiele sposób, jednak najczęstsze to "dobrowolne odejście z~pracy pracowników dojrzałych"\cite{wozniak-2016}
lub szerzej jako "tempo w~jakim pracownicy opuszczają firmę"\cite{cron-2006}.
Czasami fluktuacja jest utożsamiana z~rotacją pracowników (ang. job rotation),
jednak częściej rotację definiuje się nie jako samo opuszczanie firmy przez pracowników, a~raczej jako proces wymiany pracowników,
który może służyć jako metoda rozwoju dla pracowników (np. poprzez przeniesienie pracownika na inne stanowisko wewnątrz organizacji lub awans)\cite{pocztowski-2009}.
Wysoka fluktuacja może nie być pożądana choćby ze względu na wysokie koszty z~nią związane\cite{philips-edwards-2009},
więc odpowiedzią pracodawcy może być próba jej ograniczenia.
Proces taki określany jest mianem retencji pracowników i~jego głównym celem jest utrzymanie zatrudnienia na poziomie pozwalającym na "sprawną realizację procesów biznesowych"\cite{pocztowski-2007}.

Fluktuacja może być skategoryzowana na kilka różnych sposobów\cite{spychala-2019}:
\begin{itemize}
    \item pożądana i~niepożądana - fluktuacja jest pożądana gdy opuszczenie pracownika pozytywnie wpływa na działanie firmy (np. pracownik o~niskich kwalifikacjach), a~niepożądana gdy pracownik jest trudny do zastąpienia i~jego pracy przynosi firmie korzyści,
    \item dobrowolna i~niedobrowolna - dobrowolna występuje gdy pracownik sam odchodzi z~firmy, a~niedobrowolna gdy pracodawca zwalnia pracownika,
    \item do uniknięcia i~nie do uniknięcia - fluktuacja możliwa do uniknięcia występuje gdy firma jest jej w~stanie zapobiec, nie możliwa do uninięcia kiedy zależy od czynników zewnętrznych, na które firma nie ma wpływu (np. wprowadzenie nowych regulacji prawnych),
    \item nadmierna - związana ściśle z~wewnętrznym działaniem firmy - cechy takie jak złe warunki pracy i~nieadekwatne wynagrodzenie mogą wpłynąć na zwiększenie fluktuacji pracowników w~danym przedsiębiorstwie.
\end{itemize}

Do mierzenia poziomu fluktuacji wykorzystuje się wskaźnik fluktuacji definiowany jako stosunek osób opuszczającej w~firmę w~danym roku do średniej liczby pracowników zatrudnionych w~danym roku.
Do wyliczenia fluktuacji można wykorzystać liczbę wszystkich pracowników, którzy opuścili organizację - niezależnie od powodu opuszczenia tejże organizacji - ale można też obliczyć ten wskaźnik uwzględniając jedynie fluktuację dobrowolną.
Badanie fluktuacji dobrowolnej jest o~tyle istotne, że może pozwolić na wykrycie problemów w~firmie i~opracowanie sposobu na ich przeciwdziałanie\cite{spychala-2019}.
Co więcej, badania pokazują, że większość odejść w~organizacjach stanowią właśnie odejścia dobrowolne\cite{dalton-1982}.
Z~tego względu, dalsze rozważania będą dotyczyły fluktuacji dobrowolnej, o~ile wprost nie będzie napisane inaczej.


\section{Wpływ fluktuacji na firmę}\label{sec:wplyw-fluktuacji-na-firme}
Fluktuacja pracowników niekoniecznie musi oznaczać problem dla przedsiębiorstwa.
Z~analizy badań Human Capital Index przeprowadzonych przez firmę Watson Wyatt w~2005 r. wynika,
że zarówno bardzo niska i~bardzo wysoka fluktuacja nie są korzystne dla przedsiębiorstw.
Organizacje z~umiarkowanym poziomem wskaźnika fluktuacji wynoszącym ok 15\%
miały zwrot z~inwestycji akcjonariuszy (ang. total shareholders return - TSR) na poziomie 43\%
co stanowiło rezultat średnio o~9 punktów procentowych lepszy niż firmy o~niższym lub o~wyższym wskaźniku fluktuacji\cite{krol-ludwiczynski-2006}.

Na to jaki wpływ określony poziom fluktuacji znaczący wpływ ma charakterystyka branży\cite{taylor-2006}.
Heurystyka dla określenia czy poziom fluktuacji jest zbyt wysoki prezentuje się następująco:
\begin{itemize}
    \item niedobór kandydatów o~odpowiednich kompetencjach na rynku pracy,
    \item fluktuacja jest wyższa niż u~bezpośredniej konukrencji,
    \item wysokie koszty rekrutacji.
\end{itemize}
Natomiast dla określenia czy wysoki poziom fluktuacji może być akceptowalny wygląda następująco:
\begin{itemize}
    \item wielu kandydatów o~odpowiednich kompetencjach na rynku pracy,
    \item niskie koszty rekrutacji,
    \item niski koszt wdrożenia nowego pracownika,
    \item niskie ryzyko utraty wiedzy w~wyniku odejścia pracownika,
    \item przewidywanie redukcji etatów w~niedalekiej przyszłości.
\end{itemize}


\section{Przyczyny fluktuacji pracowników}\label{sec:czynniki-wplywajace-na-fluktuacje}

Najstarszy model opisujący dobrowolną fluktuację pracowników został opracowany przez Marcha i~Simona w~1958 roku.
Model ten wyróżnia 2 czynniki wpływające istotnie na poziom fluktuacji\cite{wozniak-2012}:
\begin{itemize}
    \item to jak pracownik ocenia swoją chęć zmiany pracy - związane jest to głównie z~niską satysfakcją z~wykonywanej pracy oraz niskim zaangażowaniem organizacyjnym
    \item to jak pracownik ocenia łatwość zmiany pracy - związane jest to z~dostępnością na rynku pracy posad interesujących pracownika
\end{itemize}
W~ciągu ostatnich 50 lat powstało wiele modeli prognozujących fluktuację i~większość z~nich korzysta mniej lub bardziej bezpośrednio z~cech wyróżnionych przez Marcha i~Simona.
Badania empiryczne prowadzone w~tym czasie potwierdziły, że te cechy mają znaczący wpływ na fluktuację - jednak nie oddają w~pełni istoty fluktuacji.
Przy pracy nad tymi modelami wyszczególniono wiele drugorzędnych cech wpływających na fluktuację.
Cechy te zostały przedstawione w~tabeli~\ref{tabela:fluktuacja-cechy}.

\noindent\begin{minipage}{\textwidth}
             \begin{table}[H]
                 \raggedright\caption{Najczęściej pojawiające się cechy w~modelach fluktuacji\label{tabela:fluktuacja-cechy}}
                 \begin{center}
                     \begin{tabular}{|P{.2\textwidth}|P{.7\textwidth}|}

                         \hline
                         \cellgray{Kategoria} &
                         \cellgray{Cechy} \\
                         \hline

                         cechy osobiste &
                         \begin{itemize}
                             \item osobowość
                             \item wyznawane wartości
                             \item wiek
                             \item staż pracy
                             \item wiedza
                             \item doświadczenie
                             \item profesjonalizm
                             \item odpowiedzialność wobec rodziny
                         \end{itemize} \\

                         \hline

                         cechy stanowiska pracy &
                         \begin{itemize}
                             \item postrzeganie pracy
                             \item skomplikowanie pracy
                             \item oczekiwania wobec wykonywanej pracy
                             \item wynagrodzenie i~benefity
                             \item koszt zmiany pracy
                             \item stres
                             \item dopasowanie wykonywanej pracy do oczekiwań pracownika
                             \item rozmiar firmy
                         \end{itemize} \\
                         \hline

                         mechanizmy zmiany stanowiska &
                         \begin{itemize}
                             \item chęć zmiany
                             \item oczekiwania względem przyszłej pracy
                             \item wysiłek potrzebny do zmiany bierzącej sytuacji
                             \item możliwość przejścia do firmy powiązanej lub innego oddziału firmy
                             \item możliwość awansu lub degradacji
                             \item alternatywne sposoby opuszczenia pracy
                         \end{itemize} \\
                         \hline

                         konsekwencje opuszczenia lub pozostania w~firmie &
                         \begin{itemize}
                             \item konsekwencje pozapracowe
                             \item wydajność pracy
                         \end{itemize} \\
                         \hline

                         mechanizmy wpływające na proces decyzyjny &
                         \begin{itemize}
                             \item zdarzenia nieprzewidziane
                             \item myśli o~odejściu z~firmy
                         \end{itemize} \\
                         \hline
                     \end{tabular}
                 \end{center}
                 \raggedright\source{\workbasedon{\bibentry{steel-2009}\nolink{steel-2009}}}
                 \vspace{0.75cm}
             \end{table}
\end{minipage}
%\section{Zadowolenie pracowników}\label{sec:employee-happiness}
%\todo{zadowolenie pracownikó}
%\todo{wątek pandemii}
%\todo{fluktuacja pracowników a~praca zdalna}


\section{Koszt fluktuacji pracowników}\label{sec:koszt-fluktuacji}

Z perspektywy menadżerskiej wydawać by się mogło, że głównym kosztem związanym z~fluktuacją pracowników jest koszt prowadzenia rekrutacji przez dział HR.
Problem jest jednak zdecydowanie bardziej złożony.
Edwards i~Philips\cite{philips-edwards-2009} pokazują, że - w~zależności od stanowiska i~wymaganych na nim kompetencji -
całkowity koszt związany z~odejściem pracownika i~zatrudnieniem w~jego miejsce nowego oscyluje od 30 do nawet 400 procent
rocznego wynagrodzenia na danym stanowisku. Wyszczególnione przez nich koszty przedstawiono w~tabeli~\ref{tabela:fluktuacja-koszty}.

\noindent\begin{minipage}{\textwidth}
             \begin{table}[H]
                 \raggedright\caption{Typy kosztów związanych z~fluktuacją\label{tabela:fluktuacja-koszty}}
                 \begin{center}
                     \begin{tabular}{|P{.15\textwidth}|P{.75\textwidth}|}

                         \hline
                         \cellgray{Typ} &
                         \cellgray{Cechy} \\
                         \hline

                         koszty związane z~odejściem starego pracownika &
                         \begin{itemize}
                             \item przekazanie wiedzy innym pracownikom
                             \item po podjęciu decyzji o~odejściu z~pracy, odchodzący pracownik może być mniej zaangażowana w~wykonywane obowiązki
                         \end{itemize} \\

                         \hline

                         koszt prowadzenia rekrutacji &
                         \begin{itemize}
                             \item koszt związany z~publikowaniem ogłoszeń o~pracę
                             \item selekcja aplikantów
                             \item prowadzenie rozmów rekrutacyjnych
                             \item koszt operacyjny zakontraktowania nowego pracownika - związany między innymi z~procesowaniem umowy czy skierowaniem na badania lekarskie
                         \end{itemize} \\
                         \hline

                         koszty związane z~wdrożeniem nowego pracownika &
                         \begin{itemize}
                             \item czas nowego pracownika potrzebny na zapoznanie się z~obowiązkami i~wdrożenie na nowe stanowisko pracy (w zależności od stanowiska może to trwać nawet kilka miesięcy)
                             \item czas doświadczonych pracowników potrzebny na wdrażanie nowego pracownika
                             \item w~zależności od specyfiki stanowiska zakup odpowiedniego sprzętu dla pracownika, np ubrań roboczych czy laptopa
                         \end{itemize} \\
                         \hline

                         szacowane utracone korzyści &
                         \begin{itemize}
                             \item potencjalnie większe obciążenie pracowników którzy pozostali w~firmie (przy brakach kadrowych)
                             \item utrata części wiedzy odchodzącego pracownika
                         \end{itemize} \\
                         \hline

                         inne koszty powiązane &
                         \begin{itemize}
                             \item możliwe pogorszenie relacji z~klientem, co może negatywnie wpłynąć na sprzedaż
                             \item odejście pracownika może zachęcić innych do rozważenia zmiany pracy
                         \end{itemize} \\
                         \hline
                     \end{tabular}
                 \end{center}
                 \raggedright\source{\workbasedon{\bibentry{philips-edwards-2009}\nolink{philips-edwards-2009}}}
                 \vspace{0.75cm}
             \end{table}
\end{minipage}


\section{Sposoby na retencję pracowników}\label{sec:retencja}

Przy rozważaniach na temat zwiększenia retencji pracowników warto zwrócić uwagę czynniki fluktuacji z modelu Marcha i Simona,
które zostały opisane w sekcji \ref{sec:czynniki-wplywajace-na-fluktuacje}.
Wynika z nich, że przedsiębiorstwa powinny dążyć do zwiększenia zadowolenia pracowników oraz do zwiększenia ich zaangażowania organizacyjnego.
Akcje te powinny być podejmowane przekrojowo w całej organizacji na 3 poziomach relacji:
\begin{itemize}
    \item firma - pracownik
    \item firma - były pracownik
    \item firma - potencjalny przyszły pracownik
\end{itemize}

Kluczem do pracy w zakresie tych relacji jest koncepcja "employer branding"\cite{spychala-2019}.
W literaturze koncepcja ta jest definiowana jako
całokształt działań danej organizacji ukierunkowanych do pracowników obecnych, byłych oraz potencjalnych,
które służą kreowaniu wizerunku firmy jako atrakcyjnego miejsca pracy,
jednocześnie tworząc odpowiednie środowisko do realizacji jej celów biznesowych\cite{kozlowski-2012}.

%Działania związane z realizacją koncepcji employer branding powinny być r


\thispagestyle{normal}
