% !TeX spellcheck = pl_PL


\chapter{Problem fluktuacji pracowników w literaturze}\label{ch:knowladge-state}


\section{Zjawisko fluktuacji}\label{sec:employee-turnover}
\todo{zjawisko fluktuacji + fluktuacja w zależności od branży}


\section{Zadowolenie pracowników}\label{sec:employee-happiness}
\todo{zadowolenie pracownikó}
\todo{wątek pandemii}
\todo{fluktuacja pracowników a praca zdalna}


\section{Koszt fluktuacji pracowników}\label{sec:employee-selection}

Z perspektywy menadżerskiej wydawać by się mogło, że głównym kosztem związanym z fluktuacją pracowników jest koszt prowadzenia rekrutacji przez dział HR.
Problem jest jednak zdecydowanie bardziej złożony.
Edwards i Philips\cite{philips-edwards-2009} pokazują, że - w zależności od stanowiska i wymaganych na nim kompetencji -
całkowity koszt związany z odejściem pracownika i zatrudnieniem w jego miejsce nowego oscyluje od 30 do nawet 400 procent
rocznego wynagrodzenia na danym stanowisku.

W swojej pracy wyszczególnili 5 typów kosztów:
\begin{enumerate}
    \item koszty związane z odejściem starego pracownika
    \begin{itemize}
        \item odchodzący pracownik zwykle musi przekazać swoją wiedzę i obowiązki innym pracownikom -
        wiąże się to więc z poświeceniem znacznej ilości czasu co najmniej 2 pracowników
        \item po podjęciu decyzji o odejściu z pracy, odchodząca osoba jest zwykle mniej zaangażowana w wykonywane obowiązki, jej praca będzie więc mniej efektywna
    \end{itemize}
    \item koszt prowadzenia rekrutacji
    \begin{itemize}
        \item koszt związany z publikowaniem ogłoszeń o pracę
        \item selekcja aplikantów
        \item prowadzenie rozmów rekrutacyjnych - na specjalistyczne stanowiska np w branży IT cały cykl rozmów rekrutacyjncych z kandydatem może trwać nawet kilka godzin i angażować 2-3 pracowników
        \item koszt operacyjny zakontraktowania nowego pracownika związany między innymi z procesowaniem umowy czy skierowaniem na badania lekarskie
    \end{itemize}
    \item koszty związane z wdrożeniem nowego pracownika
    \begin{itemize}
        \item czas nowego pracownika potrzebny na zapoznanie się z obowiązkami i wdrożenie na nowe stanowisko pracy - na stanowiskach wymagających znajomości wewnętrznych procedur firmy wdrażanie nowego pracownika może trwać nawet kilka miesięcy
        \item czas doświadczonych pracowników potrzebny na wdrażania nowego pracownika
        \item w zależności od specyfiki danego stanowiska w koszty wdrożenia nowego pracownika może wchodzić także zakup odpowiedniego sprzętu dla pracownika, np ubrań roboczych czy laptopa
    \end{itemize}

    \item szacowane utracone korzyści
    \begin{itemize}
        \item większe obciążenie pracowników którzy pozostali w firmie
        \item utrata części wiedzy odchodzącego pracownika
        \item jeśli odchodzący pracownik jako jedyny w firmie posiadał określone kompetencje do wykonania określonych czynności, jego odejście może doprowadzić do pewnych przestojów
    \end{itemize}
    \item inne koszty powiązane
    \begin{itemize}
        \item możliwe pogorszenie relacji z klientem, co może negatywnie wpłynąć na sprzedaż
        \item odejście pracownika może zachęcić innych pracowników do rozważenia zmiany pracy
    \end{itemize}
\end{enumerate}
\thispagestyle{normal}
