% !TeX spellcheck = pl_PL


\chapter{Problem fluktuacji pracowników w literaturze}\label{ch:knowladge-state}


\section{Zjawisko fluktuacji}\label{sec:zjawisko-fluktuacji}
%\todo{zjawisko fluktuacji + fluktuacja w zależności od branży}
Fluktuacja pracowników (ang. employee turnover) w literaturze definiowana jest na wiele sposób, jednak najczęstsze to "dobrowolne odejście z pracy pracowników dojrzałych"\cite{wozniak-2016}
lub szerzej jako "tempo w jakim pracownicy opuszczają firmę"\cite{cron-2006}.
Czasami fluktuacja jest utożsamiana z rotacją pracowników (ang. job rotation),
jednak częściej rotację definiuje się nie jako samo opuszczanie firmy przez pracowników, a raczej jako proces wymiany pracowników,
który może służyć jako metoda rozwoju dla pracowników (np. poprzez przeniesienie pracownika na inne stanowisko wewnątrz organizacji lub awans)\cite{pocztowski-2009}.
Wysoka fluktuacja może nie być pożądana choćby ze względu na wysokie koszty z nią związane\cite{philips-edwards-2009},
więc odpowiedzią pracodawcy może być próba jej ograniczenia.
Proces taki określany jest mianem retencji pracowników i jego głównym celem jest utrzymanie zatrudnienia na poziomie pozwalającym na "sprawną realizację procesów biznesowych"\cite{pocztowski-2007}.

Fluktuacja może być skategoryzowana na kilka różnych sposobów\cite{spychala-2019}:
\begin{itemize}
    \item pożądana i niepożądana - fluktuacja jest pożądana gdy opuszczenie pracownika pozytywnie wpływa na działanie firmy (np. pracownik o niskich kwalifikacjach), a niepożądana gdy pracownik jest trudny do zastąpienia i jego pracy przynosi firmie korzyści,
    \item dobrowolna i niedobrowolna - dobrowolna występuje gdy pracownik sam odchodzi z firmy, a niedobrowolna gdy pracodawca zwalnia pracownika,
    \item do uniknięcia i nie do uniknięcia - fluktuacja możliwa do uniknięcia występuje gdy firma jest jej w stanie zapobiec, nie możliwa do uninięcia kiedy zależy od czynników zewnętrznych, na które firma nie ma wpływu (np. wprowadzenie nowych regulacji prawnych),
    \item nadmierna - związana ściśle z wewnętrznym działaniem firmy - cechy takie jak złe warunki pracy i nieadekwatne wynagrodzenie mogą wpłynąć na zwiększenie fluktuacji pracowników w danym przedsiębiorstwie.
\end{itemize}

Do mierzenia poziomu fluktuacji wykorzystuje się wskaźnik fluktuacji definiowany jako stosunek osób opuszczającej w firmę w danym roku do średniej liczby pracowników zatrudnionych w danym roku.
Do wyliczenia fluktuacji można wykorzystać liczbę wszystkich pracowników, którzy opuścili organizację - niezależnie od powodu opuszczenia tejże organizacji - ale można też obliczyć ten wskaźnik uwzględniając jedynie fluktuację dobrowolną.
Badanie fluktuacji dobrowolnej jest o tyle istotne, że może pozwolić na wykrycie problemów w firmie i opracowanie sposobu na ich przeciwdziałanie\cite{spychala-2019}.
Co więcej, badania pokazują, że większość odejść w organizacjach stanowią właśnie odejścia dobrowolne\cite{dalton-1982}.
Z tego względu, dalsze rozważania będą dotyczyły fluktuacji dobrowolnej, o ile wprost nie będzie napisane inaczej.


\section{Wpływ fluktuacji na firmę}\label{sec:wplyw-fluktuacji-na-firme}
Fluktuacja pracowników niekoniecznie musi oznaczać problem dla przedsiębiorstwa.
Z analizy badań Human Capital Index przeprowadzonych przez firmę Watson Wyatt w 2005 r. wynika,
że zarówno bardzo niska i bardzo wysoka fluktuacja nie są korzystne dla przedsiębiorstw.
Organizacje z umiarkowanym poziomem wskaźnika fluktuacji wynoszącym ok 15\%
miały zwrot z inwestycji akcjonariuszy (ang. total shareholders return - TSR) na poziomie 43\%
co stanowiło rezultat średnio o 9 punktów procentowych lepszy niż firmy o niższym lub o wyższym wskaźniku fluktuacji\cite{krol-ludwiczynski-2006}.

Na to jaki wpływ określony poziom fluktuacji znaczący wpływ ma charakterystyka branży.\cite{taylor-2006}
Heurystyka dla określenia czy poziom fluktuacji jest zbyt wysoki prezentuje się następująco:
\begin{itemize}
    \item niedobór kandydatów o odpowiednich kompetencjach na rynku pracy,
    \item fluktuacja jest wyższa niż u bezpośredniej konukrencji,
    \item wysokie koszty rekrutacji.
\end{itemize}
Natomiast dla określenia czy wysoki poziom fluktuacji może być akceptowalny wygląda następująco:
\begin{itemize}
    \item wielu kandydatów o odpowiednich kompetencjach na rynku pracy,
    \item niskie koszty rekrutacji,
    \item niski koszt wdrożenia nowego pracownika,
    \item niskie ryzyko utraty wiedzy w wyniku odejścia pracownika,
    \item przewidywanie redukcji etatów w niedalekiej przyszłości.
\end{itemize}


\section{Przyczyny fluktuacji pracowników}\label{sec:czynniki-wplywajace-na-fluktuacje}

Najstarszy model opisujący dobrowolną fluktuację pracowników został opracowany przez Marcha i Simona w 1958 roku.
Model ten wyróżnia 2 czynniki wpływające istotnie na poziom fluktuacji\cite{wozniak-2012}:
\begin{itemize}
    \item to jak pracownik ocenia swoją chęć zmiany pracy - związane jest to głównie z niską satysfakcją z wykonywanej pracy oraz niskim zaangażowaniem organizacyjnym
    \item to jak pracownik ocenia łatwość zmiany pracy - związane jest to z dostępnością na rynku pracy posad interesujących pracownika
    \end{itemize}
W ciągu ostatnich 50 lat powstało wiele modeli prognozujących fluktuację i większość z nich korzysta mniej lub bardziej bezpośrednio z cech wyróżnionych przez Marcha i Simona.
Badania empiryczne prowadzone w tym czasie potwierdziły, że te cechy mają znaczący wpływ na fluktuację - jednak nie oddają w pełni istoty fluktuacji.
Wyszczególniono wiele drugorzędnych cech wpływających na fluktuację.
Najczęściej pojawiające się w różnych teoriach cechy można skatogeryzować następująco\cite{steel-2009}:
\begin{itemize}
    \item cechy osobiste (osobowość, wyznawane wartości, wiek, staż pracy, wiedza i doświadczenie, profesjonalizm, odpowiedzialność wobec rodziny)
    \item cechy stanowiska pracy (postrzeganie pracy, skomplikowanie pracy, oczekiwania wobec wykonywanej pracy, wynagrodzenie i benefity, koszt zmiany pracy, stres, dopasowanie wykonywanej pracy do oczekiwań pracownika, rozmiar firmy)
    \item mechanizmy zmiany stanowiska (chęć zmiany, oczekiwania względem przyszłej pracy, wysiłek potrzebny do zmiany bierzącej sytuacji, możliwość przejścia do firmy powiązanej lub innego oddziału firmy, możliwość awansu lub degradacji, alternatywne sposoby opuszczenia pracy)
    \item konsekwencje opuszczenia lub pozostania w firmie (konsekwencje pozapracowe, wydajność pracy)
    \item mechanizmy wpływające na proces decyzyjny (zdarzenia nieprzewidziane, myśli o odejściu z firmy)
    \end{itemize}

%\section{Zadowolenie pracowników}\label{sec:employee-happiness}
%\todo{zadowolenie pracownikó}
%\todo{wątek pandemii}
%\todo{fluktuacja pracowników a praca zdalna}


\section{Koszt fluktuacji pracowników}\label{sec:koszt-fluktuacji}

Z perspektywy menadżerskiej wydawać by się mogło, że głównym kosztem związanym z fluktuacją pracowników jest koszt prowadzenia rekrutacji przez dział HR.
Problem jest jednak zdecydowanie bardziej złożony.
Edwards i Philips\cite{philips-edwards-2009} pokazują, że - w zależności od stanowiska i wymaganych na nim kompetencji -
całkowity koszt związany z odejściem pracownika i zatrudnieniem w jego miejsce nowego oscyluje od 30 do nawet 400 procent
rocznego wynagrodzenia na danym stanowisku.

W swojej pracy wyszczególnili 5 typów kosztów:
\begin{enumerate}
    \item koszty związane z odejściem starego pracownika
    \begin{itemize}
        \item odchodzący pracownik zwykle musi przekazać swoją wiedzę i obowiązki innym pracownikom -
        wiąże się to więc z poświeceniem znacznej ilości czasu co najmniej 2 pracowników
        \item po podjęciu decyzji o odejściu z pracy, odchodząca osoba jest zwykle mniej zaangażowana w wykonywane obowiązki, jej praca będzie więc mniej efektywna
    \end{itemize}
    \item koszt prowadzenia rekrutacji
    \begin{itemize}
        \item koszt związany z publikowaniem ogłoszeń o pracę
        \item selekcja aplikantów
        \item prowadzenie rozmów rekrutacyjnych - na specjalistyczne stanowiska np w branży IT cały cykl rozmów rekrutacyjncych z kandydatem może trwać nawet kilka godzin i angażować 2-3 pracowników
        \item koszt operacyjny zakontraktowania nowego pracownika związany między innymi z procesowaniem umowy czy skierowaniem na badania lekarskie
    \end{itemize}
    \item koszty związane z wdrożeniem nowego pracownika
    \begin{itemize}
        \item czas nowego pracownika potrzebny na zapoznanie się z obowiązkami i wdrożenie na nowe stanowisko pracy - na stanowiskach wymagających znajomości wewnętrznych procedur firmy wdrażanie nowego pracownika może trwać nawet kilka miesięcy
        \item czas doświadczonych pracowników potrzebny na wdrażania nowego pracownika
        \item w zależności od specyfiki danego stanowiska w koszty wdrożenia nowego pracownika może wchodzić także zakup odpowiedniego sprzętu dla pracownika, np ubrań roboczych czy laptopa
    \end{itemize}

    \item szacowane utracone korzyści
    \begin{itemize}
        \item większe obciążenie pracowników którzy pozostali w firmie
        \item utrata części wiedzy odchodzącego pracownika
        \item jeśli odchodzący pracownik jako jedyny w firmie posiadał określone kompetencje do wykonania określonych czynności, jego odejście może doprowadzić do pewnych przestojów
    \end{itemize}
    \item inne koszty powiązane
    \begin{itemize}
        \item możliwe pogorszenie relacji z klientem, co może negatywnie wpłynąć na sprzedaż
        \item odejście pracownika może zachęcić innych pracowników do rozważenia zmiany pracy
    \end{itemize}
\end{enumerate}

\thispagestyle{normal}
