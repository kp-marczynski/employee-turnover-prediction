% !TeX~program = latexmk
% !TeX spellcheck = pl_PL
% !TeX~root = example.tex

\chapter{LaTeX demo}

\textcolor{red}{
    Niniejszy rozdział zawiera prezentację możliwości przygotowanego szablonu pracy
    i zostanie usunięty we właściwej pracy.
}

ĄĆĘŁŃÓŚŹŻ ąćęłńóśźż\footnote{Przykład użycia polskich znaków diakrytycznych oraz przypisu w~miejscu}.

Reszta dokumentacji znajduje się w\nolink{docker_compose_reference}.

Jak pisze Harel w\nolink{harel_rzecz_2008}: \lipsum[1]

Natomiast Kaleta uważał\nolink{kaleta_experimental_2005} \lipsum[1]

Jak widać na rys. \ref{fig:network} Docker ma wewnętrzną sieć.

\imagewidth[\bibentry{docker_compose_reference}\nolink{docker_compose_reference}]{0.6}{demo/img/swarm-network}{Docker ma sieć}{network}

Jak widać na rys.\ref{fig:kotek} Ala ma kota. \lipsum[1]

\imagewidth{0.4}{demo/img/kotek}{Ala ma kota}{kotek}

Co uwzględniono w~tabeli \ref{tabela:coktoma}. \lipsum[1]

\noindent\begin{minipage}{\textwidth}
    \begin{table}[H]
        \raggedright\caption{Co kto ma (patrz też dodatek~\ref{app:dod1}) \label{tabela:coktoma}}
        \begin{center}\begin{tabular}{|l|l|l|}% wyrównanie kolumn tabeli -> l~c r~- do lewej, środka, do prawej
                          \hline
                          Ala & ma & kota \\
                          \hline
                          Ola & ma & psa \\
                          \hline
                          Ula & ma & małpę\\
                          \hline
        \end{tabular}\end{center}
        \raggedright\source{\bibentry{harel_rzecz_2008}\nolink{harel_rzecz_2008}}
        \vspace{0.75cm}
    \end{table}
\end{minipage}

W~moim kodzie \ref{listing:liquibase} zrobiłem coś wspaniałego. \lipsum[1]
% lub {java} albo {bash} albo {text}
%\noindent\begin{minipage}{\textwidth}
%\begin{listing}[float=h!]
%    \caption{Przykładowy algorytm w~języku C} \label{listing:moj}
%    \begin{minted}{c}
%        int main()
%        {
%        int a=2*3;
%        printf("**Ala ma kota\n**");
%        while(!I2C_CheckEvent(I2C1, I2C_EVENT_MASTER_MODE_SELECT)); /* EV5 */
%        return 0;
%        }
%    \end{minted}
%    \raggedright\source{\ownwork}
%    \vspace{0.75cm}
%\end{listing}

%\end{minipage}
\noindent\begin{minipage}{\textwidth}
    \begin{lstlisting}[caption={Skrpyt ładujący dane z~pliku CSV}, label={listing:liquibase}]
    \end{lstlisting}
    \hspace{.075\textwidth}\begin{minipage}{.85\textwidth}
    \begin{minted}{xml}
<changeSet id="201902130001" author="kmarczynski" context="usda">
  <loadData file="config/liquibase/usda_sr_db/household_measure.csv"
    quotchar='"' separator=";" tableName="household_measure">
    <column header="product_id" name="product_id" type="numeric"/>
    <column header="measure_description" name="description" type="string"/>
    <column header="grams_weight" name="grams_weight" type="numeric"/>
    <column header="is_visible" name="is_visible" type="boolean"/>
  </loadData>
</changeSet>
    \end{minted}
    \end{minipage}

    \raggedright\source{\ownwork}
    \vspace{0.75cm}
\end{minipage}

\lipsum[1]

W tabelach \ref{tabela:rozwiazania-konkurencyjne-funkcjonalne}, \ref{tabela:sformulowanie-problemu}, \ref{tabela:uzytkownicy} przedstawiono przykładowe formatowanie tabel.

\noindent\begin{minipage}{\textwidth}
    \begin{table}[H]
        \raggedright\caption{Rozwiązania konkurencyjne~- cechy funkcjonalne\label{tabela:rozwiazania-konkurencyjne-funkcjonalne}}
        \begin{center}\begin{tabular}{|P{.22\textwidth}|P{.09\textwidth}|P{.09\textwidth}|P{.09\textwidth}|P{.09\textwidth}|P{.09\textwidth}|P{.09\textwidth}|}
            \hline
            & \cellgray{Rozw1}      & \cellgray{Rozw2}          & \cellgray{Rozw3}         & \cellgray{Rozw4}           & \cellgray{Rozw5}      & \cellgray{Rozw6}    \\ \hline
            \cellgray{Funkcjonalność 1}        & \cellgreen{TAK}       & \cellgreen{TAK}           & \cellgreen{TAK}          & \cellgreen{TAK}            & \cellgreen{TAK}       & \cellgreen{TAK}     \\ \hline
            \cellgray{Funkcjonalność 2}        & \cellgreen{TAK}       & \cellgreen{TAK}           & \cellgreen{TAK}          & \cellgreen{TAK}            & \cellred{NIE}         & \cellred{NIE}       \\ \hline
            \cellgray{Funkcjonalność 3}        & \cellgreen{TAK}       & \cellgreen{TAK}           & \cellgreen{TAK}          & \cellgreen{TAK}            & \cellgreen{TAK}       & \cellgreen{TAK}     \\ \hline
            \cellgray{Funkcjonalność 4}        & \cellgreen{TAK}       & \cellgreen{TAK}           & \cellgreen{TAK}          & \cellgreen{TAK}            & \cellred{NIE}         & \cellgreen{TAK}     \\ \hline
            \cellgray{Funkcjonalność 5}        & \cellgreen{TAK}       & \cellgreen{TAK}           & \cellgreen{TAK}          & \cellred{NIE}              & \cellred{NIE}         & \cellgreen{TAK}     \\ \hline
            \cellgray{Funkcjonalność 6}        & \cellgreen{TAK}       & \cellgreen{TAK}           & \cellgreen{TAK}          & \cellgreen{TAK}            & \cellgreen{TAK}       & \cellgreen{TAK}     \\ \hline
            \cellgray{Funkcjonalność 7}        & \cellgreen{TAK}       & \cellgreen{TAK}           & \cellgreen{TAK}          & \cellred{NIE}              & \cellgreen{TAK}       & \cellgreen{TAK}     \\ \hline
            \cellgray{Funkcjonalność 8}        & \cellgreen{TAK}       & \cellgreen{TAK}           & \cellgreen{TAK}          & \cellgreen{TAK}            & \cellgreen{TAK}       & \cellred{NIE}       \\ \hline
            \cellgray{Funkcjonalność 9}        & \cellgreen{TAK}       & \cellgreen{TAK}           & \cellgreen{TAK}          & \cellgreen{TAK}            & \cellgreen{TAK}       & \cellgreen{TAK}     \\ \hline
            \cellgray{Funkcjonalność 10}       & \cellgreen{TAK}       & \cellgreen{TAK}           & \cellgreen{TAK}          & \cellgreen{TAK}            & \cellgreen{TAK}       & \cellgreen{TAK}     \\ \hline
        \end{tabular}\end{center}
        \raggedright\source{\ownwork}
        \vspace{0.75cm}
    \end{table}
\end{minipage}

\noindent\begin{minipage}{\textwidth}
    \begin{table}[H]
        \raggedright\caption{Sformułowanie problemu\label{tabela:sformulowanie-problemu}}
        \begin{center}\begin{tabular}{|P{.2\textwidth}|p{.7\textwidth}|}

            \hline
            \cellgray{Problem} &
            \inlinetodo{todo} \\
            \hline

            \cellgray{Dotyczy} &
            \inlinetodo{todo} \\
            \hline

            \cellgray{Wpływ problemu} &
            \begin{itemize}
                \item \inlinetodo{todo}
                \item \inlinetodo{todo}
                \item \inlinetodo{todo}
            \end{itemize} \\
            \hline

            \cellgray{Pomyślne rozwiązanie} &
            \begin{itemize}
                \item \inlinetodo{todo}
                \item \inlinetodo{todo}
                \item \inlinetodo{todo}
            \end{itemize} \\
            \hline
        \end{tabular}\end{center}
        \raggedright\source{\ownwork}
        \vspace{0.75cm}
    \end{table}
\end{minipage}

\noindent\begin{minipage}{\textwidth}
    \begin{table}[H]
        \raggedright\caption{Użytkownicy\label{tabela:uzytkownicy}}
        \begin{center}\begin{tabular}{|P{.15\textwidth}|P{.25\textwidth}|P{.5\textwidth}|}

            \hline
            \cellgray{Nazwa} & \cellgray{Opis} & \cellgray{Odpowiedzialności}\\

            \hline
            Gość &
            Niezalogowany użytkownik &
            \begin{itemize}
                \item Zakłada konto użytkownika.
                \item Wyświetla stronę główną.
            \end{itemize} \\
            \hline
            Administrator &
            Osoba zarządzająca działaniem aplikacji &
            \begin{itemize}
                \item Przydzielanie i~odbieranie użytkownikom uprawnień.
                \item Zarządzanie definicjami wartości odżywczych, typami diet, typami posiłków, typami dań i~wyposażeniem kuchennym.
            \end{itemize} \\
            \hline
        \end{tabular}\end{center}
        \raggedright\source{\ownwork}
        \vspace{0.75cm}
    \end{table}
\end{minipage}

\thispagestyle{normal}
